\theoremstyle{plain}
\newtheorem{defCoalgebra}{Definici\'{o}n}[section]
\newtheorem{defBialgebra}[defCoalgebra]{Definici\'{o}n}
\newtheorem{teoIsomorfismoProductoTensorialDeMorfismosDeModulos}[defCoalgebra]%
	{Teorema}
\newtheorem{coroIsomorfismoProductoTensorialDeMorfismosDeModulos}%
	[defCoalgebra]{Corolario}
\newtheorem{propoAlgebraDual}[defCoalgebra]{Proposici\'{o}n}
\newtheorem{propoCoalgebraDual}[defCoalgebra]{Proposici\'{o}n}

\theoremstyle{definition}
\newtheorem{obsCoalgebra}[defCoalgebra]{Observaci\'{o}n}
\newtheorem{ejemploCoalgebra}[defCoalgebra]{Ejemplo}
\newtheorem{ejemploCoalgebraProductoTensorial}[defCoalgebra]{Ejemplo}
\newtheorem{ejemploAlgebraDual}[defCoalgebra]{Ejemplo}
\newtheorem{ejemploCoalgebraDualDeMatrices}[defCoalgebra]{Ejemplo}
\newtheorem{obsPolinomios}[defCoalgebra]{Observaci\'{o}n}
\newtheorem{obsBialgebraDeMatrices}[defCoalgebra]{Observaci\'{o}n}
\newtheorem{ejemploBialgebra}[defCoalgebra]{Ejemplo}

%-------------

\subsection{Co\'{a}lgebras}\label{subsec:coalgebras:coalgebras}

\subsubsection{Definiciones}

En la Definici\'{o}n~\ref{def:algebraalternativa}, dimos una definici\'{o}n de
$k$-\'{a}lgebra en t\'{e}rminos de ciertos diagramas. La idea en la
definici\'{o}n de $k$-co\'{a}lgebra es dar vuelta todas las flechas que
aparecen en aquellos diagramas. Adelant\'{a}ndonos a la secci\'{o}n
\ref{sec:gruposafines}, la noci\'{o}n de co\'{a}lgebra aparece naturalmente,
teniendo en cuenta que el funtor $\Homalg\big(-,-\big)$ es
\emph{contra}variante en el primer lugar: si un grupo af\'{\i}n es un funtor
representable por una $k$-\'{a}lgebra conmutativa, $G=\Homalg\big(H,-\big)$,
junto con (entre otras cosas) una transformaci\'{o}n natural
$m:\,G\times G\xrightarrow\cdot G$ --la multiplicaci\'{o}n en el grupo--,
entonces esta transformaci\'{o}n $m$ deber\'{a} estar inducida por un morfismo
correspondiente $\coproducto:\,H\rightarrow H\tensor H$ (!`en la direcci\'{o}n
opuesta!) y las propiedades de $m$ deber\'{a}n verse reflejadas en propiedades
de $\coproducto$.

\begin{defCoalgebra}\label{def:coalgebra}
	Una $k$-co\'{a}lgebra es un $k$-m\'{o}dulo $C$ junto con morfismos de
	$k$-m\'{o}dulos $\counidad[C]:\,C\rightarrow k$ --la counidad de $C$--
	y $\coproducto[C]:\,C\rightarrow C\tensor C$ --el coproducto en $C$--
	que verifican que los siguientes diagramas conmutan.
	\begin{center}
	\begin{tikzcd}[column sep=large]
		C\tensor C\tensor C & C\tensor C
			\arrow[l,"{\coproducto[C]\tensor\id[C]}"'] \\
		C\tensor C \arrow[u,"{\id[C]\tensor\coproducto[C]}"] &
		C\arrow[l,"{\coproducto[C]}"]\arrow[u,"{\coproducto[C]}"']
	\end{tikzcd}
	\begin{tikzcd}[column sep=large]
		k\tensor C & C\tensor C
			\arrow[l,"{\counidad[C]\tensor\id[C]}"']
			\arrow[r,"{\id[C]\tensor\counidad[C]}"] & C\tensor k \\
		& C \arrow[ul,"\sim"] \arrow[ur,"\sim"']
			\arrow[u,"{\coproducto[C]}"'] &
	\end{tikzcd}
	\end{center}
	Una $k$-co\'{a}lgebra es coconmutativa, si
	\begin{align*}
		\swap\circ\coproducto[C] & \,=\,\coproducto[C]
		\text{ .}
	\end{align*}
	%
	Un morfismo de $k$-co\'{a}lgebras es un morfismo de $k$-m\'{o}dulos
	$f:\,C\rightarrow D$ tal que
	\begin{align*}
		\coproducto[D]\circ f \,=\,(f\tensor f)\circ\coproducto[C]
			& \quad\text{y}\quad
		\counidad[D]\circ f \,=\,\counidad[C]
		\text{ .}
	\end{align*}
\end{defCoalgebra}

\begin{obsCoalgebra}\label{obs:coalgebraopuesta}
	Si $(C,\coproducto[C],\counidad[C])$ es una $k$-co\'{a}lgebra, su
	\emph{co\'{a}lgebra opuesta} es $(C,\coproducto[C]^\opp,\counidad[C])$,
	donde
	\begin{align*}
		\coproducto[C]^\opp & \,=\,\swap\circ\coproducto[C]
		\text{ .}
	\end{align*}
	%
	Dentamos esta co\'{a}lgebra por $C^\copp$. La co\'{a}lgebra $C$ es
	coconmutativa, si $C^\copp=C$, es decir, si el morfismo de m\'{o}dulos
	$\id[C]:\,C\rightarrow C^\copp$ es morfismo de co\'{a}lgebras.
\end{obsCoalgebra}

\subsubsection{Ejemplos}

\begin{ejemploCoalgebra}\label{ejemplo:coalgebraanillodebase}
	El $k$-m\'{o}dulo $k$ es el $k$-m\'{o}dulo libre generado por el
	elemento $1$. Las expresiones
	\begin{align*}
		\coproducto(1) \,=\,1\tensor 1 & \quad\text{y}\quad
			\counidad(1)\,=\,1
	\end{align*}
	%
	determinan un\'{\i}vocamente morfismos de $k$-m\'{o}dulos
	$\coproducto:\,k\rightarrow k\tensor k$ y $\counidad:\,k\rightarrow k$.
	Estos morfismos dan a $k$ una estructura de co\'{a}lgebra que
	denominaremos ``estructura usual de co\'{a}lgebra en $k$''. Nos
	estaremos refiriendo a esta estructura, si usamos la notaci\'{o}n
	$\coproducto[k]$ o $\counidad[k]$. Con esta estructura, dada una
	co\'{a}lgebra $(C,\coproducto[C],\counidad[C])$, el morfismo de
	m\'{o}dulos $\counidad[C]:\,C\rightarrow k$ es morfismo de
	co\'{a}lgebras.
\end{ejemploCoalgebra}

\begin{ejemploCoalgebra}\label{ejemplo:coalgebradeconjunto}
	Sea $X$ un conjunto y sea $k[X]$ el $k$-m\'{o}dulo libre con base $X$.
	Definimos morfismos de m\'{o}dulos
	$\coproducto:\,k[X]\rightarrow k[X]\tensor k[X]$ y
	$\counidad:\,k[X]\rightarrow k$ dando sus valores en los generadores:
	\begin{align*}
		\coproducto(x) \,=\, x\tensor x & \quad\text{y}\quad
			\counidad(x)\,=\,1
		\text{ .}
	\end{align*}
	%
	La terna $(k[X],\coproducto,\counidad)$ es una co\'{a}lgebra: como los
	diagramas en la Definici\'{o}n~\ref{def:coalgebra} son diagramas de
	$k$-m\'{o}dulos, basta notar que, para todo generador $x\in X$, se
	cumple
	\begin{align*}
		(\coproducto\tensor\id)\circ\coproducto(x) & \,=\,
			(x\tensor x)\tensor x \,=\,
			x\tensor (x\tensor x) \,=\,
			(\id\tensor\coproducto)\circ\coproducto(x) \text{ ,} \\
		(\counidad\tensor\id)\circ\coproducto(x) & \,=\,
			1\tensor x \,\sim\, x \quad\text {y} \\
		(\id\tensor\counidad)\circ\coproducto(x) & \,=\,
			x\tensor 1 \,\sim\,x
	\end{align*}
	%
	El Ejemplo~\ref{ejemplo:coalgebraanillodebase} es un caso particular de
	esta construcci\'{o}n.
\end{ejemploCoalgebra}

\begin{ejemploCoalgebra}\label{ejemplo:coalgebradepolinomios}
	Sea $k[x]$ el \'{a}lgebra de polinomios en una indeterminada. Sean
	$\coproducto:\,k[x]\rightarrow k[x]\tensor k[x]$ y
	$\counidad:\,k[x]\rightarrow k$ los morfismos \emph{de \'{a}lgebras}
	dados en el generador $x$ por
	\begin{align*}
		\coproducto(x) \,=\,x\tensor 1 + 1\tensor x
			& \quad\text{y}\quad
		\counidad(x) \,=\,0
		\text{ .}
	\end{align*}
	%
	En particular, $\coproducto$ y $\counidad$ definen, olvid\'{a}ndonos de
	la estructura adicional, morfismos de m\'{o}dulos cuyo dominio es el
	$k$-m\'{o}dulo libre con base en el conjunto $\{1,\,x,\,x^2,\,\dots\}$.
	Por ejemplo,
	\begin{align*}
		\coproducto(1) & \,=\,1\tensor 1 \quad\text{y} \\
		\coproducto(x^2) & \,=\,\coproducto(x)^2 \,=\,
			\big(x\tensor 1 + 1\tensor x\big)^2 \,=\,
			x^2\tensor 1 \,+\, 2\,(x\tensor x) \,+\, 1\tensor x^2
		\text{ .}
	\end{align*}
	%
	Estos morfismos satisfacen
	\begin{align*}
		(\coproducto\tensor\id)\circ\coproducto(x) & \,=\,
			(x\tensor 1)\tensor 1 + x \tensor (1\tensor 1) \,=\,
			x\tensor (1\tensor 1) + (x\tensor 1)\tensor 1 \\
		& \,=\, (\id\tensor\coproducto)\circ\coproducto(x) \text{ ,} \\
		(\counidad\tensor\id)\circ\coproducto(x) & \,=\,
			0\tensor 1 + 1\tensor x \,=\, 1\tensor x
			\quad\text{y} \\
		(\id\tensor\counidad)\circ\coproducto(x) & \,=\,
			x\tensor 1 + 1\tensor 0 \,=\, x\tensor 1
		\text{ .}
	\end{align*}
	%
	Para demostrar que los diagramas de la Definici\'{o}n~%
	\ref{def:coalgebra} conmutan y que $(k[x],\coproducto,\counidad)$
	es una co\'{a}lgebra, necesitamos, en principio, demostrar, al menos,
	que estas igualdades son v\'{a}lidas reemplazando $x$ por cualquier
	otra potencia, ya que los morfismos en la definici\'{o}n de
	co\'{a}lgebra son morfismos de m\'{o}dulos. Pero, por definici\'{o}n,
	$\coproducto$ y $\counidad$ son morfismos de \'{a}lgebras. Esto implica
	que $\id\tensor\coproducto$, $\coproducto\tensor\id$,
	$\counidad\tensor\id$ e $\id\tensor\counidad$ son morfismos de
	\'{a}lgebras, como as\'{\i} tambi\'{e}n las identificaciones
	$k\tensor k[x]\simeq k[x]\simeq k[x]\tensor k$. Asumiendo que esta
	afirmaci\'{o}n es cierta (ver las Observaciones~%
	\ref{obs:productotensorialdealgebrasejemplomodulos} y
	\ref{obs:productotensorialdealgebrasesfuntorial}), basta verificar las
	igualdades
	\begin{align*}
		(\coproducto\tensor\id)\circ\coproducto & \,=\,
			(\id\tensor\coproducto)\circ\coproducto \text{ ,} \\
		(\counidad\tensor\id)\circ\coproducto & \,=\, j
			\quad\text{y} \\
		(\id\tensor\counidad)\circ\coproducto & \,=\, j
		\text{ ,}
	\end{align*}
	%
	para el generador $x$ del \'{a}lgebra $k[x]$ (aqu\'{\i},
	$j:\,k[x]\rightarrow k\tensor k[x]$ es el isomorfismo de \'{a}lgebras
	$x\mapsto 1\tensor x$ y $j:\,k[x]\rightarrow k[x]\tensor k$ es
	$x\mapsto x\tensor 1$). Pero esto ya lo hemos demostrado.
\end{ejemploCoalgebra}

\subsubsection{Producto tensorial de co\'{a}lgebras}

Dadas co\'{a}lgebras $(C,\coproducto[C],\counidad[C])$ y
$(D,\coproducto[D],\counidad[D])$, el producto tensorial de m\'{o}dulos
$C\tensor D$ admite una estructura natural de co\'{a}lgebra de manera que los
morfismos de m\'{o}dulos $p_C:\,C\tensor D\rightarrow C$ y
$p_D:\,C\tensor D\rightarrow D$ dados por
\begin{align*}
	p_C(c\tensor d) \,=\,c\,\counidad[D](d)
		& \quad\text{y}\quad
	p_D(c\tensor d) \,=\,\counidad[C](c)\,d
\end{align*}
%
sean morfismos de co\'{a}lgebras. Definimos
\begin{equation}
	\label{eq:productotensorialdecoalgebras}
	\coproducto[C\tensor D] \,=\,
		(\id[C]\tensor\swap[C\tensor D]\tensor\id[D])\circ
		(\coproducto[C]\tensor\coproducto[D])
		\quad\text{y}\quad
	\counidad[C\tensor D] \,=\,\counidad[C]\tensor\counidad[D]
	\text{ .}
\end{equation}
%
En cuanto a la counidad, estamos identificando $k\tensor k\simeq k$ (como
$k$-m\'{o}dulos):
\begin{align*}
	\counidad[C\tensor D](c\tensor d) & \,=\,
		(\counidad[C]\tensor\counidad[D])(c\tensor d) \,=\,
		\counidad[C](c)\,\counidad[D](d)
\end{align*}
%
(es decir,
\begin{math}
	\counidad[C\tensor D]=
		\producto[k]\circ(\counidad[C]\tensor\counidad[D])
\end{math}). Veamos que, con estas definiciones, $p_C$ y $p_D$ son morfismos de
co\'{a}lgebras: por un lado,
\begin{align*}
	\coproducto[C]\circ p_C(c\tensor d) & \,=\,
		\coproducto[C](c\,\counidad[D](d)) \,=\,
		\coproducto[C](c)\,\counidad[D](d)
\end{align*}
%
y, por otro, si escribimos $\coproducto[C](c)=\sum_{(c)}\,c'\tensor c''$ y
$\coproducto[D](d)=\sum_{(d)}\,d'\tensor d''$, 
\begin{align*}
	& (p_C\tensor p_C)\circ\coproducto[C\tensor D] (c\tensor d) \,=\,
		(p_C\tensor p_C)\Big(\sum_{(c)\,(d)}\,
			c'\tensor d'\tensor c''\tensor d''\Big) \\
	& \qquad\qquad\,=\,
		\sum_{(c)\,(d)}\,c'\,\counidad[D](d')\tensor
			c''\,\counidad[D](d'') \,=\,
		\Big(\sum_{(c)}\,c'\tensor c''\Big)\,\cdot\,
			\counidad[D]\Big(\sum_{(d)}\,d'\counidad[D](d'')
			\Big) \\
	& \qquad\qquad\,=\,
		\coproducto[C](c)\,\counidad[D](d)
	\text{ .}
\end{align*}
%
Entonces $p_C$ respeta coproductos; en cuanto a las counidades,
\begin{align*}
	\counidad[C]\circ p_C(c\tensor d) & \,=\,
		\counidad[C](c\,\counidad[D](d)) \,=\,
		\counidad[C](c)\,\counidad[D](d) \,=\,
		\counidad[C\tensor D](c\tensor d)
	\text{ .}
\end{align*}
%
La verificaci\'{o}n para $p_D$ es an\'{a}loga.

\begin{ejemploCoalgebra}\label{ejemplo:coalgebradeconjuntoproducto}
	Dados conjuntos $X$ e $Y$, el isomorfismo de m\'{o}dulos
	\begin{align*}
		k[X\times Y] & \,\simeq\,k[X]\tensor k[Y]
		\text{ ,}
	\end{align*}
	%
	dado por $(x,y)\mapsto x\tensor y$, es un morfismo de co\'{a}lgebras:
	\begin{align*}
		\coproducto[{k[X\times Y]}](x,y) & \,=\,(x,y)\tensor (x,y)
			\,\mapsto\,(x\tensor y)\tensor (x\tensor y)
			\text{ ,} \\
		\coproducto[{k[X]\tensor k[Y]}](x\tensor y) & \,=\,
			(\id\tensor\swap\tensor\id)
				((x\tensor x)\tensor (y\tensor y))
		\text{ ;}
	\end{align*}
	%
	en cuanto a la counidad,
	\begin{align*}
		\counidad[{k[X\times Y]}](x,y) & \,=\, 1 \,=\,1\cdot 1\,=\,
			(\counidad[{k[X]}]\tensor\counidad[{k[Y]}])(x\tensor y)
		\text{ .}
	\end{align*}
	%
\end{ejemploCoalgebra}

\subsubsection{Dualidad}\label{subsubsec:coalgebras:dualidad}

Sean $U$, $U'$, $V$ y $V'$ cuatro $k$-m\'{o}dulos. Sean $f:\,U\rightarrow U'$ y
$g:\,V\rightarrow V'$ morfismos de m\'{o}dulos. Recordemos que podemos definir
el producto tensorial $f\tensor g:\,U\tensor V\rightarrow U'\tensor V'$ como el
morfismo determinado por
\begin{align*}
	(f\tensor g)(u\tensor v) & \,=\,f(u)\tensor g(v)
	\text{ ,}
\end{align*}
%
en tensores elementales. Esto determina un\'{\i}vocamente un morfismo
\begin{equation}
	\label{eq:productotensorialdemorfismosdemodulos}
	\lambda \,:\,\Hom[k]\big(U,U'\big)\,\tensor\,\Hom[k]\big(V,V'\big)
		\,\rightarrow\,\Hom[k]\big(V\tensor U,U'\tensor V'\big)
	\text{ ,}
\end{equation}
%
de manera que $\lambda(f\tensor g)$ sea el morfismo dado por
$v\tensor u\mapsto f(u)\tensor g(v)$. Notamos que este morfismo incorpora un
intercambio en el orden de $U$ y de $V$. Notamos, tambi\'{e}n, que los grupos
abelianos $\Hom[k]\big(-,-\big)$ son $k$-m\'{o}dulos y el morfismo $\lambda$ es
$k$-lineal.

\begin{teoIsomorfismoProductoTensorialDeMorfismosDeModulos}%
	\label{thm:isomorfismoproductotensorialdemorfismosdemodulos}
	Si, en \eqref{eq:productotensorialdemorfismosdemodulos}, alguno de los
	pares $(U,U')$, $(V,V')$ o $(U,V)$ est\'{a} compuesto por
	$k$-m\'{o}dulos libres f.g., entonces $\lambda$ es un isomorfismo.
\end{teoIsomorfismoProductoTensorialDeMorfismosDeModulos}

\begin{proof}
	Usar que $\prod_i=\bigoplus_i$, si el conjunto de \'{\i}ndices es
	finito y que conmutan con $\Hom[k]$ y con $\tensor$.
\end{proof}

\begin{coroIsomorfismoProductoTensorialDeMorfismosDeModulos}%
	\label{coro:isomorfismoproductotensorialdemorfismosdemodulos}
	\begin{enumerate}
		\item El morfismo de m\'{o}dulos
			\begin{math}
				\lambda:\,\dual U\tensor\dual V\rightarrow
					\dual{(V\tensor U)}
			\end{math} es un isomorfismo, si $U$ o si $V$ son
			libres f.g.
		\item El morfismo de m\'{o}dulos
			\begin{math}
				\lambda:\,V\tensor\dual U\rightarrow
					\Hom[k]\big(U,V\big)
			\end{math} es un isomorfismo, si $U$ o si $V$ son
			libres f.g.
	\end{enumerate}
\end{coroIsomorfismoProductoTensorialDeMorfismosDeModulos}

En el primer caso, $\lambda$ est\'{a} dado por
\begin{align*}
	\lambda(f\tensor g) & \,=\,\big(v\tensor u\mapsto f(u)\,g(v)\big)
\end{align*}
%
y, en el segundo caso, por
\begin{align*}
	\lambda(v\tensor f) & \,=\,\big(u\mapsto f(u)\,v\big)
	\text{ .}
\end{align*}
%

\begin{propoAlgebraDual}\label{propo:algebradual}
	Sea $(C,\coproducto,\counidad)$ una co\'{a}lgebra. Entonces, la terna
	$(\dual C,\dual\coproducto\circ(\lambda\circ\swap),\dual\counidad)$,
	donde $\lambda:\,\dual C\tensor\dual C\rightarrow\dual{(C\tensor C)}$
	es el morfismo \eqref{eq:productotensorialdemorfismosdemodulos}, es un
	\'{a}lgebra.
\end{propoAlgebraDual}

\begin{proof}
	El primer diagrama de la Definici\'{o}n~\ref{def:coalgebra} induce
	\begin{center}
		\begin{tikzcd}[column sep=large]
			\dual C\tensor\dual C\tensor\dual C
				\arrow[r,"{\lambda\circ\swap\tensor\id}"]
				\arrow[d,"{\id\tensor\lambda\circ\swap}"'] &
			\dual{(C\tensor C)}\tensor\dual C
				\arrow[r,dashed,
					"{\dual\coproducto\tensor\dual\id}"]
				\arrow[d,"{\lambda\circ\swap}"'] &
			\dual C\tensor\dual C
				\arrow[d,"{\lambda\circ\swap}"] \\
			\dual C\tensor\dual{(C\tensor C)}
				\arrow[r,"{\lambda\circ\swap}"]
				\arrow[d,dashed,
					"{\dual\id\tensor\dual\coproducto}"'] &
			\dual{\big(C\tensor C\tensor C\big)}
				\arrow[r,"\dual{(\coproducto\tensor\id)}"]
				\arrow[d,"\dual{(\id\tensor\coproducto)}"'] &
			\dual{\big(C\tensor C\big)}
				\arrow[d,"\dual\coproducto"] \\
			\dual C\tensor\dual C
				\arrow[r,"{\lambda\circ\swap}"'] &
			\dual{\big(C\tensor C\big)}
				\arrow[r,"\dual\coproducto"'] &
			\dual C
		\end{tikzcd}
	\end{center}
	Es decir, $\dual\coproducto\circ(\lambda\circ\swap)$ es asociativo. De
	manera an\'{a}loga, mediante dualizando el diagrama de la counidad, se
	deduce que $\dual\counidad$ es una unidad para este producto.
\end{proof}

\begin{propoCoalgebraDual}\label{propo:coalgebradual}
	Sea $(A,\producto,\unidad)$ un \'{a}lgebra. Si $A$ es libre y f.g. como
	$k$-m\'{o}dulo, entonces la terna
	$(\dual A,(\lambda\circ\swap)^{-1}\circ\dual\producto,\dual\unidad)$ es
	una co\'{a}lgebra.
\end{propoCoalgebraDual}

\begin{proof}
	La demostraci\'{o}n es dual a la de la Proposici\'{o}n~%
	\ref{propo:algebradual}. Para poder definir el coproducto, necesitamos
	que $\lambda$ sea un isomorfismo, lo cual es cierto, bajo las
	hip\'{o}tesis del enunciado.
\end{proof}

\subsubsection{Ejemplos}\label{subsubsec:coalgebras:dualidad:ejemplos}

\begin{ejemploAlgebraDual}\label{ejemplo:algebradual}
	Sea $C$ la co\'{a}lgebra de un conjunto $X$. El \'{a}lgebra dual
	$\dual C$ se identifica con el \'{a}lgebra de funciones
	$X\rightarrow k$. Toda funcional $f:\,C\rightarrow k$ est\'{a}
	determinada por sus valores en $X$. Dadas $f,f_1\in\dual C$, el
	producto $\producto(f\tensor f_1)$ es la funcional determinada por
	\begin{align*}
		\producto(f\tensor f_1) (x) & \,=\,
			(\lambda\circ\swap)(f\tensor f_1)(\coproducto(x)) \,=\,
			f(x)\,f_1(x)
		\text{ .}
	\end{align*}
	%
	(El orden del producto del lado derecho, si bien superfluo, es el
	correcto). En cuanto a la unidad, el isomorfismo $\dual k\simeq k$
	est\'{a} dado por identificar $\id[k]$ con $1$. Entonces la unidad de
	$\dual C$ es la funci\'{o}n
	\begin{align*}
		\unidad(1)(x) & \,=\,\id[k](\counidad(x))\,=\,1
		\text{ .}
	\end{align*}
	%
\end{ejemploAlgebraDual}

\begin{ejemploCoalgebraDualDeMatrices}\label{ejemplo:coalgebradualdematrices}
	Sea $A=\MM[n\times n](k)$ el \'{a}lgebra de matrices cuadradas con
	coeficientes en el anillo conmutativo $k$, con el producto e identidad
	usuales. Como $k$-m\'{o}dulo, $A$ es libre con base las matrices
	$E^{ij}$. Sea $\{x_{ij}\}_{i,j}$ la base dual en $\dual A$. La
	estructura de co\'{a}lgebra dual est\'{a} dada por los morfismos
	\begin{align*}
		\coproducto(x_{ij}) \,=\,\sum_{k=1}^n\,x_{ik}\tensor x_{kj}
			& \quad\text{y}\quad
		\counidad(x_{ij}) \,=\,\delta_{ij}
		\text{ .}
	\end{align*}
	%
	Notamos que
	\begin{math}
		\dual\producto(x_{ij})(E^{pq}\tensor E^{rs})=
			x_{ij}(E^{pq}\,E^{rs})=\delta^{qr}\,x_{ij}(E^{ps})=
			\delta^{qr}\delta_{ip}\delta_{js}
	\end{math} y que esto coincide con
	\begin{align*}
		\lambda\circ\swap\Big(\sum_{k=1}^n\,x_{ik}\tensor x_{kj}\Big)
			(E^{pq}\tensor E^{rs}) & \,=\,
			\sum_{k=1}^n\,x_{ik}(E^{pq})\,x_{kj}(E^{rs})
		\text{ .}
	\end{align*}
	%
	En cuanto a la counidad, $\counidad(x_{ij})\in \dual k\simeq k$ y el
	isomorfismo est\'{a} dado por identificar $f\in\dual k$ con $f(1)$.
	Entonces, como $\unidad(1)=I$, la matriz identidad,
	\begin{align*}
		\counidad(x_{ij})(1) & \,=\,x_{ij}(I) \,=\,\delta_{ij}
		\text{ .}
	\end{align*}
	%
\end{ejemploCoalgebraDualDeMatrices}

\subsection{Bi\'{a}lgebras}\label{subsec:coalgebras:bialgebras}

El $k$-m\'{o}dulo $k$ tiene estructura de \'{a}lgebra, dada por el producto y
la unidad del anillo $k$, y  estructura de co\'{a}lgebra como en el Ejemplo~%
\ref{ejemplo:coalgebraanillodebase}. Denotamos, por el momento, estas
estructuras por $(k,\producto,\unidad)$ y por $(k,\coproducto,\counidad)$,
respectivamente. Entonces podemos verificar que se cumplen las siguientes
igualdades:
\begin{align*}
	(\producto\tensor\producto)\circ(\id\tensor\swap\tensor\id)\circ
		(\coproducto\tensor\coproducto)(1\tensor 1) & \,=\,
		1\tensor 1 \,=\,
		\coproducto\circ\producto(1\tensor 1) \text{ ,} \\
	\producto\circ(\counidad\tensor\counidad) (1\tensor 1) & \,=\,
		1 \,=\,\counidad\circ\producto(1\tensor 1) \text{ ,} \\
	(\unidad\tensor\unidad)\circ\coproducto(1) & \,=\, 1\tensor 1\,=\,
		\coproducto\circ\unidad(1) \text{ ,} \\
	\counidad\circ\unidad(1) \,=\,1\,=\,\counidad(1) & \quad\text{y}\quad
	\unidad\circ\counidad(1) \,=\,1\,=\,\unidad(1)
	\text{ .}
\end{align*}
%
Las primeras dos ecuaciones implican que $\producto:\,k\tensor k\rightarrow k$
es morfismo de co\'{a}lgebras. An\'{a}logamente, la primera y la tercera
implican que $\coproducto:\,k\rightarrow k\tensor k$ es morfismo de
\'{a}lgebras. Tambi\'{e}n se comprueba que $\unidad$ es morfismo de
co\'{a}lgebras y que $\counidad$ es morfismo de \'{a}lgebras. Es decir,
$(k,\producto,\unidad)$ y $(k,\coproducto,\counidad)$ son compatibles y
forman lo que se llama una bi\'{a}lgebra.

\subsubsection{Definiciones}

\begin{defBialgebra}\label{def:bialgebra}
	Sea $B$ un $k$-m\'{o}dulo con estructuras de \'{a}lgebra
	$(B,\producto,\unidad)$ y de co\'{a}lgebra $(B,\coproducto,\counidad)$.
	Entonces $B$ se dice \emph{$k$-bi\'{a}lgebra}, si se cumple cualquiera
	de las dos condiciones equivalentes siguientes:
	\begin{itemize}
		\item $\producto$ y $\unidad$ son morfismos de co\'{a}lgebras;
		\item $\coproducto$ y $\counidad$ son morfismos de
			\'{a}lgebras.
	\end{itemize}
	%
	Un morfismo de bi\'{a}lgebras es un morfismo de m\'{o}dulos que es
	morfismo de \'{a}lgebras y co\'{a}lgebras.
\end{defBialgebra}

Si $(B,\producto,\unidad,\coproducto,\counidad)$ es una bi\'{a}lgebra,
\begin{align*}
	B^\opp & \,:=\,(B,\producto^\opp,\unidad,\coproducto,\counidad)
		\text{ ,} \\
	B^\copp & \,:=\,(B,\producto,\unidad,\coproducto^\opp,\counidad)
		\quad\text{y} \\
	B^{\opp\,\copp} & \,:=\,
		(B,\producto^\opp,\unidad,\coproducto^\opp,\counidad)
\end{align*}
%
son bi\'{a}lgebras. Por ejemplo, si queremos ver que $B^\opp$ es bi\'{a}lgebra,
tenemos que verificar que $\producto^\opp:\,B\tensor B\rightarrow B$ es
morfismo de co\'{a}lgebras, es decir, que se cumple
\begin{align*}
	(\producto^\opp\tensor\producto^\opp)\circ\coproducto[\tensor] & \,=\,
		\coproducto\circ\producto^\opp \quad\text{y} \\
	\counidad\circ\producto^\opp & \,=\,\counidad[\tensor]
	\text{ ,}
\end{align*}
%
donde
\begin{math}
	\coproducto[\tensor]=(\id\tensor\swap\tensor\id)\circ
		(\coproducto\tensor\coproducto)
\end{math}. El lado derecho de la primera igualdad es igual a
\begin{align*}
	\coproducto\circ\producto\circ\swap & \,=\,
		(\producto\tensor\producto)\circ\coproducto[\tensor]\circ\swap
	\text{ ,}
\end{align*}
%
porque $\producto$ es morfismo de co\'{a}lgebras, mientras que el lado
izquierdo es igual a
\begin{align*}
	(\producto\tensor\producto)\circ(\swap\tensor\swap)\circ
		\coproducto[\tensor]
	\text{ .}
\end{align*}
%
Ser\'{a} suficiente demostrar que
\begin{align*}
	(\swap\tensor\swap)\circ(\id\tensor\swap\tensor\id)\circ
		(\coproducto\tensor\coproducto) & \,=\,
	(\id\tensor\swap\tensor\id)\circ(\coproducto\tensor\coproducto)\circ
		\swap
	\text{ .}
\end{align*}
%
Evaluando el lado derecho en un tensor elemental $x\tensor y$, se obtiene
\begin{align*}
	& x\tensor y \,\mapsto\, y\tensor x\,\mapsto\,
		\Big(\sum_{(y)}\,y'\tensor y''\Big)\tensor
			\Big(\sum_{(x)}\,x'\tensor x''\Big)\,\mapsto\,
		\sum_{(y)\,(x)}\,y'\tensor x'\tensor y''\tensor x''
	\text{ ,}
\end{align*}
%
pero si evaluamos el lado izquierdo,
\begin{align*}
	& x\tensor y\,\mapsto\, \Big(\sum_{(x)}\,x'\tensor x''\Big)\tensor
		\Big(\sum_{(y)}\,y'\tensor y''\Big)\,\mapsto\,
		\sum_{(x)\,(y)}\,x'\tensor y'\tensor x''\tensor y''\\
	& \qquad\qquad\,\mapsto\,
		\sum_{(x)\,(y)}\,y'\tensor x'\tensor y''\tensor x''
	\text{ .}
\end{align*}
%
Para ver que $\producto^\opp$ respeta la counidad, como $k$ es conmutativo, se
comprueba que
\begin{align*}
	\counidad\circ\producto^\opp & \,=\,\counidad[\tensor]\circ\swap\,=\,
		\producto[k]\circ(\counidad\tensor\counidad)\circ
			\swap[B\tensor B] \,=\,
		\producto[k]\circ\swap[k\tensor k]\circ
			(\counidad\tensor\counidad) \\
	& \,=\, \producto[k]\circ(\counidad\tensor\counidad)\,=\,
		\counidad[\tensor]
	\text{ .}
\end{align*}
%

\subsubsection{Ejemplos}

\begin{ejemploBialgebra}\label{ejemplo:bialgebramatrices}
	Sea $\MM(m)=k[\lista[11]{x}{mm}]$ el \'{a}lgebra de polinomios en $m^2$
	variables. Si definimos
	\begin{align*}
		\coproducto(x_{ij}) \,=\,\sum_k\,x_{ik}\tensor x_{kj}
			& \quad\text{y}\quad
		\counidad(x_{ij}) \,=\,\delta_{ij}
		\text{ ,}
	\end{align*}
	%
	entonces $\coproducto$ y $\counidad$ se extienden de manera \'{u}nica
	como \emph{morfismos de \'{a}lgebras}
	$\coproducto:\,\MM(m)\rightarrow\MM(m)\tensor\MM(m)$ y
	$\counidad:\,\MM(m)\rightarrow k$. Entonces, por c\'{o}mo fueron
	definidos, $\MM(m)$, junto con $\coproducto$ y $\counidad$ es una
	bi\'{a}lgebra (lo \'{u}nico que hay que verificar es que
	$\coproducto$ y $\counidad$ dan una estructura de co\'{a}lgebra, la
	compatibilidad con el \'{a}lgebra polinomial es autom\'{a}tica).
\end{ejemploBialgebra}

\begin{ejemploBialgebra}\label{ejemplo:bialgebramonoide}
	Sea $X$ un monoide con producto $\producto:\,X\times X\rightarrow X$ y
	elemento neutro $e\in X$ y sea $k[X]$ la co\'{a}lgebra en el conjunto
	subyacente a $X$ (c.f. el Ejemplo~\ref{ejemplo:coalgebradeconjunto}).
	Le damos al $k$-m\'{o}dulo $k[X]$ (libre en $X$) una estructura de
	\'{a}lgebra con producto y unidad
	\begin{align*}
		x\tensor y\,\mapsto\,\producto(x,y) & \quad\text{y}\quad
			\unidad(1_k) \,=\,e\in X
		\text{ .}
	\end{align*}
	%
	Contamos, entonces, con estructuras de \'{a}lgebra y de co\'{a}lgebra
	en $k[X]$. Pero, como $\producto(x\tensor y)=\producto(x,y)\in X$, si
	$x,y\in X$, deducimos que
	\begin{align*}
		\coproducto(\producto(x,y)) & \,=\,
			\producto(x,y)\tensor\producto(x,y) \,=\,
			\producto[\tensor]((x\tensor x)\tensor (y\tensor y))
				\,=\,
			\producto[\tensor](\coproducto(x)\tensor\coproducto(y))
				\text{ ,} \\
		\counidad(\producto(x,y)) & \,=\,1\,=\,
			\producto[k](\counidad(x)\tensor\counidad(y))
				\text{ ,} \\
		\coproducto(\unidad(1)) & \,=\,\coproducto(e)\,=\,e\tensor e
			\,=\,\unidad\tensor\unidad(1) \text{ ,} \\
		\counidad\circ\unidad(1) & \,=\,\counidad(e) \,=\,1 \,=\,
			\unidad[k](1)
		\text{ ,}
	\end{align*}
	%
	de lo que se deduce que $\coproducto$ y $\counidad$ son morfismos de
	\'{a}lgebras.
\end{ejemploBialgebra}
