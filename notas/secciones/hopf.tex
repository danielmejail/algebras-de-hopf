\theoremstyle{plain}
\newtheorem{defAntipoda}{Definici\'{o}n}[section]
\newtheorem{defHopf}[defAntipoda]{Definici\'{o}n}
\newtheorem{propoProductoDeConvolucion}[defAntipoda]{Proposici\'{o}n}
\newtheorem{teoGrupoDeMorfismos}[defAntipoda]{Teorema}

\theoremstyle{definition}
\newtheorem{obsHopf}[defAntipoda]{Observaci\'{o}n}
\newtheorem{obsAntipoda}[defAntipoda]{Observaci\'{o}n}
\newtheorem{ejemploAlgebraDeGrupo}[defAntipoda]{Ejemplo}
\newtheorem{obsGeneralLinealNoEsCoconmutativa}[defAntipoda]{Observaci\'{o}n}
\newtheorem{obsElementosPrimitivos}[defAntipoda]{Observaci\'{o}n}
\newtheorem{obsElementosDeGrupo}[defAntipoda]{Obervaci\'{o}n}

%-------------

\subsection{La ant\'{\i}poda}

Sea $(A,\producto,\unidad)$ un \'{a}lgebra y $(C,\coproducto,\counidad)$ una
co\'{a}lgebra. El grupo abeliano $\Hom[k]\big(C,A\big)$ de $k$-morfismos es,
en realidad, un $k$-m\'{o}dulo y viene equipado con una aplicaci\'{o}n $k$-%
bilineal denominada \emph{producto de convoluci\'{o}n}: dados
$f,g:\,C\rightarrow A$ (de m\'{o}dulos), la \emph{convoluci\'{o}n de $f$ con %
$g$} es la composici\'{o}n
\begin{equation}
	\label{eq:convolucion}
	f\convol g \,=\,\producto\circ(f\tensor g)\circ\coproducto
	\text{ .}
\end{equation}
%
Dado que \eqref{eq:convolucion} es una composici\'{o}n de morfismos de
m\'{o}dulos, deducimos que $f\convol g\in\Hom[k]\big(C,A\big)$. Si $x\in C$,
entonces $\coproducto(x)=\sum_{(x)}\,x'\tensor x''$ y
% en t\'{e}rminos de la expansi\'{o}n de $x$,
\begin{equation}
	\label{eq:convolucionexplicita}
	f\convol g(x) \,=\,\sum_{(x)}\,f(x')\,g(x'')
	\text{ .}
\end{equation}
%

\begin{propoProductoDeConvolucion}\label{propo:productodeconvolucion}
	La terna $(\Hom[k]\big(C,A\big),\convol,\unidad[A]\circ\counidad[C])$
	es una $k$-\'{a}lgebra. Adem\'{a}s, el morfismo de $k$-m\'{o}dulos
	$\lambda:\,A\tensor\dual C\rightarrow\Hom[k]\big(C,A\big)$ es morfismo
	de $k$-\'{a}lgebras.
\end{propoProductoDeConvolucion}

En el enunciado, haciendo abuso de notaci\'{o}n, estamos denotando la unidad
del \'{a}lgebra por $\unidad[A]\circ\counidad[C]$, cuando, precisamente, es el
morfismo $1\mapsto\unidad[A]\circ\counidad[C]$. Notemos que
$\unidad[A]\circ\counidad[C]\in\Hom[k]\big(C,A\big)$ por ser composici\'{o}n de
morfismos de m\'{o}dulos, con lo que la unidad propuesta tiene el dominio y el
codominio correctos.

\begin{proof}
	En primer lugar, observamos que, tanto $f\tensor g\mapsto f\convol g$,
	como $1\mapsto\unidad[A]\,\counidad[C]$ son morfismos de m\'{o}dulos.
	Lo que se afirma en el enunciado es que la convoluci\'{o}n es, como
	operaci\'{o}n binaria, asociativa y que $\unidad[A]\,\counidad[C]$ es
	un elemento neutro (a izquierda y a derecha) para este producto.
	\begin{align*}
		f\convol (g\convol h) & \,=\,\producto[A]\,(f\tensor
			(\producto[A]\,(g\tensor h)\,\coproducto[C]))\,
				\coproducto[C] \\
		& \,=\, \producto[A]\,(\id[A]\tensor\producto[A])\,
				(f\tensor(g\tensor h))\,
			(\id[C]\tensor\coproducto[C])\,\coproducto[C] \\
		& \,=\, \producto[A]\,(\producto[A]\tensor\id[A])\,
				((f\tensor g)\tensor h)\,
			(\coproducto[C]\tensor\id[C])\,\coproducto[C] \\
		& \,=\,(f\convol g)\convol h
		\text{ .}
	\end{align*}
	%
	En cuanto a la unidad,
	\begin{align*}
		(\unidad[A]\,\counidad[C])\convol f & \,=\,\producto[A]\,
			(\unidad[A]\tensor\id[A])\,(\id[k]\tensor f)\,
			(\counidad[C]\tensor\id[C])\,\coproducto[C] \,=\,
			\id[k]\tensor f\,=\,f
		\text{ .}
	\end{align*}
	%
	An\'{a}logamente, $f\convol(\unidad[A]\,\counidad[C])=f$.

	En cuanto al morfismo
	$\lambda:\,A\tensor\dual C\rightarrow\Hom[k]\big(C,A\big)$, dados
	$a\tensor f,a_1\tensor f_1\in A\tensor\dual C$, y $x\in C$,
	\begin{align*}
		\lambda(a\tensor f)\convol\lambda(a_1\tensor f_1)(x) & \,=\,
			\sum_{(x)}\,\big(f(x')\,a\big)\,\big(f_1(x'')\,a_1\big)
			\,=\,\Big(\sum_{(x)}f(x')\,f_1(x'')\Big)\,aa_1
		\text{ ,}
	\end{align*}
	%
	que es igual a $\lambda(aa_1\tensor ff_1)(x)$, !`pues el producto en
	$\dual C$ es exactamente, la convoluci\'{o}n en $\Hom[k]\big(C,k\big)$!
\end{proof}

\subsubsection{Definiciones}

En particular, si $(H,\producto,\unidad,\coproducto,\counidad)$ es una
bi\'{a}lgebra, podemos considerar la convoluci\'{o}n en $\Endo[k](H)$.

\begin{defAntipoda}\label{def:antipoda}
	Un endomorfismo $S\in\Endo[k](H)$ es una \emph{ant\'{\i}poda}, si
	\begin{equation}
		\label{eq:antipoda}
		S\convol\id \,=\,\id\convol S\,=\,\unidad\circ\counidad
		\text{ .}
	\end{equation}
\end{defAntipoda}

\begin{defHopf}\label{def:hopf}
	Un \emph{\'{a}lgebra de Hopf} es una bi\'{a}lgebra con ant\'{\i}poda y
	un morfismo de \'{a}lgebras de Hopf es un morfismo de las
	bi\'{a}lgebras correspondientes.
\end{defHopf}

\begin{obsHopf}\label{obs:hopf}
	La condici\'{o}n natural $S'\circ f=f\circ S$ para un morfismo
	$f:\,H\rightarrow H'$ es redundante.
\end{obsHopf}

\begin{obsAntipoda}\label{obs:antipoda:unicidad}
	No toda bi\'{a}lgebra admite una ant\'{\i}poda, pero, si existe, es
	\'{u}nica: si $S,S_1\in\Endo[k](H)$ son ant\'{\i}podas en una
	bi\'{a}lgebra $H$, entonces
	\begin{align*}
		S & \,=\,S\convol(\unidad\,\counidad) \,=\,
			S\convol (\id\convol S_1) \,=\,
			(S\convol\id)\convol S_1 \,=\,
			(\unidad\,\counidad)\convol S_1 \,=\, S_1
		\text{ .}
	\end{align*}
	%
\end{obsAntipoda}

\begin{obsAntipoda}\label{obs:antipoda:tensoreselementales}
	Un endomorfismo $S\in\Endo[k](H)$ es una ant\'{\i}poda, si satisface
	\begin{equation}
		\label{eq:antipoda:tensoreselementales}
		\sum_{(x)}\,x'\,S(x'') \,=\,\counidad(x)\,1 \,=\,
			\sum_{(x)}\,S(x')\,x''
		\text{ ,}
	\end{equation}
	%
	para todo $x\in H$.
\end{obsAntipoda}

\begin{obsAntipoda}\label{obs:antipoda:morfismodealgebrasopuestas}
	Sea $B$ una bi\'{a}lgebra. Toda ant\'{\i}poda $S\in\Endo[k](B)$ da
	lugar a un morfismo de bi\'{a}lgebras
	$S:\,B\rightarrow B^{\opp\,\copp}$. Rec\'{\i}procamente, si
	$S:\,B\rightarrow B^\opp$ es un morfismo \emph{de \'{a}lgebras} y $B$
	est\'{a} generada, como \'{a}lgebra, por un conjunto $X$, entonces $S$
	es ant\'{\i}poda, si verifica \eqref{eq:antipoda:tensoreselementales}
	para todo $x\in X$.
\end{obsAntipoda}

\begin{ejemploAlgebraDeGrupo}\label{ejemplo:algebradegrupo}
	Sea $G$ un grupo y sea $k[G]$ la bi\'{a}lgebra del monoide subyacente.
	Si definimos $S:\,k[G]\rightarrow k[G]$ por $S(x)=x^{-1}$ para
	$x\in G$, podemos ver, usando que $\coproducto(x)=x\tensor x$, que
	$S$ es ant\'{\i}poda:
	\begin{equation}
		\label{eq:ejemplo:algebradegrupo}
		x\,S(x) \,=\, S(x)\,x \,=\, 1 \,=\, \counidad(x)\,1
		\text{ .}
	\end{equation}
	Rec\'{\i}procamente, si $G$ es un monoide y $k[G]$ admite una
	ant\'{\i}poda $S$, entonces \eqref{eq:ejemplo:algebradegrupo} implica
	que $S(x)\in G$ y es un inverso para $x$ en $G$.
\end{ejemploAlgebraDeGrupo}

\subsubsection{Relaci\'{o}n con \S~\ref{subsec:kalgebras:ejemplos}}

\begin{propoLinealGeneral}\label{propo:bialgebralinealgeneral}
	Las \'{a}lgebras $\GL(2)$ y $\SL(2)$, junto con los morfismos
	$\coproducto$ y $\counidad$ dadas por las expresiones
	\begin{align*}
		\coproducto\,\begin{bmatrix} a & b \\ c & d \end{bmatrix} \,=\,
			\begin{bmatrix} a & b \\ c & d \end{bmatrix} \tensor
			\begin{bmatrix} a & b \\ c & d \end{bmatrix}
			& \quad\text{,}\quad
			\coproducto(t) \,=\,t\tensor t \text{ ,} \\
		\counidad\,\begin{bmatrix} a & b \\ c & d \end{bmatrix} \,=\,
			\begin{bmatrix} 1 & \\ & 1 \end{bmatrix}
			& \quad\text{,}\quad
			\counidad(t) \,=\,1
		\text{ ,}
	\end{align*}
	%
	son ejemplos de bi\'{a}lgebras conmutativas.
\end{propoLinealGeneral}

% \begin{proof}
	% Basta con verificar las condiciones de coasociatividad y counidad en
	% los generadores:
	% \begin{align*}
		% \bigg(\begin{bmatrix} a & b \\ c & d \end{bmatrix}\tensor
			% \begin{bmatrix} a & b \\ c & d \end{bmatrix}\bigg)
			% \tensor
			% \begin{bmatrix} a & b \\ c & d \end{bmatrix} & \,=\,
			% \begin{bmatrix} a & b \\ c & d \end{bmatrix} \tensor
			% \bigg(\begin{bmatrix} a & b \\ c & d \end{bmatrix}
				% \tensor
			% \begin{bmatrix} a & b \\ c & d \end{bmatrix}\bigg) \\
		% (t\tensor t) \tensor t & \,=\, t\tensor (t\tensor t) \\
		% \begin{bmatrix} a & b \\ c & d \end{bmatrix}\,
			% \begin{bmatrix} 1 & \\ & 1 \end{bmatrix} & \,=\,
		% \begin{bmatrix} 1 & \\ & 1 \end{bmatrix}\,
			% \begin{bmatrix} a & b \\ c & d \end{bmatrix} \\
			% t\,1 & \,=\,1\,t
		% \text{ .}
	% \end{align*}
	% %
% \end{proof}

\begin{propoLinealGeneral}\label{propo:hopflinealgeneral}
	Las bi\'{a}lgebras $\GL(2)$ y $\SL(2)$, junto con el morfismo $S$
	definido por
	\begin{align*}
		S\,\begin{bmatrix} a & b \\ c & d \end{bmatrix} \,=\,
			(ad-bc)^{-1}\,
				\begin{bmatrix} d & -b \\ -c & a \end{bmatrix}
			& \quad\text{y}\quad
		S(t) \,=\,t^{-1}\,=\,ad-bc
			\quad\text{(\phantom)}=1\text{\phantom(),}
	\end{align*}
	%
	son ejemplos de \'{a}lgebras de Hopf conmutativas.
\end{propoLinealGeneral}

% \begin{proof}
	% \begin{align*}
		% \begin{bmatrix} a & b \\ c & d \end{bmatrix}\,
			% S\,\begin{bmatrix} a & b \\ c & d \end{bmatrix}
		% & \,=\,
		% \counidad\,\begin{bmatrix} a & b \\ c & d \end{bmatrix}\,1
			% \,=\,
		% S\,\begin{bmatrix} a & b \\ c & d \end{bmatrix}\,
			% \begin{bmatrix} a & b \\ c & d \end{bmatrix} \\
		% t\,S(t) & \,=\,\counidad(t)\,1\,=\,S(t)\,t
		% \text{ .}
	% \end{align*}
	% %
% \end{proof}

La conmutatividad de $\GL(2)$ y $\SL(2)$ implica que $S$ es una
\emph{involuci\'{o}n}: $S^2=\id$.

\begin{obsGeneralLinealNoEsCoconmutativa}%
	\label{obs:generallinealnoescoconmutativa}
	Las \'{a}lgebras de Hopf $\GL(2)$ y $\SL(2)$ no son coconmutativas:
	\begin{align*}
		\coproducto(a) & \,=\,a\tensor a+b\tensor c \,\not=\,
			a\tensor a+c\tensor b\,=\,\swap\circ\coproducto(a)
		\text{ .}
	\end{align*}
	%
\end{obsGeneralLinealNoEsCoconmutativa}

\subsection{Observaciones}\label{subsec:hopf:observaciones}

\begin{obsElementosPrimitivos}\label{obs:elementosprimitivos}
	Dada una co\'{a}lgebra $(C,\coproducto,\counidad)$, decimos que
	$x\in C$ es \emph{primitivo}, si
	\begin{equation}
		\label{eq:elementosprimitivos}
		\coproducto(x) \,=\,x\tensor 1+1\tensor x
		\text{ .}
	\end{equation}
	%
	El conjunto $\Prim C$ de elementos primitivos es un $k$-subm\'{o}dulo.
	Definimos el \emph{conmutador} de $x,y\in C$ como
	\begin{equation}
		\label{eq:conmutador}
		\conmutador{x,y} \,=\,\producto(x\tensor y)-
			\producto(y\tensor x) \,=\,x\cdot y-y\cdot x
		\text{ .}
	\end{equation}
	%
	Si ahora $B$ es bi\'{a}lgebra, entonces, para todo $x\in\Prim B$, se
	verifica que
	\begin{align*}
		x & \,=\,\counidad(x)\,1+x\,\counidad(1) \,=\,
			\counidad(x)\,1+x
		\text{ ,}
	\end{align*}
	%
	pues $\counidad(1)=1$ ($\counidad:\,B\rightarrow k$ es morfismo de las
	estructuras de \'{a}lgebras) y, por lo tanto,
	\begin{align*}
		\counidad(x) & \,=\,0
		\text{ .}
	\end{align*}
	%
	Por otro lado, la igualdad
	$\coproducto(x\cdot y)=\coproducto(x)\,\coproducto(y)$ ($\coproducto$
	es morfismo) implica que
	\begin{align*}
		\coproducto\conmutador{x,y} & \,=\,
			\conmutador{\coproducto(x),\coproducto(y)}
		\text{ .}
	\end{align*}
	%
	En particular, para $x,y\in\Prim B$,
	\begin{align*}
		\coproducto(x\cdot y) & \,=\,1\tensor(x\cdot y)+
			(x\cdot y)\tensor 1 + x\tensor y + y\tensor x
			\quad\text{y} \\
		\coproducto\,\conmutador{x,y} & \,=\,
			\conmutador{x,y}\tensor 1+1\tensor\conmutador{x,y}
		\text{ .}
	\end{align*}
	%
	En definitiva, el subm\'{o}dulo $\Prim B$ es cerrado por el
	``corchete'' $\conmutador{x,y}$. Toda bi\'{a}lgebra tiene asociada, de
	esta manera, una $k$-\'{a}lgebra de Lie.
\end{obsElementosPrimitivos}

\begin{obsElementosDeGrupo}\label{obs:elementosdegrupo}
	Dada una co\'{a}lgebra $(C,\coproducto,\counidad)$, decimos que
	$x\in C$ es \emph{grouplike} o \emph{de grupo}, si $x\not=0$ y
	$\coproducto(x)=x\tensor x$. Denotamos por $\grouplike C$ el conjunto
	de elementos de grupo de $C$. Si $B$ es bi\'{a}lgebra, entonces
	$1\in\grouplike B\not=\varnothing$ y, m\'{a}s aun, el conjunto de
	elementos de grupo es un monoide con el producto y la unidad de $B$. Y,
	si $H$ es un \'{a}lgebra de Hopf, entonces todo elemento
	$x\in\grouplike H$ posee inverso en $\grouplike H$. En primer lugar,
	notemos que, por la Observaci\'{o}n~%
	\ref{obs:antipoda:morfismodealgebrasopuestas}, la ant\'{\i}poda
	$S$ cumple que
	\begin{align*}
		(S\tensor S)\circ\coproducto & \,=\,\coproducto^\opp\circ S
			\,=\, \swap\circ\coproducto\circ S
		\text{ .}
	\end{align*}
	%
	Si, adem\'{a}s, $x\in\grouplike H$, entonces
	\begin{align*}
		\coproducto(S(x)) & \,=\,
			\swap\circ (S\tensor S)\circ\coproducto(x) \,=\,
			\swap (S(x)\tensor S(x))\,=\,S(x)\tensor S(x)
		\text{ .}
	\end{align*}
	%
	($\swap^2=\id$). Es decir, $S(x)$ es un elemento de grupo, tambi\'{e}n.
	La condici\'{o}n de ser de grupo implica, adem\'{a}s, que
	\begin{align*}
		x\,S(x) & \,=\,\producto(x\tensor S(x))\,=\,
			\producto\circ(\id\tensor S)\circ\coproducto(x)\,=\,
			\counidad(x)\,1
		\text{ .}
	\end{align*}
	%
	An\'{a}logamente, $S(x)\,x=\counidad(x)\,1$. Pero la igualdad
	\begin{align*}
		x & \,=\,\producto\circ(\counidad\tensor\id)\circ\coproducto(x)
			\,=\,\counidad(x)\,x
	\end{align*}
	%
	implica que $\counidad(x)$ es idempotente. Bajo la hip\'{o}tesis de que
	$k$ no posea idempotentes distintos de $0$ y $1$, se cumple que
	$\counidad(x)=1$ para todo $x\in\grouplike H$ y $S(x)\in\grouplike H$
	es un inverso.
	% !`Las \'{a}lgebras de Hopf son, verdaderamente, objetos
	% geom\'{e}tricos!
	% $\grouplike{k[G]}=G$.
\end{obsElementosDeGrupo}

\subsection{El grupo $\Homalg\big(H,A\big)$}\label{subsec:hopf:elgrupo}

\begin{teoGrupoDeMorfismos}\label{thm:grupodemorfismos}
	Sean $H$ un \'{a}lgebra de Hopf y $A$ un \'{a}lgebra conmutativa.
	Entonces el conjunto $\Homalg\big(H,A\big)$ es un grupo con la
	convoluci\'{o}n heredada de $\Hom[k]\big(H,A\big)$. El inverso de
	$f:\,H\rightarrow A$ est\'{a} dado por $f\circ S$.
\end{teoGrupoDeMorfismos}

\begin{teoGrupoDeMorfismos}\label{thm:grupodecomorfismos}
	Sean $H$ un \'{a}lgebra de Hopf y $C$ una co\'{a}lgebra coconmutativa.
	Entonces el conjunto $\Homcoalg\big(C,H\big)$ es un grupo con la
	convoluci\'{o}n heredada de $\Hom[k]\big(C,H\big)$. El inverso de
	$g:\,C\rightarrow H$ est\'{a} dado por $S\circ g$.
\end{teoGrupoDeMorfismos}

\begin{proof}[Demostraci\'{o}n de \ref{thm:grupodemorfismos}]
	Sean $\psi,\phi\in\Homalg[k]\big(H,A\big)$. La convoluci\'{o}n
	$\psi\convol\phi$ es morfismo de \'{a}lgebras, si cumple que
	\begin{align*}
		(\psi\convol\phi)\circ\producto[H] & \,=\,
			\producto[A]\circ(\psi\convol\phi\tensor
				\psi\convol\phi)
		\text{ .}
	\end{align*}
	%
	Como $H$ es bi\'{a}lgebra, $\coproducto[H]:\,H\rightarrow H\tensor H$
	es morfismo de \'{a}lgebras. Entonces
	\begin{align*}
		(\psi\convol\phi)\,\producto[H] & \,\equiv\,
			\producto[A]\,(\psi\tensor\phi)\,\coproducto[H]\,
				\producto[H] \\
		& \,=\, \producto[A]\,(\psi\tensor\phi)\,
				(\producto[H]\tensor\producto[H])\,
				(\id[H]\tensor\swap[H\tensor H]\tensor\id[H])\,
				(\coproducto[H]\tensor\coproducto[H])
		\text{ .}
	\end{align*}
	%
	Como $\psi$ y $\phi$ son morfismos de \'{a}lgebras,
	\begin{align*}
		(\psi\tensor\phi)\,(\producto[H]\tensor\producto[H]) & \,=\,
			(\producto[A]\,(\psi\tensor\psi))\tensor
			(\producto[A]\,(\phi\tensor\phi)) \,=\,
			(\producto[A]\tensor\producto[A])\,
			((\psi\tensor\psi)\tensor(\phi\tensor\phi))
		\text{ .}
	\end{align*}
	%
	Entonces
	\begin{align*}
		(\psi\convol\phi)\,\producto[H] & \,=\,\producto[A]\,
			(\producto[A]\tensor\producto[A])\,
			(\psi\tensor\psi\tensor\phi\tensor\phi)\,
			(\id[H]\tensor\swap[H\tensor H]\tensor\id[H])\,
			(\coproducto[H]\tensor\coproducto[H]) \\
		& \,=\,\producto[A]\,((\psi\convol\phi)\tensor
			(\psi\convol\phi))
		\text{ .}
	\end{align*}
	%
	En cuanto a la unidad,
	\begin{align*}
		(\psi\convol\phi)\,\unidad[H] & \,=\,
			\producto[A]\,(\psi\tensor\phi)\,
			(\unidad[H]\tensor\unidad[H])\,\coproducto[k]
		\,=\,\producto[A]\,(\unidad[A]\tensor\unidad[A])\,
			\coproducto[k] \,=\,\unidad[A]
		\text{ .}
	\end{align*}
	%
	En definitiva, $\psi\convol\phi\in\Homalg[k]\big(H,A\big)$. Si
	$c=\unidad[A]\,\counidad[H]$, entonces
	$c:\,H\rightarrow k\rightarrow A$ es composici\'{o}n de morfismos de
	\'{a}lgebras y, por lo tanto $c\in\Homalg[k]\big(H,A\big)$. La
	Proposici\'{o}n~\ref{propo:productodeconvolucion} implica que
	$c\convol\psi=\psi\convol c=\psi$ para todo morfismo de \'{a}lgebras
	$\psi$. En particular, $(\Homalg[k]\big(H,A\big),\convol,c)$ es un
	monoide. Aun no hemos usado que $A$ es conmutativa, ni que $H$ posee
	ant\'{\i}poda.

	Sea $S_H$ la ant\'{\i}poda en $H$. Entonces
	\begin{align*}
		\psi\convol (\psi\,S_H) & \,=\,\producto[A]\,
			(\psi\tensor(\psi\,S_H))\,\coproducto[H] \,=\,
			\producto[A]\,(\psi\tensor\psi)\,(\id[H]\tensor S_H)\,
				\coproducto[H] \\
		& \,=\,\psi\,\producto[H]\,(\id[H]\tensor S_H)\,\coproducto[H]
			\,=\,\psi\,(\id[H]\convol S_H) \\
		& \,=\,\psi\,(\unidad[H]\,\counidad[H]) \,=\,
			\unidad[A]\,\counidad[H]\,=\,c
		\text{ .}
	\end{align*}
	%
	An\'{a}logamente, $(\psi\,S_H)\convol\psi=c$. Entonces, los morfismos
	de \'{a}lgebras poseen un inverso en el \'{a}lgebra
	$\Hom[k]\big(H,A\big)$ dado por precomponer con $S_H$. Para ver que
	este inverso es un inverso en el monoide, usamos que $A$ es
	conmutativa: como $S_H:\,H\rightarrow H^\opp$ es morfismo de
	\'{a}lgebras, deducimos que
	\begin{align*}
		(\psi\,S_H)\,\producto[H] & \,=\,\psi\,
			(\producto[H]\,\swap[H\tensor H])\,(S_H\tensor S_H)
			\,=\,\producto[A]\,(\psi\tensor\psi)\,
				\swap[H\tensor H]\,(S_H\tensor S_H) \\
		& \,=\,(\producto[A]\,\swap[A\tensor A])\,
			((\psi\,S_H)\tensor(\psi\,S_H))
		\text{ .}
	\end{align*}
	%
	Si $\producto[A]\,\swap[A\tensor A]=\producto[A]$, entonces $\psi\,S_H$
	respeta productos. Esta condici\'{o}n quiere decir que $A$ es
	conmutativa. En cuanto a la unidad,
	\begin{align*}
		(\psi\,S_H)\,\unidad[H] & \,=\,\psi\,\unidad[H]\,=\,\unidad[A]
		\text{ .}
	\end{align*}
	%
\end{proof}
