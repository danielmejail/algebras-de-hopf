\theoremstyle{plain}
\newtheorem{teoEquivalencia}{Teorema}[section]
\newtheorem{coroGrupoDeMorfismos}[teoEquivalencia]{Corolario}
\newtheorem{propoYoneda}[teoEquivalencia]{Proposici\'{o}n}
\newtheorem{teoMorfismoDeGrupos}[teoEquivalencia]{Teorema}

\theoremstyle{definition}
\newtheorem{obsGrupoDeMorfismos}[teoEquivalencia]{Observaci\'{o}n}
\newtheorem{obsProductosEnGrupos}[teoEquivalencia]{Observaci\'{o}n}
\newtheorem{obsRepresentabilidad}[teoEquivalencia]{Observaci\'{o}n}

%-------------

\subsection{Recapitulaci\'{o}n}\label{subsec:gruposafines:recap}

Una manera un poco m\'{a}s clara de hacer estas cuentas es usar diagramas:
el coproducto $\coproducto[H]:\,H\rightarrow H\tensor H$ induce
\begin{center}
	\begin{tikzcd}
		H\tensor H &
		\Homalg\big(H\tensor H,A\big)
			\arrow[d,"{\pull{\coproducto[H]}}"'] &
		\Homalg\big(H,A\big)\,\times\,
			\Homalg\big(H,A\big) \arrow[l,"\sim"'] \\
		H \arrow[u,"{\coproducto[H]}"] &
		\Homalg\big(H,A\big) &
	\end{tikzcd}
\end{center}
el isomorfismo horizontal est\'{a} dado por
$(\psi,\phi)\mapsto\producto[A]\,(\psi\tensor\phi)$ y el morfismo vertical es
precomponer con $\coproducto[H]$, mostrando que el producto es la
convoluci\'{o}n; la counidad $\counidad[H]:\,H\rightarrow k$ induce
\begin{center}
	\begin{tikzcd}
		k &
		\Homalg\big(k,A\big)\,=\,\big\{\unidad[A]\big\}
			\arrow[d,"{\pull{\counidad[H]}}"'] &
		\big\{1\big\} \arrow[l,"\sim"'] \\
		H \arrow[u,"{\counidad[H]}"] &
		\Homalg\big(H,A\big) &
	\end{tikzcd}
\end{center}
el isomorfismo horizontal es $1\mapsto\unidad[A]$ y el morismo vertical es
precomponer con $\counidad[H]$, mostrando que el elemento neutro es
$\unidad[A]\,\counidad[H]$; la ant\'{\i}poda $S_H:\,H\rightarrow H$ induce
\begin{center}
	\begin{tikzcd}
		H & \Homalg\big(H,A\big) \arrow[d,"{\pull S_H}"'] \\
		H \arrow[u,"S_H"] &
		\Homalg\big(H,A\big)
	\end{tikzcd}
\end{center}
es decir, el inverso es $\pull S_H(\psi)=\psi\,S_H$. Para que $\pull S_H$
est\'{e} bien definido, se usa la conmutatividad de $A$. La estructura de
\'{a}lgebra de $H$ define la base $\Homalg\big(H,A\big)$, la estructura de
co\'{a}lgebra (bi\'{a}lgebra) da lugar al monoide
$(\Homalg\big(H,A\big),\convol,c)$ y la ant\'{\i}poda, junto con la
conmutatividad de $A$, permiten dar una noci\'{o}n de inverso. Los axiomas de
monoide/grupo son consecuencia de los axiomas de coasociatividad, counidad y de
la definici\'{o}n de ant\'{\i}poda.

Sean $G_A=\Homalg\big(H,A\big)$,
$G_A^{\otimes i}=\Homalg\big(H^{\otimes i},A\big)$ para $i\geq 1$. Entonces, la
asociatividad del producto tensorial implica que el diagrama siguiente
conmuta con todas las flechas isomorfismos:
\begin{center}
	\begin{tikzcd}
		& G\times (G\times G) \arrow[r,
			"\id\times({\producto[A]}\circ\tensor)"] &
		G\times G^{\otimes 2}
			\arrow[dr,"{\producto[A]}\circ\tensor"] & \\
		G\times G\times G \arrow[ur,"\sim"]
			\arrow[dr,"\sim"'] & & &
		G^{\otimes 3} \\
		& (G\times G)\times G\arrow[r,
			"({\producto[A]}\circ\tensor)\times\id"'] &
		G^{\otimes 2}\times G
			\arrow[ur,"{\producto[A]}\circ\tensor"'] &
	\end{tikzcd}
\end{center}
Ahora, el diagrama de coasociatividad para $\coproducto[H]$ induce un diagrama
para $G_A$ que se interpreta como la asociatividad del producto en $G_A$:
\begin{center}
	\begin{tikzcd}
		H\tensor H\tensor H &
			H\tensor H
				\arrow[l,"\id\tensor\Delta"'] \\
		H\tensor H \arrow[u,"\Delta\tensor\id"] &
			H \arrow[u,"\Delta"']
				\arrow[l,"\Delta"]
	\end{tikzcd}
	$\rightsquigarrow$
	\begin{tikzcd}
		G_A^{\otimes 3} \arrow[r,"\pull{(\id\tensor\Delta)}"]
			\arrow[d,"\pull{(\Delta\tensor\id)}"'] &
			G_A^{\otimes 2} \arrow[d,"\pull\Delta"] \\
		G_A^{\otimes 2} \arrow[r,"\pull\Delta"'] &
			G_A
	\end{tikzcd}
\end{center}
Expl\'{\i}citamente, $f\convol (g\convol h)=(f\convol g)\convol h$. De manera
similar, el isomorfismo $k\tensor H\simeq H$ induce un isomorfismo
\begin{align*}
	G_A & \,=\,\Homalg\big(H,A\big)
		\,\simeq\,\Homalg\big(k\tensor H,A\big) \\
	& \,\simeq\,\Homalg\big(k,A\big)\times\Homalg\big(H,A\big)
		\,=\,\{\unidad[A]\}\times G_A
\end{align*}
%
y, an\'{a}logamente, $H\tensor k\simeq H$ induce
$G_A\simeq G_A\times \{\unidad[A]\}$. As\'{\i},
\begin{center}
	\begin{tikzcd}
		k\tensor H &
		H\tensor H \arrow[l,"\counidad\tensor\id"']
			\arrow[r,"\id\tensor\counidad"] &
		H\tensor k \\
		& H \arrow[u,"\coproducto"'] \arrow[ul,"\sim"]
			\arrow[ur,"\sim"']
	\end{tikzcd}
	$\rightsquigarrow$
	\begin{tikzcd}
		\{\unidad[A]\}\times G_A
			\arrow[r,"\pull{(\counidad\tensor\id)}"]
			\arrow[dr, "\sim"'] &
		G_A^{\otimes 2} \arrow[d,"\pull\coproducto"] &
		G_A\times\{\unidad[A]\}
			\arrow[l,"\pull{(\id\tensor\counidad)}"']
			\arrow[dl,"\sim"] \\
		& G_A &
	\end{tikzcd}
\end{center}
Observamos, entonces que el elemento neutro de $G_A$ est\'{a} dado por
$\pull\counidad(\unidad[A])=\unidad[A]\,\counidad$. Por \'{u}ltimo, para la
ant\'{\i}poda tenemos diagramas conmutativos
\begin{center}
	\begin{tikzcd}
		H\tensor H \arrow[r,"\producto"] &
		H &
		H\tensor H \arrow[l,"\producto"'] \\
		H\tensor H \arrow[u,"S\tensor\id"] &
		H \arrow[l,"\coproducto"] \arrow[r,"\coproducto"']
			\arrow[u,"\unidad\,\counidad"'] &
		H\tensor H \arrow[u,"\id\tensor S"']
	\end{tikzcd}
	$\rightsquigarrow$
	\begin{tikzcd}
		G_A^{\otimes 2} \arrow[d,"\pull{(S\tensor\id)}"'] &
		G_A \arrow[l,"\pull\producto"']
			\arrow[d,"\pull{(\unidad\,\counidad)}"]
			\arrow[r,"\pull\producto"] &
		G_A^{\otimes 2} \arrow[d,"\pull{(\id\tensor S)}"] \\
		G_A^{\otimes 2} \arrow[r,"\pull\coproducto"'] &
		G_A &
		G_A^{\otimes 2} \arrow[l,"\pull\coproducto"]
	\end{tikzcd}
\end{center}
Si $f\in G_A$, entonces
\begin{math}
	\pull{(\unidad\,\counidad)}\,f=f\circ(\unidad\,\counidad)=
		\unidad[A]\,\counidad
\end{math}, que es el elemento neutro de $G_A$; la aplicaci\'{o}n
$\pull\producto:\,G_A\rightarrow G_A^{\otimes 2}$ est\'{a} dada por
$f\mapsto f\circ\producto=\producto[A]\circ(f\tensor f)$ y, componiendo con el
isomorfismo $G_A^{\otimes 2}\simeq G_A\times G_A$ se obtiene la diagonal
$\diag:\,f\mapsto (f,f)$; v\'{\i}a este mismo isomorfismo,
$\pull{(S\tensor\id)}$ se corresponde con $\pull S\times\id[G_A]$,
$\pull{(\id\tensor S)}$ con $\id[G_A]\times\pull S$ y $\pull\coproducto$ con
$(f,g)\mapsto f\convol g$, el producto en $G_A$. En definitiva, el diagrama
siguiente conmuta, mostrando que $f\mapsto f\circ S$ es el inverso en $G_A$:
\begin{center}
	\begin{tikzcd}
		G_A\times G_A \arrow[d,"\pull S\times\id"'] &
		G_A \arrow[l,"\diag"'] \arrow[d,"1"]
			\arrow[r,"\diag"] &
		G_A\times G_A \arrow[d,"\id\times\pull S"] \\
		G_A\times G_A \arrow[r,"\convol"'] & G_A &
		G_A\times G_A \arrow[l,"\convol"]
	\end{tikzcd}
\end{center}

\subsection{El grupo $\Homalg\big(H,-\big)$}%
	\label{subsec:gruposafines:elgrupodemorfismos}

\subsubsection{El funtor $\Homalg\big(H,-\big)$}
Para cada \'{a}lgebra de Hopf $H$ y cada \'{a}lgebra conmutativa $A$,
obtenemos un grupo en $G_A=\Homalg\big(H,A\big)$.%
\footnote{
	Si $H$ es bi\'{a}lgebra y $A$ no necesariamente es conmutativa,
	entonces se obtiene un monoide.
}
Supongamos que tenemos, adem\'{a}s, un morfismo $\varphi:\,A\rightarrow B$ de
$k$-\'{a}lgebras. Como $\Homalg\big(H,-\big):\,\Alg[k]\rightarrow\Set$ es
funtor, $\varphi$ induce una funci\'{o}n $\push\varphi:\,G_A\rightarrow G_B$,
dada por $f\mapsto \varphi\circ f$. Similarmente, se obtiene una
$\push\varphi:\,G_A^{\otimes 2}\rightarrow G_B^{\otimes 2}$. Notamos que
\begin{center}
	\begin{tikzcd}
		G_A^{\otimes 2} \arrow[d,"\push\varphi"']
			\arrow[r,"\pull{\coproducto[A]}"] &
		G_A \arrow[d,"\push\varphi"] \\
		G_B^{\otimes 2} \arrow[r,"\pull{\coproducto[B]}"'] &
		G_B
	\end{tikzcd}
\end{center}
conmuta. Aqu\'{\i} $\pull{\coproducto[A]}$ y $\pull{\coproducto[B]}$ denotan
precomposici\'{o}n con $\coproducto$ en $G_A$ y en $G_B$, respectivamente.
Entonces ambos caminos son iguales: ambos son componer a derecha con
$\coproducto$ y componer a izquierda con $\varphi$. En t\'{e}rminos de la
convoluci\'{o}n,
\begin{align*}
	\push\varphi(f\convol g) & \,=\,
		\varphi\circ\producto[A]\circ(f\tensor g)\circ\coproducto
		\,=\,\producto[B]\circ(\varphi\tensor\varphi)\circ
			(f\tensor g)\circ\coproducto
		\,=\,\push\varphi(f)\convol\push\varphi(g)
	\text{ .}
\end{align*}
%
Es decir, $\push\varphi$ es morfismo de grupos. Dicho de otra manera, el funtor
$\Homalg\big(H,-\big)$ se factoriza por la categor\'{\i}a de grupos.

\begin{coroGrupoDeMorfismos}\label{coro:grupodemorfismos}
	La aplicaci\'{o}n que a un \'{a}lgebra conmutativa $A$ le asigna el
	grupo dado por el conjunto $\Homalg\big(H,A\big)$ junto con la
	estructura definida en el Teorema~\ref{thm:grupodemorfismos} y que a un
	morfismo $\varphi:\,A\rightarrow B$ le asigna $\push\varphi$ determina
	un funtor $G:\,\CommAlg[k]\rightarrow\Grp$. Este funtor verifica
	\begin{align*}
		U\circ G & \,=\,\Homalg\big(H,-\big)
		\text{ ,}
	\end{align*}
	%
	donde $U:\,\Grp\rightarrow\Set$ denota el funtor olvido.
\end{coroGrupoDeMorfismos}

\begin{obsGrupoDeMorfismos}\label{obs:grupodemorfismos}
	Los diagramas conmutativos que expresan que $\Homalg\big(H,-\big)$ se
	factoriza por $\Grp$, es decir, que las $\push\varphi$ son morfismos de
	grupos, son los mismos diagramas que expresan la naturalidad de
	\begin{align*}
		\pull\coproducto & \,:\,\Homalg\big(H,-\big)
			\,\xrightarrow{\cdot}\,
			\Homalg\big(H\tensor H,-\big)
		\text{ .}
	\end{align*}
	%
\end{obsGrupoDeMorfismos}

\subsubsection{Grupos en $\CommAlg[k]\rightarrow\Set$}
En la categor\'{\i}a $\CommAlg[k]\rightarrow\Set$ existen productos y objetos
terminales. Denotamos por $X\times Y$ el producto de los conjuntos $X$ e $Y$ y
por $1$ el conjunto con un \'{u}nico elemento, el objeto terminal en $\Set$.
Entonces, dados $F,F':\,\CommAlg[k]\rightarrow\Set$, definimos un nuevo funtor
\begin{align*}
	(F\times F')(\varphi) & \,=\,F(\varphi)\times F'(\varphi):\,
		F(A)\times F'(A)\rightarrow F(B)\times F'(B)
\end{align*}
%
al que llamamos \emph{producto de $F$ con $F'$}; definimos, tambi\'{e}n el
funtor $\mathsf{1}$ dado por $A\mapsto 1$ en objetos y por
$\varphi\mapsto\id[1]$ en morfismos. Notemos que existe un isomorfismo natural
\begin{align*}
	\mathsf 1 & \,\simeq\,\Homalg\big(k,-\big)	
\end{align*}
%
Para cada $F:\,\CommAlg[k]\rightarrow\Set$ existe una \'{u}nica
transformaci\'{o}n natural $F\xrightarrow{\cdot}\mathsf{1}$; en cada objeto $A$
est\'{a} dada por la \'{u}nica funci\'{o}n
\begin{align*}
	t & \,:\, F(A)\,\xrightarrow{\cdot}\,\mathsf 1(A)
	\text{ .}
\end{align*}
%
Esto nos permite definir \emph{grupos en $\CommAlg[k]\rightarrow\Set$}.

\begin{obsProductosEnGrupos}\label{obs:productosengrupos}
	Tambi\'{e}n existen productos y objetos terminales en la categor\'{\i}a
	de funtores en $\Grp$, pero no los necesitaremos para definir grupos
	afines.
\end{obsProductosEnGrupos}

\begin{coroGrupoDeMorfismos}\label{coro:grupodemorfismos:grupoenlacategoria}
	Sea $G:\,\CommAlg[k]\rightarrow\Grp$ el funtor del Corolario~%
	\ref{coro:grupodemorfismos}. Los morfismos
	$\coproducto:\,H\rightarrow H\tensor H$, $\counidad:\,H\rightarrow k$ y
	$S:\,H\rightarrow H$ determinan transformaciones naturales
	\begin{align*}
		\pull\coproducto\,:\,
			UG\times UG\,\xrightarrow{\cdot}\,UG
		& \quad\text{,}\quad
		\pull\counidad\,:\,\mathsf 1\,\xrightarrow{\cdot}\,UG
		\quad\text{y}\quad
		\pull S\,:\,UG\xrightarrow{\cdot}\,UG
	\end{align*}
	%
	que cumplen
	\begin{equation}
		\label{eq:grupodemorfismos}
		\begin{aligned}
			\pull\Delta\circ(\id\times\pull\Delta) & \,=\,
				\pull\Delta\circ(\pull\Delta\times\id) \\
			\pull\Delta\circ((\pull\varepsilon\circ t)\times\id)
				\circ\diag & \,=\,\id
				\,=\, \pull\Delta\circ
					(\id\times (\pull\varepsilon\circ t))
					\circ\diag \\
			\pull\Delta\circ(\id\times\pull S)\circ\diag & \,=\,
				\pull\varepsilon\circ t
				\,=\, \pull\Delta\circ(\pull S\times\id)
					\circ\diag
		\end{aligned}
	\end{equation}
	%
	donde $\id=\id[UG]$ y $\diag:\,UG\xrightarrow\cdot UG\times UG$
	es la transformaci\'{o}n diagonal.
\end{coroGrupoDeMorfismos}

\begin{obsProductosEnGrupos}\label{obs:productosengrupos:objetoterminal}
	En realidad, sabemos un poco m\'{a}s. La t.n. $\pull\counidad$ debe
	provenir de la \'{u}nica flecha $\mathsf 1\xrightarrow\cdot G$,
	donde $\mathsf 1$ es el objeto nulo (objeto inicial y final) de la
	categor\'{\i}a.
\end{obsProductosEnGrupos}

\subsection{Equivalencia con grupos afines}%
	\label{subsec:gruposafines:equivalencia}

Sean $G,G':\,\CommAlg[k]\rightarrow\Grp$ dos funtores. Denotamos por
$U:\,\Grp\rightarrow\Set$ el funtor olvido. En este contexto nos hacemos dos
preguntas:
\begin{itemize}
	\item dada $\tilde\tau:\,UG\xrightarrow\cdot UG'$, ?`existe
		$\tau:\,G\xrightarrow\cdot G'$ tal que $U\tau=\tilde\tau$?
	\item dadas $\tau_1,\tau_2:\,G\xrightarrow\cdot G'$ tales que
		$U\tau_1=U\tau_2$, ?`vale que $\tau_1=\tau_2$?
\end{itemize}
%
En cuanto a la segunda pregunta, como $U$ es fiel, la conmutatividad del
diagrama de la izquierda implica la conmutatividad del de la derecha:
\begin{center}
	\begin{tikzcd}
		U(G(A)) \arrow[r,"U(\tau_{1A})"] \arrow[d,equal] &
		U(G'(A)) \arrow[d,equal] \\
		U(G(A)) \arrow[r,"U(\tau_{2A})"'] &
		U(G'(A))
	\end{tikzcd}
	\begin{tikzcd}
		G(A) \arrow[r,"\tau_{1A}"] \arrow[d,equal] &
		G'(A) \arrow[d,equal] \\
		G(A) \arrow[r,"\tau_{2A}"'] &
		G'(A)
	\end{tikzcd}
\end{center}
En cuanto a la primera pregunta, que $\tilde\tau$ sea igual a $U\tau$ es
equivalente, por fidelidad de $U$ a que, para cada objeto $A$,
$\tilde\tau_A:\,UG(A)\rightarrow UG'(A)$ sea morfismo de grupos.

Para cada \'{a}lgebra $A$, el objeto $G(A)$ es un grupo. En particular, existe
una \emph{funci\'{o}n} $m_A^G:\,U(G(A))\times U(G(A))\rightarrow U(G(A))$ para
la cual se verifican los axiomas de grupos. Que $\tilde\tau_A$ sea morfismo de
grupos de $G(A)$ en $G'(A)$ significa que existe un cuadrado conmutativo
\begin{center}
	\begin{tikzcd}
		UG(A)\times UG(A) \arrow[r,"m_A^G"]
			\arrow[d,"\tilde\tau_A\times\tilde\tau_A"'] &
		UG(A) \arrow[d,"\tilde\tau_A"] \\
		UG'(A) \times UG'(A) \arrow[r,"m_A^{G'}"'] &
		UG'(A)
	\end{tikzcd}
\end{center}
Pero esto querr\'{\i}a decir que, en cierto sentido, la multiplicaci\'{o}n
deber\'{\i}a ser natural en los funtores $G$ y $G'$.

Supongamos, entonces que $G$ y $G'$ son representables, en tanto existen
\'{a}lgebras de Hopf $H$ y $H'$ tales que%
\footnote{
	En cuanto a por qu\'{e} deben ser de Hopf, ver la Observaci\'{o}n~%
	\ref{obs:representabilidad}.
	}
\begin{align*}
	U\circ G \,=\,\Homalg\big(H,-\big) & \quad\text{y}\quad
	U\circ G' \,=\,\Homalg\big(H',-\big)
	\text{ .}
\end{align*}
%

\begin{propoYoneda}\label{propo:yoneda}
	La aplicaci\'{o}n $\phi\mapsto(\pull\phi:\,f\mapsto f\circ\phi)$
	determina una biyecci\'{o}n
	\begin{equation}
		\label{eq:yoneda}
		\Homalg\big(H',H\big)\,=\,U\circ G'(H) \,\simeq\,
			\Nat(U\circ G,U\circ G')
	\end{equation}
	%
\end{propoYoneda}

Los morfismos de la Proposici\'{o}n~\ref{propo:yoneda} son, \emph{a priori},
morfismos de \'{a}lgebras, no necesariamente de \'{a}lgebras de Hopf, ni de
bi\'{a}lgebras. Volviendo a las preguntas anteriores, dada
$\phi:\,H'\rightarrow H$, ?`existe una t.n. $\tau:\,G\xrightarrow\cdot G'$ tal
que $U\tau=\pull\phi$? Como ya hemos mencionado, esto significa que
$\pull\phi:\,UG(A)\rightarrow UG'(A)$ es morfismo de grupos, para cada $A$. La
estructura de grupo en estos conjuntos est\'{a} dada por el Teorema~%
\ref{thm:grupodemorfismos}. En particular, $\pull\phi$ es morfismo de grupos,
si y s\'{o}lo si el diagrama siguiente conmuta:
\begin{center}
	\begin{tikzcd}
		UG(A)\times UG(A) \arrow[r,"\pull\phi\times\pull\phi"]
			\arrow[d,"\pull{\coproducto[H]}"'] &
		UG'(A)\times UG'(A) \arrow[d,"\pull{\coproducto[H']}"] \\
		UG(A) \arrow[r,"\pull\phi"'] & UG'(A)
	\end{tikzcd}
\end{center}
o, equivalentemente, v\'{\i}a el isomorfismo
\begin{equation}
	\label{eq:isomorfismoproducto}
	\Homalg\big(H,A\big)\times\Homalg\big(H,A\big) \,\simeq\,
		\Homalg\big(H\tensor H,A\big)
\end{equation}
%
y el isomorfismo an\'{a}logo para $H'$, si y s\'{o}lo si el diagrama
\begin{center}
	\begin{tikzcd}
		\Homalg\big(H\tensor H,A\big)
			\arrow[r,"\pull{(\phi\tensor\phi)}"]
			\arrow[d,"\pull{\coproducto[H]}"'] &
		\Homalg\big(H'\tensor H',A\big)
			\arrow[d,"\pull{\coproducto[H']}"] \\
		\Homalg\big(H,A\big) \arrow[r,"\pull\phi"'] &
		\Homalg\big(H',A\big)
	\end{tikzcd}
\end{center}
conmuta. Es decir, para todo par $f,g:\,H\rightarrow A$, debe ser
\begin{align*}
	\producto[A]\,(f\tensor g)\,(\phi\tensor\phi)\,\coproducto[H'] & \,=\,
		\producto[A]\,(f\tensor g)\,\coproducto[H]\,\phi
	\text{ .}
\end{align*}
%
Necesitamos que estos diagramas conmuten \emph{para toda $A$}. Pero la
conmutatividad para toda \'{a}lgebra $A$ equivale a la conmutatividad de un
\'{u}nico diagrama: tomando $A=H$ y evaluando en el par $(\id[H],\id[H])$,
deducimos que
\begin{center}
	\begin{tikzcd}
		H\tensor H &
		H'\tensor H' \arrow[l,"\phi\tensor\phi"'] \\
		H \arrow[u,"{\coproducto[H]}"] &
		H' \arrow[u,"{\coproducto[H']}"'] \arrow[l,"\phi"]
	\end{tikzcd}
\end{center}
debe conmutar. Pero esto quiere decir, exactamente, que
$\phi:\,H'\rightarrow H$ es morfismos de co\'{a}lgebras, tambi\'{e}n.%
\footnote{
	En realidad, resta ver que respeta la counidad, pero esto se deduce
	haciendo un razonamiento an\'{a}logo con los diagramas que involucran
	$\pull{\counidad[H]}$ y $\pull{\counidad[H']}$ y el objeto terminal.
	Si los diagramas
	\begin{center}
		\begin{tikzcd}[ampersand replacement=\&]
			\mathsf 1(A) \arrow[r,equal]
				\arrow[d,"\pull{\counidad[H]}"'] \&
			\mathsf 1(A) \arrow[d,"\pull{\counidad[H']}"] \\
			UG(A) \arrow[r,"\pull\phi"'] \&
			UG'(A)
		\end{tikzcd}
	\end{center}
	conmutan para toda \'{a}lgebra $A$, en particular, conmutan para $A=H$
	y, evaluando en $\unidad[H]$ --el \'{u}nico elemento de
	$\mathsf 1(H)$--, se deduce que, para $x\in H'$,
	\begin{align*}
		\unidad[H]\,(\counidad[H]\,\phi) & \,=\,
			\pull\phi\,\pull{\counidad[H]}(\unidad[H]) \,=\,
			\pull{\counidad[H']}(\unidad[H]) \,=\,
			\unidad[H]\,\counidad[H']
		\text{ .}
	\end{align*}
	%
	Pero entonces, componiendo con $\counidad[H]$ a izquierda, podemos
	``cancelar'' y obtener $\counidad[H]\,\phi=\counidad[H']$.
}
En particular, $\phi$ es morfismo de bi\'{a}lgebras y, por lo tanto, de
\'{a}lgebras de Hopf. Rec\'{\i}procamente, si $\phi$ es de \'{a}lgebras de
Hopf, entonces el \'{u}ltimo diagrama conmuta y, aplicando el funtor
$\Homalg\big(-,A\big)$ se obtiene el ante\'{u}ltimo diagrama, lo que muestra
que, en ese caso, $\pull\phi:\,UG(A)\rightarrow UG'(A)$ es morfismo de grupos
para toda $A$.

\begin{teoMorfismoDeGrupos}\label{thm:morfismodegrupos}
	Dadas \'{a}lgebras de Hopf $H,H'$, dado un morfismo de \'{a}lgebras
	$\phi:\,H'\rightarrow H$, la transformaci\'{o}n natural
	\begin{math}
		\pull\phi:\,\Homalg\big(H,-\big)\xrightarrow\cdot
			\Homalg\big(H',-\big)
	\end{math} es morfismo de grupos, si y s\'{o}lo si $\phi$ es morfismo
	de \'{a}lgebras de Hopf.
\end{teoMorfismoDeGrupos}

\begin{teoEquivalencia}\label{thm:equivalencia}
	La aplicaci\'{o}n
	\begin{align*}
		H\,\mapsto\,\Homalg\big(H,-\big) &\quad\text{,}\quad
			\phi\,\mapsto\,\pull\phi
	\end{align*}
	%
	define un funtor contravariante fiel y pleno de la categor\'{\i}a de
	\'{a}lgebras de Hopf en la categor\'{\i}a de grupos afines,
	$(G,m,u,\sigma)$ donde
	\begin{itemize}
		\item $G:\,\CommAlg[k]\rightarrow\Grp$ es funtor,
		\item $UG$ es representable: existe $H$ tal que
			$UG\simeq\Homalg\big(H,-\big)$,
		\item $m,u,\sigma$ son transformaciones naturales que hacen de
			$UG$ un grupo en $\Set^{\CommAlg[k]}$.
	\end{itemize}
	%
\end{teoEquivalencia}

\begin{obsRepresentabilidad}\label{obs:representabilidad}
	Sea $(G,m,u,\sigma)$ un grupo af\'{\i}n y sea $R$ un \'{a}lgebra
	conmutativa que lo representa. Entonces $R$ admite una estructura de
	\'{a}lgebra ed Hopf. Por definici\'{o}n, $m$ define, componiendo con el
	isomorfismo \eqref{eq:isomorfismoproducto}, una t.n.
	\begin{align*}
		m & \,:\,\Homalg\big(R\tensor R,-\big) \,\xrightarrow\cdot\,
			\Homalg\big(R,-\big)
		\text{ .}
	\end{align*}
	%
	Por la Proposici\'{o}n~\ref{propo:yoneda}, existe un morfismo de
	\'{a}lgebras $\coproducto:\,R\rightarrow R\tensor R$ tal que
	$\pull\coproducto=m$. Expl\'{\i}citamente, siguiendo la
	demostraci\'{o}n de la Proposici\'{o}n~\ref{propo:yoneda}, definimos
	\begin{align*}
		\coproducto & \,:=\, m_{R\tensor R}(\id[R\tensor R])\,\in\,
			\Homalg\big(R,R\tensor R\big)
		\text{ .}
	\end{align*}
	%
	Para cada $k$-\'{a}lgebra $A$, existe una funci\'{o}n
	$m_A:\,\Homalg\big(R\tensor R,A\big)\rightarrow\Homalg\big(R,A\big)$.
	Dado $f:\,R\tensor R\rightarrow A$, ?`qu\'{e} morfismo es $m_A(f)$?
	?`Qu\'{e} funci\'{o}n es $m_A$? Por naturalidad,
	\begin{center}
		\begin{tikzcd}
			\Homalg\big(R\tensor R,R\tensor R\big)
				\arrow[r,"m_{R\tensor R}"]
				\arrow[d,"\push f"'] &
			\Homalg\big(R,R\tensor R\big)
				\arrow[d,"\push f"] \\
			\Homalg\big(R\tensor R,A\big) \arrow[r,"m_A"'] &
			\Homalg\big(R,A\big)
		\end{tikzcd}
	\end{center}
	conmuta y
	\begin{align*}
		f\circ\coproducto & \,=\,
			\push f\circ m_{R\tensor R}(\id[R\tensor R])
			\,=\,m_A\circ\push f(\id[R\tensor R])
			\,=\,m_A(f)
		\text{ .}
	\end{align*}
	%
	Entonces $\pull\coproducto=m$. Hay que ver que $\coproducto$ es
	coproducto, pero esto se deduce de aplicar la correspondencia de la
	Proposici\'{o}n~\ref{propo:yoneda} a las transformaciones naturales que
	aparecen en los diagramas que hacen de $m$ un producto.
\end{obsRepresentabilidad}
