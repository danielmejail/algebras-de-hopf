
\documentclass{beamer}
\mode<presentation>
{
  \usetheme{Singapore}

  \setbeamercovered{transparent}
}

\usepackage[utf8]{inputenc}
\usepackage{tikz-cd}
\usetikzlibrary{matrix}
\usetikzlibrary{decorations.pathmorphing}
\usepackage{nombres}
\usepackage{abreviaciones}

\title{\'{A}lgebras de Hopf y grupos afines}
\subtitle{}
\author{}
\institute{
	% Departamento de Matem\'{a}tica\\
	% Facultad de Ciencias Exactas y Naturales\\
	% Universidad de Buenos Aires
}
\date{}
\subject{Talks}

% \pgfdeclareimage[height=0.5cm]{university-logo}{university-logo-filename}
% \logo{\pgfuseimage{university-logo}}

\AtBeginSubsection[]
{
  \begin{frame}<beamer>{Contenidos}
    \tableofcontents[currentsection,currentsubsection]
  \end{frame}
}

% \beamerdefaultoverlayspecification{<+->}

\setbeamertemplate{itemize subitem}{$\circ$}

\newcounter{saveenumi}
\newcommand{\seti}{\setcounter{saveenumi}{\value{enumi}}}
\newcommand{\conti}{\setcounter{enumi}{\value{saveenumi}}}
\resetcounteronoverlays{saveenumi}

\begin{document}

\begin{frame}
  \titlepage
\end{frame}

\begin{frame}{Contenidos}
  \tableofcontents
  %% You might wish to add the option [pausesections]
  % Empezamos recordando la definici\'{o}n de $k$-\'{a}lgebra y algunas construcciones,
  % como el \'{a}lgebra libre, y vemos c\'{o}mo definir un grupo a partir del conjunto de
  % morfismos de $k$-\'{a}lgebras en algunos casos particulares/.
  %
  % A continuaci\'{o}n introducimos las definiciones de co\'{a}lgebra y bi\'{a}lgebra y
  % damos algunos ejemeplos.
  %
  % Seguimos, luego, con la noci\'{o}n de convoluci\'{o}n, ant\'{\i}poda y \'{a}lgebra de
  % Hopf, terminando esta parte definiendo un grupo a partir de un \'{a}lgebra de Hopf y
  % un \'{a}lgebra conmutativa.
  %
  % Por \'{u}ltimo, veremos c\'{o}mo estas estructuras describen una ley de grupo de
  % manera universal, es decir, independientemente del \'{a}lgebra conmutativa. Esto
  % permitir\'{a} asociarle a un \'{a}lgebra de Hopf (conmutativa) un esquma af\'{\i}n en
  % grupos.
  %
\end{frame}

\section{$k$-\'{a}lgebras}
\theoremstyle{plain}
\newtheorem{defAlgebra}{Definici\'{o}n}[section]
\newtheorem{propoLibre}[defAlgebra]{Proposici\'{o}n}
\newtheorem{propoRectaAfin}[defAlgebra]{Proposici\'{o}n}
\newtheorem{coroRectaAfin}[defAlgebra]{Corolario}
\newtheorem{propoMultiplicativo}[defAlgebra]{Proposici\'{o}n}
\newtheorem{coroMultiplicativo}[defAlgebra]{Corolario}
\newtheorem{propoProductoDeMatrices}[defAlgebra]{Proposici\'{o}n}
\newtheorem{propoLinealGeneral}[defAlgebra]{Proposici\'{o}n}
\newtheorem{propoProductoTensorialDeModulos}[defAlgebra]{Proposici\'{o}n}
\newtheorem{propoProductoTensorialDeAlgebras}[defAlgebra]{Proposici\'{o}n}
\newtheorem{propoProductoDeCociente}[defAlgebra]{Proposici\'{o}n}

\theoremstyle{definition}
\newtheorem{ejemploLibre}[defAlgebra]{Ejemplo}
\newtheorem{obsRectaAfin}[defAlgebra]{Observaci\'{o}n}
\newtheorem{obsProductoDeMatrices}[defAlgebra]{Observaci\'{o}n}
\newtheorem{ejemploProductoTensorialDeModulos}[defAlgebra]{Ejemplo}
\newtheorem{obsProductoTensorialDeModulos}[defAlgebra]{Observaci\'{o}n}
\newtheorem{obsAlgebra}[defAlgebra]{Observaci\'{o}n}
\newtheorem{obsProductoTensorialDeAlgebras}[defAlgebra]{Observaci\'{o}n}
\newtheorem{obsAlgebraEjemplos}[defAlgebra]{Observaci\'{o}n}

%------------

$k$ denota un anillo conmutatico (con unidad).

\subsection{Definiciones}\label{subsec:kalgebras:definiciones}

\begin{defAlgebra}\label{def:algebra}
	Una $k$-\'{a}lgebra es un anillo $A$, junto con un morfismo de anillos
	\begin{align*}
		\eta_A & \,:\,k\,\rightarrow\,A
	\end{align*}
	%
	cuya imagen est\'{a} contenida en el centro
	$\centre(A)$ de $A$.
\end{defAlgebra}

Si $A$ es una $k$-\'{a}lgebra, la aplicaci\'{o}n
$(\lambda,a)\mapsto\eta_A(\lambda)\,a$ define una estructura de $k$-m\'{o}dulo
en $A$. Con respecto a esta estructura, la multiplicaci\'{o}n
$\mu_A:\,A\times A\rightarrow A$ es una transformaci\'{o}n $k$-bilineal.

\begin{defAlgebra}\label{def:morfismodealgebras}
	Un morfismo de $k$-\'{a}lgebras es un morfismo de anillos
	$f:\,A\rightarrow B$ que es morfismo de $k$-m\'{o}dulos, es decir,
	\begin{equation}
		\label{eq:morfismodealgebras}
		f\circ\eta_A \,=\,\eta_B
	\end{equation}
	%
	Denotamos el conjunto de morfismos de $k$-\'{a}lgebras $A\rightarrow B$
	por $\Homalg\big(A,B\big)$.
\end{defAlgebra}

\subsubsection{El \'{a}lgebra libre}

Dado un conjunto $X$, llamamos \emph{palabra en $X$} a las sucesiones finitas
de elementos de $X$, $x_{i_1}\,\cdots\,x_{i_n}$ ($n\geq 1$), o bien a la
\emph{palabra vac\'{\i}a}, $\varnothing$. Denotamos por $k\{X\}$
($k\{\lista{x}{n}\}$, si $X=\{\lista{x}{n}\}$) el $k$-m\'{o}dulo libre con base
las palabras en $X$. La concatenaci\'{o}n de palabras,
\begin{align*}
	(x_{i_1}\,\cdots\,x_{i_n})\,(x_{i_{n+1}}\,\cdots\,x_{i_m}) & \,=\,
		(x_{i_1}\,\cdots\,x_{i_n}\,x_{i_{n+1}}\,\cdots\,x_{i_m})
	\text{ ,}
\end{align*}
%
define, extendiendo $k$-bilinealmente, un producto en $k\{X\}$ que lo convierte
en $k$-\'{a}lgebra. Llamamos a este \'{a}legbra, el \emph{\'{a}lgebra libre %
en $X$}.

\begin{ejemploLibre}\label{ejemplo:libre}
	Si $X=\{x\}$, $k\{x\}=k[x]$, el \'{a}lgebra de polinomios en una
	variable. Si $X=\{x,y\}$, entonces $k\{x,y\}\not=k[x,y]$, pues
	$xy\not=yx$.
\end{ejemploLibre}

El \'{a}lgebra libre est\'{a} caracterizada por la siguiente propiedad
universal:

\begin{propoLibre}\label{propo:libre}
	Dados un conjunto $X$, una $k$-\'{a}lgebra $A$ y una funci\'{o}n
	$f:\,X\rightarrow A$, existe un \'{u}nico morfismo de $k$-\'{a}lgebras
	$\tilde f:\,k\{X\}\rightarrow A$ tal que $\tilde f(x)=f(x)$ para todo
	$x\in X$.
\end{propoLibre}

Dicho de otra manera, existe una biyecci\'{o}n natural
\begin{equation}
	\label{eq:propo:libre}
	\Homalg\big(k\{X\},A\big) \,\simeq\,\Hom[\Set]\big(X,\olvido A\big)
	\text{ ,}
\end{equation}
%
donde $\olvido A$ denota el conjunto subyacente al \'{a}lgebra $A$. Por
ejemplo, como conjuntos,
\begin{align*}
	\Homalg\big(k\{x,y\},A\big) \,=\,A^2
	\text{ ,}
\end{align*}
%
v\'{\i}a $f\mapsto (f(x),f(y))$.

Toda $k$-\'{a}lgebra es cociente de un \'{a}lgebra libre. En general,
\begin{equation}
\label{eq:libre:cociente}
	\Homalg\big(k\{X\}/I,A\big) \,\simeq\,
		\Big\{f\in\Hom[\Set]\big(X,\olvido A\big)\,:\,
			\tilde f(I)=0\Big\}
	\text{ .}
\end{equation}
%

\begin{ejemploLibre}\label{ejemplo:libre:cociente}
	En $k\{x,y\}$ podemos considerar el ideal bil\'{a}tero generado por
	$xy-yx$. En este caso, se obtiene el \'{a}lgebra de polinomios,
	$k[x,y]\simeq k\{x,y\}/\generado{xy-yx}$ y vale que
	\begin{align*}
		\Homalg\big(k[x,y],A\big) & \,\simeq\,
			\big\{(a,b)\in A^2\,:\,ab=ba\big\}
		\text{ .}
	\end{align*}
	%
\end{ejemploLibre}

\subsection{Ejemplos}\label{subsec:kalgebras:ejemplos}

\subsubsection{La recta y el plano afines}

De ahora en adelante, asumiremos que $A$ es un \'{a}lgebra conmutativa. La
correspondencia \eqref{eq:libre:cociente} implica, entonces, que existe una
biyecci\'{o}n natural
\begin{equation}
	\label{eq:morfismosdesdepolinomios}
	\Homalg\big(k[\lista{x}{n}],A\big) \,\simeq\, A^n
		\qquad f\mapsto\big(f(x_1),\,\dots,\,f(x_n)\big)
	\text{ ,}
\end{equation}
%
es decir, todo morfismo de \'{a}lgebras $f:\,k[\lista{x}{n}]\rightarrow A$, en
un \'{a}lgebra conmutativa, est\'{a} determinado por los valores que toma en
los generadores $\lista{x}{n}$.

El grupo abeliano $(A,+)$ de un \'{a}lgebra est\'{a} determinado por tres
funciones: la suma, el neutro y el inverso,
\begin{align*}
	+\,:\,A\times A\,\rightarrow\,A & \quad\text{,}\quad
		0\,:\,\{0\}\,\rightarrow\,A \quad\text{y}\quad
		-\,:\,A\,\rightarrow\,A
	\text{ ,}
\end{align*}
%
que obedecen ciertas reglas ($+$ es asociativa, $0$ es un neutro para $+$ y
$-a$ es el inverso de $a$ con respecto a $+$). Queremos expresar las leyes de
grupo de manera, en alg\'{u}n sentido, universal; buscamos un objeto algebraico
independiente de toda $k$-\'{a}lgebra conmutativa $A$ que describa estas
reglas. Con este objetivo, vamos a darle una estructura de grupo (abeliano) al
conjunto $\Homalg\big(k[x],A\big)$ que sea compatible con
\eqref{eq:morfismosdesdepolinomios}.

Vamos a definir tres morfismos $\Delta:\,k[x]\rightarrow k[x',x'']$,
$\varepsilon:\,k[x]\rightarrow k$ y $S:\,k[x]\rightarrow k[x]$. \'{E}stos son
los morfismos \emph{de \'{a}lgebras} determinados por
\begin{equation}
	\label{eq:rectafin}
	\Delta(x)\,=\,x'+x'' \quad\text{,}\quad \varepsilon(x)\,=\,0
		\quad\text{y}\quad S(x)\,=\,-x
	\text{ .}
\end{equation}
%
Cada uno de ellos induce, por precomposici\'{o}n, una funci\'{o}n en morfismos:
\begin{align*}
	\pull\Delta & \,:\,\Homalg\big(k[x',x''],A\big)\,\rightarrow\,
		\Homalg\big(k[x],A\big) \text{ ,} \\
	\pull\varepsilon & \,:\,\Homalg\big(k,A\big)\,\rightarrow\,
		\Homalg\big(k[x],A\big) \quad\text{y} \\
	\pull S & \,:\,\Homalg\big(k[x],A\big)\,\rightarrow\,
		\Homalg\big(k[x],A\big)
\end{align*}
%
(la composici\'{o}n de morfismos de \'{a}lgebras es un morfismo de
\'{a}lgebras).

\begin{propoRectaAfin}\label{propo:rectaafin}
	Los siguientes diagramas conmutan.
	\begin{center}
		\begin{tikzcd}
			\Homalg\big(k[x,x],A\big) \arrow[r,"\sim"]
				\arrow[d,"\pull\Delta"'] &
				A^2 \arrow[d,"+"] \\
			\Homalg\big(k[x],A\big) \arrow[r,"\sim"] & A
		\end{tikzcd}
		\begin{tikzcd}
			\Homalg\big(k,A\big) \arrow[r,"\sim"]
				\arrow[d,"\pull\varepsilon"'] &
				\{0\} \arrow[d,"0"] \\
			\Homalg\big(k[x],A\big) \arrow[r,"\sim"] & A
		\end{tikzcd}
		\begin{tikzcd}
			\Homalg\big(k[x],A\big) \arrow[r,"\sim"]
				\arrow[d,"\pull S"'] &
				A \arrow[d,"-"] \\
			\Homalg\big(k[x],A\big) \arrow[r,"\sim"] & A
		\end{tikzcd}
	\end{center}
\end{propoRectaAfin}
Las flechas horizontales est\'{a}n dadas por
\eqref{eq:morfismosdesdepolinomios}. El conjunto $\Homalg\big(k,A\big)$ posee
un \'{u}nico elemento, $\eta_A:\,k\rightarrow A$; en este caso, la funci\'{o}n
$\Homalg\big(k,A\big)\rightarrow \{0\}$ es la identificaci\'{o}n
$\eta_A\mapsto 0$.

\begin{proof}
	Por ejemplo, si $f:\,k[x',x'']\rightarrow A$, siguiendo las flechas
	superior y derecha, obtenemos
	\begin{align*}
		f\,\mapsto\,\big(f(x'),f(x'')\big)\,\mapsto\,f(x')+f(x'')
		\text{ .}
	\end{align*}
	%
	Recorriendo el otro camino, llegamos a
	\begin{align*}
		f\,\mapsto\,\pull\Delta(f)=f\circ\Delta\,\mapsto\,
			f\circ\Delta(x)=f(x'+x'')
		\text{ .}
	\end{align*}
	%
	Pero $f$ es morfismo de \'{a}lgebras, as\'{\i} que los dos resultados
	coinciden.
\end{proof}

\begin{obsRectaAfin}\label{obs:rectaafin}
	Existe una correspondencia entre morfismos $k[x',x'']\rightarrow A$ y
	pares de morfismos $k[x]\rightarrow A$. Podemos describir esta
	correspondencia de la siguiente manera: si $f:\,k[x',x'']\rightarrow A$
	es el morfismo de \'{a}lgebras determinado por $f(x')=a'$ y
	$f(x'')=a''$, el par correspondiente a $f$ es $(f',f'')$, donde
	$f',f'':\,k[x]\rightarrow A$ son los morfismos de \'{a}lgebras
	determinados por $f'(x)=a'$ y $f''(x)=a''$. Rec\'{\i}procamente, dados
	$f',f'':\,k[x]\rightarrow A$, definimos $f:\,k[x',x'']\rightarrow A$
	por $f(x')=f'(x)$ y $f(x'')=f''(x)$. Esta relaci\'{o}n es, simplemente,
	\eqref{eq:morfismosdesdepolinomios} en los casos $n=1$ y $2$:
	\begin{equation}
		\label{eq:obs:rectaafin}
		\begin{aligned}
			\Homalg\big(k[x',x''],A\big) & \,\simeq\, A^2\,=\,
				A\,\times\,A \\
			& \,\simeq\,\Homalg\big(k[x],A\big)\,\times\,
				\Homalg\big(k[x],A\big)
			\text{ .}
		\end{aligned}
	\end{equation}
	%
\end{obsRectaAfin}

Dados morfismos de \'{a}lgebras $f,g:\,k[x]\rightarrow A$, podemos definir la
suma de $f$ con $g$, que denotamos $\pull\Delta(f,g)$, como el morfismo
determinado por
\begin{equation}
	\label{eq:rectaafin:suma}
	\pull\Delta(f,g)(x) \,=\,f(x)+g(x)
	\text{ .}
\end{equation}
%
Por \eqref{eq:obs:rectaafin}, este morfismo no es otro m\'{a}s que el que se
obtiene aplicando $\pull\Delta$ al morfismo $k[x',x'']\rightarrow A$
correspondiente al par $(f,g)$.

\begin{propoRectaAfin}\label{propo:rectaafin:suma}
	La suma \eqref{eq:rectaafin:suma} posee las siguientes propiedades:
	\begin{itemize}
		\item
			\begin{math}
				\pull\Delta(f,\pull\Delta(g,h))=
					\pull\Delta(\pull\Delta(f,g),h)
			\end{math};
		\item
			\begin{math}
				\pull\Delta(f,\pull\varepsilon(\eta_A))=
					\pull\Delta(\pull\varepsilon(\eta_A),f)
					=f
			\end{math};
		\item
			\begin{math}
				\pull\Delta(f,\pull S(f))=
					\pull\Delta(\pull S(f),f)=
					\pull\varepsilon(\eta_A)
			\end{math}; y
		\item
			\begin{math}
				\pull\Delta(f,g)=\pull\Delta(g,f)
			\end{math}.
	\end{itemize}
	%
\end{propoRectaAfin}

\begin{proof}
	Las igualdades
	\begin{align*}
		\pull\Delta(f,\pull\Delta(g,h))(x) & \,=\,
			f(x)+\pull\Delta(g,h)(x)\,=\,f(x)+(g(x)+h(x))
			\quad\text{y} \\
		\pull\Delta(\pull\Delta(f,g),h)(x) & \,=\,
			\pull\Delta(f,g)(x)+h(x)\,=\,(f(x)+g(x))+h(x)
	\end{align*}
	%
	demuestran, por ejemplo, que $\pull\Delta$ es asociativa. Haciendo uso
	de
	\begin{align*}
		\pull S(f)(x) \,=\,f(-x)\,=\,-f(x)
			& \quad\text{y}\quad
			\pull\varepsilon(\eta_A)(x) \,=\,\eta_A(0)\,=\,0
		\text{ ,}
	\end{align*}
	%
	vemos que $\pull\varepsilon(\eta_A)$ es un neutro para esta suma y que
	el morfismo $\pull S(f)$ es el inverso de $f$. Por \'{u}ltimo, de
	$f(x)+g(x)=g(x)+f(x)$, deducimos que $\pull\Delta$ es abeliana.
\end{proof}

Las operaciones $\pull\Delta$, $\pull\varepsilon$ y $\pull S$ son esencialmente
independientes del \'{a}lgebra $A$ en $\Homalg\big(k[x],A\big)$.

\begin{propoRectaAfin}\label{propo:rectaafin:sumanatural}
	Todo morfismo de $k$-\'{a}lgebras $\varphi:\,A\rightarrow B$ induce
	diagramas conmutativos
	\begin{center}
		\begin{tikzcd}[column sep=small]
			\Homalg\big(k[x',x''],A\big) \arrow[r,"\pull\Delta"]
				\arrow[d,"\push\varphi"'] &
				\Homalg\big(k[x],A\big)
					\arrow[d,"\push\varphi"] \\
			\Homalg\big(k[x',x''],B\big) \arrow[r,"\pull\Delta"] &
				\Homalg\big(k[x],B\big)
		\end{tikzcd}
		\begin{tikzcd}[column sep=small]
			\Homalg\big(k,A\big) \arrow[r,"\pull\varepsilon"]
				\arrow[d,"\push\varphi"'] &
				\Homalg\big(k[x],A\big)
					\arrow[d,"\push\varphi"] \\
			\Homalg\big(k,B\big) \arrow[r,"\pull\varepsilon"] &
				\Homalg\big(k[x],B\big)
		\end{tikzcd}
		\begin{tikzcd}[column sep=small]
			\Homalg\big(k[x],A\big) \arrow[r,"\pull S"]
				\arrow[d,"\push\varphi"'] &
				\Homalg\big(k[x],A\big)
					\arrow[d,"\push\varphi"] \\
			\Homalg\big(k[x],B\big) \arrow[r,"\pull S"] &
				\Homalg\big(k[x],B\big)
		\end{tikzcd}
	\end{center}
	con $\push\varphi(f)=\varphi\circ f$.
\end{propoRectaAfin}

\begin{proof}
	\begin{math}
		\push\varphi\circ\pull{\mathtt X}(f)=
			\varphi\circ f\circ\mathtt X=
			\pull{\mathtt X}\circ\push\varphi(f)
	\end{math}.
\end{proof}

Podemos interpretar esto de dos maneras. Por un lado, la Proposici\'{o}n~%
\ref{propo:rectaafin:sumanatural} quiere decir que
\begin{equation}
	\label{eq:rectaafin:sumanatural}
	\begin{aligned}
		\pull\Delta & \,:\,\Homalg\big(k[x',x''],-\big)
			\,\xrightarrow\cdot\,
			\Homalg\big(k[x],-\big) \text{ ,} \\
		\pull\varepsilon & \,:\,\Homalg\big(k,-\big)
			\,\xrightarrow\cdot\,
			\Homalg\big(k[x],-\big) \quad\text{y} \\
		\pull S & \,:\,\Homalg\big(k[x],-\big)\,\xrightarrow\cdot\,
			\Homalg\big(k[x],-\big)
	\end{aligned}
\end{equation}
%
son transformaciones naturales. Por otro lado, v\'{\i}a la correspondencia
\eqref{eq:obs:rectaafin}, el primero de los diagramas muestra que, para cada
morfismo $\varphi:\,A\rightarrow B$, la funci\'{o}n inducida
\begin{math}
	\push\varphi:\,\Homalg\big(k[x],A\big)\rightarrow
		\Homalg\big(k[x],B\big)
\end{math}
es un morfismo de grupos, si en cada conjunto $\Homalg\big(k[x],A\big)$
la suma est\'{a} dada por $\pull\Delta$.

\begin{coroRectaAfin}\label{coro:rectaafin}
	Existe un funtor $G:\,\CommAlg[k]\rightarrow\Grp$ tal que
	\begin{align*}
		\olvido\circ G & \,=\,\Homalg\big(k[x],-\big)
		\text{ ,}
	\end{align*}
	%
	donde $\olvido:\,\Grp\rightarrow\Set$ es el funtor olvido. Este funtor
	est\'{a} definido en objetos por $A\mapsto G(A)$, donde
	$G(A)$ es el grupo cuyo conjunto subyacente es
	$\Homalg\big(k[x],A\big)$ y cuya operaci\'{o}n binaria est\'{a} dada
	por $\pull\Delta(A)=\pull\Delta$.

	La \emph{funci\'{o}n} (biyectiva, por
	\eqref{eq:morfismosdesdepolinomios})
	\begin{align*}
		\big(f\mapsto f(x)\big) & \,:\,\Homalg\big(k[x],A\big)
			\,\rightarrow\,A
	\end{align*}
	%
	determina un isomorfismo de grupos $\tau_A:\,G(A)\rightarrow (A,+)$. A
	su vez, estos isomorfismos determinan un isomorfismo natural
	$\tau:\,G\xrightarrow\cdot (-,+)$ de $G$ en el funtor
	$A\mapsto (A,+)$.
\end{coroRectaAfin}

\begin{proof}
	Que $\tau_A(f)=f(x)$ es morfismo de grupos, es consecuencia de la
	conmutatividad del primer diagrama de la Proposici\'{o}n~%
	\ref{propo:rectaafin}. Lo \'{u}nico que queda por verificar es la
	naturalidad de $\tau$. Pero
	\begin{align*}
		\tau_B\circ\push\varphi(f) & \,=\,(\varphi\circ f)(x)\,=\,
			\varphi(f(x))\,=\,\varphi\circ\tau_A(f)
		\text{ ,}
	\end{align*}
	%
	donde, en el \'{u}ltimo t\'{e}rmino, $\varphi$ denota el morfismo de
	grupos $(A,+)\rightarrow(B,+)$.
\end{proof}

\subsubsection{El grupo multiplicativo}

Dada una $k$-\'{a}legbra $A$, $A^\times$ denota el grupo multiplicativo
compuesto por las unidades de $A$. Dado un morfismo de \'{a}lgebras
$\varphi:\,A\rightarrow B$, se cumple que $\varphi(A^\times)\subset B^\times$ y
que $\varphi$ define, por restricci\'{o}n, un morfismo de grupos
$\varphi^\times:\,A^\times\rightarrow B^\times$. Nos referimos, con
$(-,\times):\,\CommAlg[k]\rightarrow\Grp$, al funtor dado por
$A\mapsto A^\times$ en objetos y por $\varphi\mapsto\varphi^\times$ en
morfismos. Escribiremos $\varphi$ en lugar de $\varphi^\times$.

Sea $I\triangleleft k[x,y]$ el ideal $I=\generado{xy-1}$. El cociente
\begin{align*}
	k[x,x^{-1}] & \,=\,k[x,y]/\generado{xy-1}
\end{align*}
%
posee la siguiente propiedad: para toda $k$-\'{a}lgebra conmutativa $A$,
la aplicaci\'{o}n $\tau_A:\,f\mapsto f(x)$ es una biyecci\'{o}n
\begin{equation}
	\label{eq:polinomiosdelaurent}
	\Homalg\big(k[x,x^{-1}],A\big) \,\simeq\,A^\times
	\text{ .}
\end{equation}
%
Esta biyecci\'{o}n es natural en $A$. De manera similar, si definimos
\begin{align*}
	k[x',x'',{x'}^{-1},{x''}^{-1}] & \,=\,k[x',y',x'',y'']/
		\generado{x'y'-1,x''y''-1}
	\text{ ,}
\end{align*}
%
obtenemos una biyecci\'{o}n natural
\begin{equation}
	\label{eq:polinomiosdelaurentproducto}
	\Homalg\big(k[x',x'',{x'}^{-1},{x''}^{-1}],A\big) \,\simeq\,
		A^\times\,\times\,A^\times
	\text{ .}
\end{equation}
%

El grupo multiplicativo de un \'{a}lgebra $A$ est\'{a} definido por tres
funciones:
\begin{align*}
	\times\,:\,A^\times\,\times\,A^\times\,\rightarrow\,A^\times
		& \quad\text{,}\quad
	1\,:\,\{1\}\,\rightarrow\,A^\times
		\quad\text{y}\quad
	\null^{-1}\,:\,A^\times\,\rightarrow\,A^\times
	\text{ .}
\end{align*}
%

\begin{propoMultiplicativo}\label{propo:multiplicativo}
	Sean $\Delta:\,k[x,x^{-1}]\rightarrow k[x',x'',{x'}^{-1},{x''}^{-1}]$,
	$\varepsilon:\,k[x,x^{-1}]\rightarrow k$ y
	$S:\,k[x,x^{-1}]\rightarrow k[x,x^{-1}]$ los morfismos de \'{a}lgebras
	determinados por
	\begin{equation}
		\label{eq:multiplicativo}
		\Delta(x)\,=\,x'\,x'' \quad\text{,}\quad
		\varepsilon(x)\,=\,1 \quad\text{y}\quad
		S(x)\,=\,x^{-1}
		\text{ .}
	\end{equation}
	%
	Entonces los siguientes diagramas conmutan.
	\begin{center}
		\begin{tikzcd}[column sep=small]
			\Homalg\big(k[x',x'',{x'}^{-1},{x''}^{-1}],A\big)
				\arrow[r,"\sim"]
				\arrow[d,"\pull\Delta"'] &
				A^\times\times A^\times \arrow[d,"\times"] \\
			\Homalg\big(k[x,x^{-1}],A\big) \arrow[r,"\sim"] &
				A^\times
		\end{tikzcd}
		\begin{tikzcd}[column sep=small]
			\Homalg\big(k,A\big) \arrow[r,"\sim"]
				\arrow[d,"\pull\varepsilon"'] &
				\{1\} \arrow[d,"1"] \\
			\Homalg\big(k[x,x^{-1}],A\big) \arrow[r,"\sim"] &
				A^\times
		\end{tikzcd}
		\begin{tikzcd}[column sep=small]
			\Homalg\big(k[x,x^{-1}],A\big) \arrow[r,"\sim"]
				\arrow[d,"\pull S"'] &
				A^\times \arrow[d,"\null^{-1}"] \\
			\Homalg\big(k[x,x^{-1}],A\big) \arrow[r,"\sim"] &
				A^\times
		\end{tikzcd}
	\end{center}
\end{propoMultiplicativo}
Las funciones $\pull\Delta$, $\pull\varepsilon$ y $\pull S$ son los pullbacks
de los morfismos correspondientes. Como en la Proposici\'{o}n~%
\ref{propo:rectaafin}, las flechas horizontales est\'{a}n dadas por
\eqref{eq:polinomiosdelaurent} y \eqref{eq:polinomiosdelaurentproducto}. En
este caso, como estamos tratando el grupo multiplicativo, identificamos el
\'{u}nico elemento del conjunto $\Homalg\big(k,A\big)=\{\eta_A\}$ con el
\'{u}nico elemento de $\{1\}$.

Se puede demostrar la existencia de una biyecci\'{o}n similar a la mencionada
en la Observaci\'{o}n~\ref{obs:rectaafin}:
\begin{equation}
	\label{eq:obs:multiplicativo}
	\Homalg\big(k[x',x'',{x'}^{-1},{x''}^{-1}],A\big) \,\simeq\,
		\Homalg\big(k[x,x^{-1}],A\big)\,\times\,
		\Homalg\big(k[x,x^{-1}],A\big)
	\text{ .}
\end{equation}
%
Esto nos permite darle una estructura de grupo al conjunto
$\Homalg\big(k[x,x^{-1}],A\big)$ para un \'{a}lgebra fija $A$ y demostrar
resultados an\'{a}logos a los demostrados en el caso del grupo aditivo. En
particular, se deduce el siguiente corolario.

\begin{coroMultiplicativo}\label{coro:multiplicativo}
	Existe un funtor $G:\,\CommAlg[k]\rightarrow\Grp$ tal que
	\begin{align*}
		\olvido\circ G & \,=\,\Homalg\big(k[x,x^{-1}],-\big)
		\text{ .}
	\end{align*}
	%
	Si $A$ es una $k$-\'{a}lgebra conmutativa, $G(A)$ es el grupo
	$\Homalg\big(k[x,x^{-1}],A\big)$ con el producto, el neutro y el
	inverso dados por $\pull\Delta$, $\pull\varepsilon$ y $\pull S$.
	Las biyecciones \eqref{eq:polinomiosdelaurent} son, con respecto a esta
	estructura de grupo, isomorfismos de grupos e inducen un isomorfismo
	natural $\tau:\,G\xrightarrow\cdot (-,\times)$.
\end{coroMultiplicativo}

\subsubsection{$\GL(2)$ y $\SL(2)$}

Apliquemos las ideas anteriores para describir el producto de matrices. Vamos a
escribir $\MM(2)$ para denotar el \'{a}lgebra de polinomios $k[a,b,c,d]$.
Entonces la aplicaci\'{o}n
\begin{equation}
	\label{eq:morfismoenmatriz}
	f\,\mapsto\,f\Big(
		\begin{bmatrix} a & b \\ c & d \end{bmatrix}\Big)\,=\,
		\begin{bmatrix}	f(a) & f(b) \\ f(c) & f(d) \end{bmatrix}
\end{equation}
%
define una biyecci\'{o}n
\begin{equation}
	\label{eq:morfismosdesdematrices}
	\Homalg\big(\MM(2),A\big) \,\simeq\,\MM[2\times 2](A)
	\text{ ,}
\end{equation}
%
identificando $A^4=\MM[2\times 2](A)$.

Si queremos expresar el producto de matrices de manera universal, lo primero
que haremos ser\'{a} duplicar las variables: definimos
\begin{align*}
	\MM(2)^{\otimes 2} & \,=\,k[a',b',c',d',a'',b'',c'',d'']
\end{align*}
%
y buscaremos luego, un morfismo de \'{a}lgebras
$\Delta:\,\MM(2)\rightarrow\MM(2)^{\otimes 2}$ que haga conmutar el diagrama
\begin{center}
	\begin{tikzcd}
		\Homalg\big(\MM(2)^{\otimes 2},A\big) \arrow[r,"\sim"]
			\arrow[d,"\pull\Delta"'] &
			\MM[2\times 2](A)^2
				\arrow[d,"\cdot"] \\
		\Homalg\big(\MM(2),A\big) \arrow[r,"\sim"] & \MM[2\times 2](A)
	\end{tikzcd}
\end{center}
donde $\cdot:\,\MM[2\times 2](A)^2\rightarrow\MM[2\times 2](A)$ es el producto
de matrices con coeficientes en el \'{a}lgebra $A$. La flecha inferior est\'{a}
dada por \eqref{eq:morfismosdesdematrices} y la flecha superior es la
biyecci\'{o}n
\begin{math}
	\Homalg\big(\MM(2)^{\otimes 2},A\big)\simeq\MM[2\times 2](A)^2
\end{math} dada por
\begin{math}
	f\mapsto\big(f\,
		\left[\begin{smallmatrix}
			a' & b' \\ c' & d'
		\end{smallmatrix}\right],
		f\,
		\left[\begin{smallmatrix}
			a'' & b'' \\ c'' & d''
		\end{smallmatrix}\right]\big)
\end{math}. Para que este diagrama conmute, lo que tiene que cumplirse es
\begin{align*}
	f\circ\Delta\Big(\begin{bmatrix} a & b \\ c & d \end{bmatrix}\Big)
		& \,=\,
		f\Big(\begin{bmatrix} a' & b' \\ c' & d' \end{bmatrix}\Big)\,
		f\Big(\begin{bmatrix} a'' & b'' \\ c'' & d'' \end{bmatrix}\Big)
			\,=\,
		f\Big(\begin{bmatrix} a' & b' \\ c' & d' \end{bmatrix}\,
			\begin{bmatrix}
				a'' & b'' \\ c'' & d''
			\end{bmatrix}\Big)
\end{align*}
%
para todo morfismo de \'{a}lgebras $f:\,\MM(2)^{\otimes 2}\rightarrow A$.

\begin{propoProductoDeMatrices}\label{propo:productodematrices}
	Sea $\Delta:\,\MM(2)\rightarrow\MM(2)^{\otimes 2}$ el morfismo de
	\'{a}lgebras determinado por
	\begin{align*}
		\Delta\Big(\begin{bmatrix} a & b \\ c & d \end{bmatrix}\Big)
			& \,=\,
			\begin{bmatrix} a' & b' \\ c' & d' \end{bmatrix}\,
			\begin{bmatrix} a'' & b'' \\ c'' & d'' \end{bmatrix}
		\text{ ,}
	\end{align*}
	%
	es decir,
	\begin{equation}
		\label{eq:productodematrices}
		\begin{aligned}
			\Delta(a) \,=\, a'a'' + b'c'' & \quad\text{,}\quad
			\Delta(b) \,=\, a'b'' + b'd'' \text{ ,} \\
			\Delta(c) \,=\, c'a'' + d'c'' & \quad\text{y}\quad
			\Delta(d) \,=\, c'b'' + d'd''
			\text{ .}
		\end{aligned}
	\end{equation}
	%
	Entonces el diagrama conmuta.
\end{propoProductoDeMatrices}

Vamos a usar este morfismo $\Delta$ para tratar los grupos $\GL[2](A)$ y
$\SL[2](A)$, matrices invertibles y, respectivamente, matrices de determinante
$1$ con coeficientes en $A$. Introducimos las \'{a}lgebras
\begin{align*}
	\GL(2) & \,=\,\MM(2)[t]/\generado{(ad-bc)\,t-1}\quad\text{y} \\
	\SL(2) & \,=\,\GL(2)/\generado{t-1}\,=\,
		\MM(2)/\generado{ad-bc-1}
	\text{ .}
\end{align*}
%

\begin{propoLinealGeneral}\label{propo:morfismosdesdelinealgeneral}
	Para toda \'{a}lgebra conmutativa $A$, la expresi\'{o}n
	\eqref{eq:morfismoenmatriz} determina biyecciones
	\begin{equation}
		\label{eq:morfismosdesdelinealgeneral}
		\Homalg\big(\GL(2),A\big) \,\simeq\,\GL[2](A)
			\quad\text{y}\quad
		\Homalg\big(\SL(2),A\big) \,\simeq\,\SL[2](A)
		\text{ .}
	\end{equation}
	%
\end{propoLinealGeneral}

\begin{proof}
	Si
	\begin{math}
		\big[\begin{smallmatrix}
			\alpha & \beta \\ \gamma & \delta
		\end{smallmatrix}\big]\in\GL[2](A)
	\end{math}, por \eqref{eq:morfismosdesdepolinomios}, existe un
	\'{u}nico morfismo de \'{a}lgebras $f:\,\MM(2)[t]\rightarrow A$ tal que
	\begin{align*}
		f\Big(\begin{bmatrix} a & b \\ c & d \end{bmatrix}\Big) \,=\,
			\begin{bmatrix}
				\alpha & \beta \\ \gamma & \delta
			\end{bmatrix} & \quad\text{y}\quad
		f(t)\,=\,(\alpha\delta-\beta\gamma)^{-1}
		\text{ .}
	\end{align*}
	%
	Como $f$ es morfismo de \'{a}lgebras, se verifica que
	\begin{math}
		f((ad-bc)\,t-1)=0
	\end{math} y $f$ pasa al cociente $\GL(2)$. Si la matriz pertenece a
	$\SL[2](A)$, entonces $f(t-1)=0$, tambi\'{e}n, y $f$ se factoriza por
	$\SL(2)$.
\end{proof}

Sean, ahora,
\begin{align*}
	\GL(2)^{\otimes 2} & \,=\,\MM(2)^{\otimes 2}[t',t'']/
		\generado{(a'd'-b'c')\,t'-1,(a''d''-b''c'')\,t''-1}
		\quad\text{y} \\
	\SL(2)^{\otimes 2} & \,=\,\GL(2)^{\otimes 2}/\generado{t'-1,t''-1}
		\,=\,\MM(2)^{\otimes 2}/\generado{a'd'-b'c'-1,a''d''-b''c''-1}
	\text{ .}
\end{align*}
%

\begin{obsProductoDeMatrices}\label{obs:productodematrices}
	El morfismo $\Delta:\,\MM(2)\rightarrow\MM(2)^{\otimes 2}$ verifica
	\begin{align*}
		\Delta(ad-bc) & \,=\,(a'd'-b'c')\,(a''d''-b''c'')
		\text{ .}
	\end{align*}
	%
	Extendemos $\Delta$ a un morfismo de \'{a}lgebras
	$\Delta:\,\MM(2)[t]\rightarrow\MM(2)^{\otimes 2}[t',t'']$, definiendo
	\begin{align*}
		\Delta(t) & \,=\,t'\,t''
		\text{ .}
	\end{align*}
	%
	En particular,
	\begin{math}
		\Delta((ad-bc)\,t-1)=(a'd'-b'c')\,(a''d''-b''c'')\,t'\,t''-1
	\end{math}. Componiendo con la proyecci\'{o}n de
	$\MM(2)^{\otimes 2}[t',t'']$ en $\GL(2)^{\otimes 2}$, concluimos que
	existe un \'{u}nico morfismo de \'{a}lgebras
	$\Delta:\,\GL(2)\rightarrow\GL(2)^{\otimes 2}$ tal que
	\begin{center}
		\begin{tikzcd}
			\MM(2)[t] \arrow[r,"\Delta"] \arrow[d] &
				\MM(2)^{\otimes 2}[t',t''] \arrow[d] \\
			\GL(2)\arrow[r,"\Delta"'] & \GL(2)^{\otimes 2}
		\end{tikzcd}
	\end{center}
	conmuta. Componiendo con
	$\GL(2)^{\otimes 2}\rightarrow\SL(2)^{\otimes 2}$, obtenemos
	$\Delta:\,\SL(2)\rightarrow\SL(2)^{\otimes 2}$.
\end{obsProductoDeMatrices}

Las \'{a}lgebras $\GL(2)^{\otimes 2}$ y $\SL(2)^{\otimes 2}$ tienen la
propiedad de que, para toda \'{a}lgebra conmutativa $A$, existen biyecciones
\begin{equation}
	\label{eq:morfismosdesdelinealgeneralproducto}
	\Homalg\big(\GL(2)^{\otimes 2},A\big)\,\simeq\,
		\GL[2](A)^2 \quad\text{y}\quad
	\Homalg\big(\SL(2)^{\otimes 2},A\big)\,\simeq\,
		\SL[2](A)^2
\end{equation}
%
dadas por evaluar un morfismo en los generadores de las \'{a}lgebras.

\begin{propoLinealGeneral}\label{propo:linealgeneralcoproducto}
	Con $\Delta$ definido como en la Observaci\'{o}n~%
	\ref{obs:productodematrices}, los diagramas siguientes conmutan.
	\begin{center}
		\begin{tikzcd}[column sep=small]
			\Homalg\big(\GL(2)^{\otimes 2},A\big) \arrow[r,"\sim"]
				\arrow[d,"\pull\Delta"'] &
				\GL[2](A)^2 \arrow[d,"\cdot"] \\
			\Homalg\big(\GL(2),A\big)\arrow[r,"\sim"] & \GL[2](A)
		\end{tikzcd}
		\begin{tikzcd}[column sep=small]
			\Homalg\big(\SL(2)^{\otimes 2},A\big) \arrow[r,"\sim"]
				\arrow[d,"\pull\Delta"'] &
				\SL[2](A)^2 \arrow[d,"\cdot"] \\
			\Homalg\big(\SL(2),A\big)\arrow[r,"\sim"] & \SL[2](A)
		\end{tikzcd}
	\end{center}
\end{propoLinealGeneral}
Las flechas horizontales est\'{a}n dadas por
\eqref{eq:morfismosdesdelinealgeneral} y por
\eqref{eq:morfismosdesdelinealgeneralproducto}. La flecha del lado derecho de
cada uno de los cuadrados es el producto de matrices.

\begin{propoLinealGeneral}\label{propo:linealgeneralcounidadyantipoda}
	Si definimos morfismos de \'{a}lgebras
	\begin{align*}
		\varepsilon\,:\,\GL(2)\,\rightarrow\,k
			& \quad\text{,}\quad
		S\,:\,\GL(2)\,\rightarrow\,\GL(2) \text{ ,} \\
		\varepsilon\,:\,\SL(2)\,\rightarrow\,k
			& \quad\text{,}\quad
		S\,:\,\SL(2)\,\rightarrow\,\SL(2)
	\end{align*}
	%
	determinados por
	\begin{equation}
		\label{eq:linealgeneralcounidadyantipoda}
		\begin{aligned}
			\varepsilon\,
				\begin{bmatrix} a & b \\ c & d \end{bmatrix}
					\,=\,
				\begin{bmatrix} 1 & \\ & 1 \end{bmatrix}
				& \quad\text{,}\quad
					\varepsilon(t)\,=\,1 \text{ ,} \\
			S\,\begin{bmatrix} a & b \\ c & d \end{bmatrix}
				\,=\,(ad-bc)^{-1}\,
				\begin{bmatrix} d & -b \\ -c & a \end{bmatrix}
				& \quad\text{y}\quad
					S(t) \,=\,t^{-1}
		\end{aligned}
		\text{ ,}
	\end{equation}
	%
	entonces los siguientes diagramas conmutan.
	\begin{center}
		\begin{tikzcd}[column sep=small]
			\Homalg\big(k,A\big) \arrow[r,"\sim"]
				\arrow[d,"\pull\varepsilon"'] &
				\{1\} \arrow[d,"1"] \\
			\Homalg\big(\GL(2),A\big) \arrow[r,"\sim"] & \GL[2](A)
		\end{tikzcd}
		\begin{tikzcd}[column sep=small]
			\Homalg\big(k,A\big) \arrow[r,"\sim"]
				\arrow[d,"\pull\varepsilon"'] &
				\{1\} \arrow[d,"1"] \\
			\Homalg\big(\SL(2),A\big) \arrow[r,"\sim"] & \SL[2](A)
		\end{tikzcd}
		\begin{tikzcd}[column sep=small]
			\Homalg\big(\GL(2),A\big) \arrow[r,"\sim"]
				\arrow[d,"\pull S"'] &
				\GL[2](A) \arrow[d,"\null^{-1}"] \\
			\Homalg\big(\GL(2),A\big) \arrow[r,"\sim"] & \GL[2](A)
		\end{tikzcd}
		\begin{tikzcd}[column sep=small]
			\Homalg\big(\SL(2),A\big) \arrow[r,"\sim"]
				\arrow[d,"\pull S"'] &
				\SL[2](A) \arrow[d,"\null^{-1}"] \\
			\Homalg\big(\SL(2),A\big) \arrow[r,"\sim"] & \SL[2](A)
		\end{tikzcd}
	\end{center}
\end{propoLinealGeneral}

Identificaciones an\'{a}logas a \eqref{eq:obs:rectaafin} y a
\eqref{eq:obs:multiplicativo},
\begin{align*}
	\Homalg\big(\GL(2)^{\otimes 2},A\big) & \,\simeq\,
		\Homalg\big(\GL(2),A\big)\,\times\,
		\Homalg\big(\GL(2),A\big) \quad\text{y} \\
	\Homalg\big(\SL(2)^{\otimes 2},A\big) & \,\simeq\,
		\Homalg\big(\SL(2),A\big)\,\times\,
		\Homalg\big(\SL(2),A\big)
	\text{ ,}
\end{align*}
%
permiten darle a los conjuntos $\Homalg\big(\GL(2),A\big)$ y
$\Homalg\big(\SL(2),A\big)$ estructuras de grupo con el producto determinado
por $\pull\Delta$. El elemento neutro con respecto a esta operaci\'{o}n
es $\pull\varepsilon(\eta_A)=\eta_A\circ\varepsilon$, donde $\eta_A$ es el
\'{u}nico morfismo en $\Homalg\big(k,A\big)$. El inverso de un morfismo $f$ es
la composici\'{o}n $\pull S(f)=f\circ S$. Se puede verificar, adem\'{a}s, que,
tanto para $\GL(2)$ como para $\SL(2)$, los pullbacks $\pull\Delta$,
$\pull\varepsilon$ y $\pull S$ son naturales en $A$.
%
En particular, la naturalidad de $\pull\Delta$ implica que, dado un morfismo
$\varphi:\,A\rightarrow B$, la funci\'{o}n
$\push\varphi:\,\Homalg\big(\GL(2),A\big)\rightarrow\Homalg\big(\GL(2),B\big)$
dada por $\push\varphi(f)=\varphi\circ f$, es, en realidad, un morfismo de
grupos. En definitiva, podemos definir un funtor
$G:\,\CommAlg[k]\rightarrow\Grp$ que en objetos est\'{a} dado por asociarle a
una $k$-\'{a}lgebra conmutativa $A$ el grupo $\Homalg\big(\GL(2),A\big)$ con el
producto dado por $\pull\Delta$ y que en morfismos est\'{a} dado por
$\varphi\mapsto\push\varphi$. Lo que vale, adem\'{a}s, es que evaluar un
morfismo $f:\,\GL(2)\rightarrow A$ en los generadores de $\GL(2)$ es un
isomorfismo de grupos $\Homalg\big(\GL(2),A\big)\rightarrow\GL[2](A)$, natural
en $A$ y, por lo tanto, determina un isomorfismo natural
$G\xrightarrow\cdot\GL[2]$, donde $\GL[2]$ es el funtor que a un \'{a}lgebra
conmutativa le asigna el grupo de matrices invertibles con coeficientes en el
\'{a}lgebra $\GL[2](A)$. Afirmaciones an\'{a}logas son ciertas para $\SL(2)$.

\subsection{Producto tensorial}\label{subsec:kalgebras:productotensorial}

Cada vez que quisimos definir un morfismo correspondiente a la operaci\'{o}n
binaria del grupo que est\'{a}bamos estudiando, necesitamos ``duplicar'' las
variables, definir una nueva \'{a}lgebra y enunciar una propiedad universal
para este \'{a}lgebra. A continuaci\'{o}n, formalizamos estas ideas.

\subsubsection{Producto tensorial de m\'{o}dulos}

Empecemos repasando la definici\'{o}n de producto tensorial de $k$-m\'{o}dulos.
Recordemos que $k$ es un anillo conmutativo.

Dados $k$-m\'{o}dulos $A$ y $B$, el \emph{producto tensorial} de $A$ con $B$ es
un par compuesto por un $k$-m\'{o}dulo, denotado en general por
$A\tensor[k]B$, y una tranformaci\'{o}n \emph{$k$-bilineal}
\begin{align*}
	\tensor & \,:\,A\,\times\,B\,\rightarrow\,A\tensor[k]B
\end{align*}
%
que posee la siguiente propiedad universal: dada una transformaci\'{o}n
$k$-bilineal $h:\,A\times B\rightarrow C$ en un $k$-m\'{o}dulo $C$, existe un
\'{u}nico morfismo de $k$-m\'{o}dulos $t:\,A\tensor[k]B\rightarrow C$ tal que
$t\circ\tensor=h$.
\begin{center}
	\begin{tikzcd}[row sep=large]
		A\times B \arrow[r,"\tensor"] \arrow[dr,"h"'] &
			A\tensor[k]B \arrow[d,dashed,"t"] \\ & C
	\end{tikzcd}
\end{center}
El producto tensorial $A\tensor[k]B$ se puede realizar como cociente del
$k$-m\'{o}dulo libre en el conjunto $A\times B$, o, lo que es lo mismo,
$A\tensor[k]B$ est\'{a} generado, como $k$-m\'{o}dulo por los elementos de la
forma $a\tensor b=\tensor(a,b)$, con $a\in A$ y $b\in B$.

\begin{ejemploProductoTensorialDeModulos}%
	\label{ejemplo:productotensorialdemodulos}
	Dado un $k$-m\'{o}dulo $A$, la funci\'{o}n
	$h_0:\,k\times A\rightarrow k$ dada por $h_0(\lambda,a)=\lambda\,a$ es
	universal entre las funciones $k$-bilineales con dominio en
	$k\times A$: si $h:\,k\times A\rightarrow C$ es bilineal y
	$h=t\circ h_0$ para cierta $t:\,A\rightarrow C$, entonces
	\begin{align*}
		t(a) & \,=\,t(h_0(1,a)) \,=\,h(1,a)
		\text{ ,}
	\end{align*}
	%
	lo que muestra que, si $t$ existe, est\'{a} determinada por los valores
	de $h$ en pares de la forma $(1,a)$. Si definimos $t:\,A\rightarrow C$
	por $t(a)=h(1,a)$, como $h$ es bilineal, $t$ es morfismo de $k$-%
	m\'{o}dulos y, m\'{a}s aun,
	\begin{align*}
		h(\lambda,a) & \,=\,\lambda\,h(1,a)\,=\,\lambda\,t(a)\,=\,
			t(\lambda\,a)\,=\,t\circ h_0(\lambda,a)
		\text{ .}
	\end{align*}
	%
	Concluimos, as\'{\i}, que $A$ tiene la propiedad universal del producto
	tensorial $k\tensor[k]A$ y podemos identificar can\'{o}nicamente
	\begin{align*}
		A & \,\simeq\,k\tensor[k]A
		\text{ .}
	\end{align*}
	%
	Esta identificaci\'{o}n est\'{a} dada, expl\'{\i}citamente, por
	$a\mapsto 1\tensor a$. Un poco m\'{a}s en general, la funci\'{o}n
	$h_0:\,k^n\times A\rightarrow A^n$ dada por
	$h_0((\lista{\lambda}{n}),a)=(\lambda_1\,a,\,\dots,\,\lambda_n\,a)$ es
	universal entre las transformaciones $k$-bilineales con dominio en
	$k^n\times A$, mostrando que $k^n\tensor[k]A$ se identifica
	naturalmente con $A^n$.
\end{ejemploProductoTensorialDeModulos}

\begin{propoProductoTensorialDeModulos}\label{propo:productotensorialdemodulos}
	El producto tensorial de $k$-m\'{o}dulos posee las siguientes
	propiedades: dados $k$-m\'{o}dulos $A$, $B$ y $C$,
	\begin{align*}
		A\tensor[k]\big(B\,\oplus\,C\big) & \,\simeq\,
			\big(A\tensor[k]B\big)\,\oplus\,
			\big(A\tensor[k]C\big) \text{ ,} \\
		A\tensor[k]\big(B\tensor[k] C\big) & \,\simeq\,
			\big(A\tensor[k]B\big)\tensor[k]C \text{ ,} \\
		A\tensor[k]B & \,\simeq\,B\tensor[k]A
		\text{ .}
	\end{align*}
	%
	Estos isomorfismos est\'{a}n determinados, respectivamente, por
	\begin{align*}
		a\tensor(b+c) & \,\mapsto\,(a\tensor b)+(a\tensor c)
			\text{ ,} \\
		a\tensor(b\tensor c) & \,\mapsto\,(a\tensor b)\tensor c
			\quad\text{y} \\
		a\tensor b & \,\mapsto\,b\tensor a
		\text{ .}
	\end{align*}
	%
\end{propoProductoTensorialDeModulos}

\begin{obsProductoTensorialDeModulos}\label{obs:productotensorialdemodulos}
	Usando el segundo isomorfismo, identificamos
	\begin{align*}
		A\tensor[k]B\tensor[k]C & \,=\,
			A\tensor[k]\big(B\tensor[k] C\big) \,=\,
			\big(A\tensor[k]B\big)\tensor[k]C
		\text{ .}
	\end{align*}
	Escribimos $\swap$ o $\swap[{A\tensor[k]B}]$ para referirnos al tercer
	isomorfismo. Es decir, dados $a\in A$ y $b\in B$,
	\begin{align*}
		\swap(a\tensor b) & \,=\,
			\swap[{A\tensor[k]B}](a\tensor b) \,=\,	b\tensor a
		\text{ .}
	\end{align*}
	%
\end{obsProductoTensorialDeModulos}
%
% \begin{proof}
	% En cuanto a la primera, la aplicaci\'{o}n
	% \begin{align*}
		% h(a,b+c) & \,=\,\inc[{A\tensor[k]B}](a\tensor b)\,+\,
			% \inc[{A\tensor[k]C}](a\tensor c)
		% \text{ ,}
	% \end{align*}
	% %
	% donde $\inc[{A\tensor[k]B}]$ e $\inc[{A\tensor[k]C}]$ denotan las
	% inclusiones en la suma directa, es lineal en $a\in A$ y lineal en el
	% par $b\in B$, $c\in C$. Por propiedad universal del producto tensorial,
	% existe un \'{u}nico morfismo de $k$-m\'{o}dulos
	% \begin{math}
		% f:\,A\tensor[k](B\oplus C)\rightarrow
			% (A\tensor[k] B)\oplus (A\tensor[k] C)
	% \end{math} tal que $f(a\tensor (b+c))=h(a,b+c)$. Es decir, obtenemos un
	% diagrama conmutativo:
	% \begin{center}
		% \begin{tikzcd}[column sep=small,row sep=large]
		% A\,\times\,\big(B\,\oplus\,C\big) \arrow[r,"\tensor"]
			% \arrow[dr,"h"'] &
			% A\tensor[k]\big(B\,\oplus\,C\big)
				% \arrow[d,dashed,"f"'] \\
		% & \big(A\tensor[k]B\big)\,\oplus\,\big(A\tensor[k]C\big)
	% \end{tikzcd}
	% \end{center}
	% Por otro lado, las inclusiones
	% \begin{math}
		% B\hookrightarrow B\oplus C\hookleftarrow C
	% \end{math} determinan inclusiones
	% \begin{center}
	% \begin{tikzcd}
		% A\,\times\,B\arrow[r,hook,"{\inc[A\times B]}"] &
		% A\times (B\oplus C) &
		% A\,\times\,C\arrow[l,hook',"{\inc[A\times C]}"']
	% \end{tikzcd}
	% \end{center}
	% dadas por $\inc[A\times B](a,b)=(a,\inc[B](b))$ e
	% $\inc[A\times C](a,c)=(a,\inc[C](c))$. La composiciones
	% $\tensor\circ\inc[A\times B]$ y $\tensor\circ\inc[A\times C]$ son
	% bilineales y, por las propiedades universales de $A\tensor[k]B$ y
	% $A\tensor[k]C$, existen morfismos
	% \begin{center}
	% \begin{tikzcd}
		% A\tensor[k]B\arrow[r,"j_{A\tensor[k]B}"] &
		% A\tensor[k]\big(B\,\oplus\,C\big) &
		% A\tensor[k]C\arrow[l,"j_{A\tensor[k]C}"']
	% \end{tikzcd}
	% \end{center}
	% \'{u}nicos de manera que
	% $j_{A\tensor[k]B}(a\tensor b)=a\tensor\inc[B](b)$ y
	% $j_{A\tensor[k]C}(a\tensor c)=a\tensor\inc[C](c)$. Por propiedad
	% universal de la suma directa, existe un \'{u}nico mofismo $g$ que hace
	% conmutar el diagrama siguiente:
	% \begin{center}
	% \begin{tikzcd}
		% A\tensor[k]B \arrow[r,"{\inc[{A\tensor[k]B}]}"]
			% \arrow[dr,"j_{A\tensor[k]B}"'] &
		% \big(A\tensor[k]B\big)\,\oplus\,\big(A\tensor[k]C\big)
			% \arrow[d,dashed,"g"] &
		% A\tensor[k]C \arrow[l,"{\inc[{A\tensor[k]C}]}"']
			% \arrow[dl,"j_{A\tensor[k]X}"] \\
		% & A\tensor[k]\big(B\,\oplus\,C\big) &
	% \end{tikzcd}
	% \end{center}
	% Ahora bien, los morfismos $f$ y $g$ cumplen que
	% \begin{align*}
		% g\circ h(a,b+c) & \,=\,a\tensor\inc[B](b)\,+\,
			% a\tensor\inc[C](c) \,=\,a\tensor(b+c) \,=\,
			% \tensor(a,(b+c)) \text{ ,} \\
		% f\circ\ j_{A\tensor[k]B}\circ\tensor(a,b) & \,=\,
			% f(a\tensor\inc[B](b))\,=\,h(a,\inc[B](b))\,=\,
			% \inc[{A\tensor[k]B}](a\tensor b) \quad\text{y} \\
		% f\circ\ j_{A\tensor[k]C}\circ\tensor(a,c) & \,=\,
			% f(a\tensor\inc[C](c))\,=\,h(a,\inc[C](c))\,=\,
			% \inc[{A\tensor[k]C}](a\tensor c)
		% \text{ .}
	% \end{align*}
	% %
	% Por la parte de unicidad de las propiedades universales,
	% $f\circ g=\id$ y $g\circ f=\id$.
% \end{proof}

\begin{propoProductoTensorialDeModulos}%
	\label{propo:productotensorialdemoduloslibres}
	Dados conjuntos $X$ e $Y$, sean $k^X$, $k^Y$ y $k^{X\times Y}$ los
	$k$-m\'{o}dulos libres en los conjuntos $X$, $Y$ y $X\times Y$.
	Existe una \'{u}nica transformaci\'{o}n bilineal
	$h_0:\,k^X\times k^Y\rightarrow k^{X\times Y}$ tal que
	$h_0(x,y)=(x,y)$. Esta transformaci\'{o}n es universal entre todas
	las transformaciones bilineales con dominio en $k^X\times k^Y$. En
	consecuencia,
	\begin{align*}
		k^{X\times Y} & \,\simeq\, k^X\tensor[k]k^Y
		\text{ .}
	\end{align*}
	%
	El isomorfismo est\'{a} dado en la base por $(x,y)\mapsto x\tensor y$.
\end{propoProductoTensorialDeModulos}

\begin{obsProductoTensorialDeModulos}%
	\label{obs:productotensorialdemodulosesfuntorial}
	Dados morfismo de $k$-m\'{o}dulos $f:\,A\rightarrow A'$ y
	$g:\,B\rightarrow B'$, la funci\'{o}n
	$A\times B\rightarrow A'\tensor[k]B'$, definida por
	$(a,b)\mapsto f(a)\tensor g(b)$ es $k$-bilineal. Por lo tanto, existe
	un \'{u}nico morfismo de $k$-m\'{o}dulos
	\begin{equation}
		\label{eq:productotensorialdemorfismosdemodulos}
		f\tensor g \,:\,A\tensor[k]B\,\rightarrow\,A'\tensor[k]B'
	\end{equation}
	%
	que cumple $(f\tensor g)(a\tensor b)=f(a)\tensor g(b)$. Se verifica,
	por unicidad, que
	\begin{align*}
		\id[A]\tensor\id[B] & \,=\,\id[{A\tensor[k]B}] \quad\text{y} \\
		(f'\tensor g')\circ (f\tensor g) & \,=\,
			(f'\circ f)\tensor (g'\circ g)
		\text{ .}
	\end{align*}
	%
	En particular, dado un $k$-m\'{o}dulo $C$, obtenemos un funtor
	$-\tensor[k]C:\,\Mod[k]\rightarrow\Mod[k]$ dado por
	\begin{align*}
		(\varphi:\,A\rightarrow B) & \,\mapsto\,
			\big(\varphi\tensor\id[C]:\,
				A\tensor[k]C\rightarrow B\tensor[k]C\big)
		\text{ .}
	\end{align*}
	%
\end{obsProductoTensorialDeModulos}

De ahora en adelante, escribiremos $A\tensor B$ en lugar de $A\tensor[k]B$.

\begin{obsAlgebra}\label{obs:algebra}
	En una $k$-\'{a}lgebra $A$, la multiplicaci\'{o}n es una operaci\'{o}n
	$k$-bilineal $A\times A\rightarrow A$ y determina un\'{\i}vocamente un
	morfismo $\mu_A:\,A\tensor A\rightarrow A$ y todo morfismo de este tipo
	define, por composici\'{o}n con
	$\tensor:\,A\times A\rightarrow A\tensor A$, una transformaci\'{o}n
	$k$-bilineal. Usando esta correspondencia, podemos dar una
	definici\'{o}n alternativa de $k$-\'{a}lgebras.
\end{obsAlgebra}

\begin{defAlgebra}\label{def:algebraalternativa}
	Una $k$-\'{a}lgebra es un $k$-m\'{o}dulo $A$ junto con morfismos de
	$k$-m\'{o}dulos $\unidad[A]:\,k\rightarrow A$ --la unidad de $A$-- y
	$\producto[A]:\,A\tensor A\rightarrow A$ --el producto en $A$-- que
	verifican que los siguientes diagramas conmutan.
	\begin{center}
		\begin{tikzcd}
			A\tensor A\tensor A
				\arrow[r,"{\producto[A]\tensor\id[A]}"]
				\arrow[d,"{\id[A]\tensor\producto[A]}"'] &
			A\tensor A \arrow[d,"{\producto[A]}"] \\
			A\tensor A\arrow[r,"{\producto[A]}"'] & A
		\end{tikzcd}
		\begin{tikzcd}
			k\tensor A\arrow[r,"{\unidad[A]\tensor\id[A]}"]
				\arrow[dr,"\sim"'] &
			A\tensor A \arrow[d,"{\producto[A]}"] &
			A\tensor k\arrow[l,"{\id[A]\tensor\unidad[A]}"']
				\arrow[dl,"\sim"] \\ & A &
		\end{tikzcd}
	\end{center}
	Una $k$-\'{a}lgebra es conmutativa, si y s\'{o}lo si
	\begin{align*}
		\producto[A]\circ\swap & \,=\,\producto[A]
		\text{ .}
	\end{align*}
	%
	Un morfismo de $k$-\'{a}lgebras es un morfismo de $k$-m\'{o}dulos
	$f:\,A\rightarrow B$ tal que
	\begin{align*}
		f\circ\producto[A] \,=\,\producto[B]\circ (f\tensor f)
		% en particular, $f\circ\mu_A$ es el morfismo de \'{a}lgebras
		% $A\tensor A\rightarrow B$ inducido por la propiedad de
		% producto tensorial (si $B$ es conmutativa\dots)
		%
			& \quad\text{y}\quad
		f\circ\unidad[A] \,=\,\unidad[B]
		\text{ .}
	\end{align*}
	%
\end{defAlgebra}

\begin{obsAlgebra}\label{obs:algebraopuesta}
	Si $(A,\producto[A],\unidad[A])$ es una $k$-\'{a}lgebra, su
	\emph{\'{a}lgebra opuesta} es $(A,\producto[A]^\opp,\unidad[A])$, donde
	\begin{align*}
		\producto[A]^\opp & \,=\,\producto[A]\circ\swap
		\text{ .}
	\end{align*}
	%
	Denotamos este \'{a}lgebra por $A^\opp$. El \'{a}lgebra $A$ es
	conmutativa, si $A^\opp=A$, es decir, si el morfismo de m\'{o}dulos
	$\id[A]:\,A\rightarrow A^\opp$ es morfimo de \'{a}lgebras.
\end{obsAlgebra}

\subsubsection{Producto tensorial de \'{a}lgebras}

Definimos el producto tensorial de \'{a}lgebras y enunciamos dos propiedades
importantes.

\begin{propoProductoTensorialDeAlgebras}%
	\label{propo:productotensorialdealgebras}
	Sean $A$ y $B$ dos $k$-\'{a}lgebras (no necesariamente conmutativas) y
	sea $A\tensor B$ el producto tensorial como $k$-m\'{o}dulos. La
	operaci\'{o}n $k$-bilineal definida por
	\begin{align*}
		(a\tensor b)\,(a_1\tensor b_1) & \,=\,(a\,a_1)\tensor (b\,b_1)
	\end{align*}
	%
	(productos en $A$ y en $B$) determina una estructura de $k$-\'{a}lgebra
	en $A\tensor B$. La unidad est\'{a} dada por $1\tensor 1$ y los
	morfismos $j_A:\,A\rightarrow A\tensor B$ y
	$j_B:\,B\leftarrow A\tensor B$, dados por $a\mapsto a\tensor 1$ y
	$b\mapsto 1\tensor b$, son morfismos de \'{a}lgebras.
\end{propoProductoTensorialDeAlgebras}

Los morfismos $j_A$ y $j_B$ de la Proposici\'{o}n~%
\ref{propo:productotensorialdealgebras} cumplen con que, para todo par $a\in A$
y $b\in B$,
\begin{align*}
	j_A(a)\,j_B(b) & \,=\,(a\tensor 1)\,(1\tensor b)\,=\,(a\tensor b)\,=\,
		(1\tensor b)\,(a\tensor 1)\,=\,j_B(b)\,j_A(a)
	\text{ .}
\end{align*}
%
% en $A\tensor B$, es decir,
% \begin{align*}
	% \mu_{A\tensor B}\circ (j_A\tensor j_B) & \,=\,
		% \mu_{A\tensor B}\circ (j_B\tensor j_A)\circ\swap[A\tensor B]
	% \text{ .}
% \end{align*}
% %
El \'{a}lgebra $A\tensor B$ est\'{a} caracterizada por la siguiente propiedad
universal.

\begin{propoProductoTensorialDeAlgebras}%
	\label{propo:productotensorialdealgebrasuniversal}
	Dada un \'{a}lgebra $C$ y morfimsos $f:\,A\rightarrow C$ y
	$g:\,B\rightarrow C$ que verifican $f(a)g(b)=g(b)f(a)$ en $C$ para todo
	par $a\in A$ y $b\in B$, existe un \'{u}nico morfismo de \'{a}lgebras
	$f\cdot g:\,A\tensor B\rightarrow C$ tal que
	\begin{equation}
		\label{eq:productotensorialdemorfismosdealgebras}
		(f\cdot g)\circ j_A\,=\,f \quad\text{y}\quad
			(f\cdot g)\circ j_B \,=\,g
		\text{ .}
	\end{equation}
	%
\end{propoProductoTensorialDeAlgebras}

\begin{proof}
	Si $\phi:\,A\tensor B\rightarrow C$ es un morfismo de \'{a}lgebras que
	cumple con \eqref{eq:productotensorialdemorfismosdealgebras}, entonces
	\begin{align*}
		\phi (a\tensor b) & \,=\,
			\phi\circ(\mu_{A\tensor B}(j_A(a)\tensor j_B(b))) \\
		& \,=\,\mu_C\circ (\phi\tensor\phi)\circ
				(j_A\tensor j_B)(a\tensor b) \\
		& \,=\,\mu_C\circ (f\tensor g)(a\tensor b)
		\text{ .}
	\end{align*}
	%
	Rec\'{\i}procamente, si $\phi:\,A\tensor B\rightarrow C$ es el morfismo
	de m\'{o}dulos $\phi=\mu_C\circ (f\tensor g)$, entonces como
	$f$ y $g$ son morfismos de \'{a}lgebras, $f(1)=1=g(1)$ y $\phi$ cumple
	\eqref{eq:productotensorialdemorfismosdealgebras}. Resta verificar que,
	si $f$ y $g$ cumplen con las hip\'{o}tesis del enunciado, entonces
	$f\cdot g:=\mu_C\circ (f\tensor g)$ es morfismo de \'{a}lgebras. Pero
	\begin{align*}
		(f\cdot g)(a\,a_1\tensor b\,b_1) & \,=\,f(a\,a_1)\,g(b\,b_1)
			\,=\,f(a)\,f(a_1)\,g(b)\,g(b_1) \\
		& \,=\, f(a)\,g(b)\,f(a_1)\,g(b_1) \\
		& \,=\, (f\cdot g)(a\tensor b)\,(f\cdot g)(a_1\tensor b_1)
		\text{ .}
	\end{align*}
	%
	%
	% La condici\'{o}n $f(a)g(b)=g(b)f(a)$ se puede reescribir de la
	% siguiente manera:
	% \begin{align*}
		% \mu_C\circ(f\tensor g) & \,=\,
			% \mu_C\circ(g\tensor f)\circ\swap[A\tensor B] \,=\,
			% \mu_C\circ\swap[C\tensor C]\circ (f\tensor g)
		% \text{ .}
	% \end{align*}
	% %
	% Por otro lado, la asociatividad de $\mu_C$ implica
	% \begin{align*}
		% \mu_C \circ (\mu_C\tensor\mu_C) & \,=\,
			% \mu_C\circ(\mu_C\tensor\id[C])\circ
			% (\id[C]\tensor\mu_C\tensor\id[C])
		% \text{ .}
	% \end{align*}
	% %
	% Entonces
	% \begin{align*}
		% & \mu_C(\mu_C\,(f\tensor g)\tensor\mu_C\,(f\tensor g)) \,=\,
			% \mu_C\,(\mu_C\,(f\tensor g)\tensor
			% \mu_C\,(g\tensor f)\,\swap[A\tensor B]) \\
		% & \qquad\qquad\,=\,
			% \mu_C\,(\mu_C\tensor\id[C])\,
				% (\id[C]\tensor\mu_C\tensor\id[C])\,
				% (f\tensor g\tensor g\tensor f)\,
				% (\id[A\tensor B]\tensor\swap[A\tensor B]) \\
		% & \qquad\qquad\,=\,
			% \mu_C\,(\mu_C\tensor\id[C])\,
				% (f\tensor(g\,\mu_B)\tensor f)\,
				% (\id[A\tensor B]\tensor\swap[A\tensor B]) \\
		% & \qquad\qquad\,=\,
			% \mu_C\,(\mu_C\tensor\id[C])\,(f\tensor g\tensor f)\,
				% (\id[A]\tensor\mu_B\tensor\id[A])\,
				% (\id[A\tensor B]\tensor\swap[A\tensor B]) \\
		% & \qquad\qquad\,=\,
			% \mu_C\,(\mu_C\tensor\id[C])\,(f\tensor f\tensor g)\,
				% (\id[A]\tensor\swap[B\tensor A])\,
				% (\id[A]\tensor\mu_B\tensor\id[A])\,
				% (\id[A\tensor B]\tensor\swap[A\tensor B]) \\
		% & \qquad\qquad\,=\,
			% \mu_C\,(f\tensor g)\,(\mu_A\tensor\id[B])\,
				% (\id[A\tensor A]\tensor\mu_B)\,
				% (\id[A]\tensor\swap[B\tensor A]\tensor\id[B])
				% \\
		% & \qquad\qquad\,=\,
			% \mu_C\,(f\tensor g)\,(\mu_A\tensor\mu_B)\,
				% (\id[A]\tensor\swap[B\tensor A]\tensor\id[B])
	% \end{align*}
	% %
\end{proof}

En particular, si $C$ es conmutativa, existe una biyecci\'{o}n natural
\begin{equation}
	\label{eq:morfismosdesdeelproductotensorialdealgebras}
	\Homalg\big(A,C\big) \,\times\,\Homalg\big(B,C\big) \,\simeq\,
		\Homalg\big(A\tensor B,C\big)
\end{equation}
%
dado por $(f,g)\mapsto \mu_C\circ(f\tensor g)$. En otras palabras, el producto
tensorial (junto con los morfismos de la Proposici\'{o}n~%
\ref{propo:productotensorialdealgebras}) es el \emph{coproducto} en la
categor\'{\i}a $\CommAlg[k]$.

\begin{obsProductoTensorialDeAlgebras}%
	\label{obs:productotensorialdealgebrasejemplomodulos}
	En el Ejemplo~\ref{ejemplo:productotensorialdemodulos}, vimos que
	$A\simeq k\tensor A$ como $k$-m\'{o}dulos, naturalmente, v\'{\i}a
	$a\mapsto 1\tensor a$. Pero este morfismo coincide con el morfismo de
	\'{a}lgebras $j_A:\,A\rightarrow k\tensor A$ de la definici\'{o}n de
	producto tensorial de \'{a}lgebras (Proposici\'{o}n~%
	\ref{propo:productotensorialdealgebras}). Teniendo esto en cuenta
	idenitficamos naturalmente $k\tensor A\simeq A\simeq A\tensor k$ como
	\'{a}lgebras.
\end{obsProductoTensorialDeAlgebras}

\begin{obsProductoTensorialDeAlgebras}%
	\label{obs:productotensorialdealgebrasesfuntorial}
	Sea $\varphi:\,A\rightarrow B$ morfismo de $k$-\'{a}lgebras y sea $C$
	otra $k$-\'{a}lgebra. Por la Observaci\'{o}n~%
	\ref{obs:productotensorialdemodulosesfuntorial}, existe un \'{u}nico
	morfismo de m\'{o}dulos
	$\varphi\tensor\id[C]:\,A\tensor C\rightarrow B\tensor C$ tal que
	$\varphi\tensor\id[C](a\tensor c)=\varphi(a)\tensor c$. Ahora bien, por
	la Proposici\'{o}n~\ref{propo:productotensorialdealgebrasuniversal},
	existe un \'{u}nico morfismo de \'{a}lgebras
	$\varphi\cdot\id[C]:\,A\tensor C\rightarrow B\tensor C$ tal que el
	diagrama siguiente conmuta.
	\begin{center}
		\begin{tikzcd}
			A \arrow[d,"\varphi"'] \arrow[r,"j_A"] &
				A\tensor C
				\arrow[d,dashed,"{\varphi\cdot\id[C]}"] &
				C \arrow[l,"j_C"'] \arrow[d,equal] \\
			B \arrow[r,"j_B"'] & B\tensor C &
				C \arrow[l,"j_C"]
		\end{tikzcd}
	\end{center}
	Pero la conmutatividad de este diagrama quiere decir que
	\begin{align*}
		\varphi\cdot\id[C](a\tensor c) & \,=\,
			\varphi\cdot\id[C](j_A(a)\,j_C(c)) \,=\,
			j_B(\varphi(a))\,j_C(\id[C](c)) \\
		& \,=\, \varphi(a)\tensor c
		\text{ .}
	\end{align*}
	%
	Es decir, el morfismo de m\'{o}dulos $\varphi\tensor\id[C]$ es, en
	realidad, morfismo de \'{a}lgebras. En particular, obtenemos un
	funtor $-\tensor C:\,\Alg[k]\rightarrow\Alg[k]$ dado por
	\begin{align*}
		\big(\varphi:\,A\rightarrow B\big) & \,\mapsto\,
			\big(\varphi\tensor\id[C]:\,
				A\tensor C\rightarrow B\tensor C\big)
		\text{ .}
	\end{align*}
	%
\end{obsProductoTensorialDeAlgebras}

\begin{obsProductoTensorialDeAlgebras}\label{obs:productotensorialdealgebras}
	En t\'{e}rminos de los morfismos $\producto[A]$, $\producto[B]$ y
	$\swap:\,A\tensor B\rightarrow B\tensor A$, el producto en $A\tensor B$
	est\'{a} dado por
	\begin{equation}
		\label{eq:productotensorialdealgebras}
		\producto[A\tensor B] \,=\,
			(\producto[A]\tensor\producto[B])\circ
			(\id[A]\tensor\swap\tensor\id[B])
		\text{ .}
	\end{equation}
	%
	En adelante, escribiremos simplemente $\producto[C]\circ (f\tensor g)$
	en lugar de $f\cdot g$ para referirnos al morfismo de \'{a}lgebras
	inducido en el producto tensorial.
\end{obsProductoTensorialDeAlgebras}

Sea $A$ una $k$-\'{a}lgebra. Dados $a,b\in A$, por
\eqref{eq:productotensorialdealgebras}, sabemos que
$(a\tensor 1)\,(1\tensor b)=(1\tensor b)\,(a\tensor 1)$ en $A\tensor A$. Si
$A=k\{X\}/I$, el siguiente resultado muestra c\'{o}mo describir el producto
$A\tensor A$ como cociente de un \'{a}lgebra libre, a partir de la
presentaci\'{o}n de $A$.

\begin{propoProductoTensorialDeAlgebras}%
	\label{propo:productotensorialdealgebrascociente}
	Sea $A=k\{X\}/I$ una $k$-\'{a}lgebra generada por el conjunto $X$.
	Sean $X',X''$ dos \emph{copias} del conjunto $X$ y sean
	$I'\triangleleft k\{X'\}$ e $I''\triangleleft k\{X''\}$ los ideales
	bil\'{a}teros correspondientes a $I$ determinados por identificar
	las copias. Entonces $A\tensor A$ es isomorfa como $k$-\'{a}lgebra a
	\begin{align*}
		A^{\otimes 2} & \,:=\,k\{X'\sqcup X''\}/
			\generado{I',I'',X'X''-X''X'}
		\text{ ,}
	\end{align*}
	%
	donde $X'\sqcup X''$ denota la uni\'{o}n disjunta de las copias de $X$
	y $\generado{I',I'',X'X''-X''X'}$ es el ideal bil\'{a}tero generado por
	$I'$, $I''$ y los elementos de la forma $x'y''-y''x'$ con $x'\in X'$ e
	$y''\in X''$ (no necesariamente correspondientes al mismo elemento de
	$X$).
\end{propoProductoTensorialDeAlgebras}

\begin{proof}
	Definimos morfismos $\varphi',\varphi'':\,A\rightarrow A^{\otimes 2}$
	por $\varphi'(x)=x'$ y $\varphi''(x)=x''$, donde $x'\in X'$ y
	$x''\in X''$ son las copias del elemento $x\in X$. Dado que
	\begin{align*}
		\varphi'(x)\,\varphi''(y) & \,=\,x'\,y''\,\,y''\,x'\,=\,
			\varphi''(y)\,\varphi'(x)
		\text{ ,}
	\end{align*}
	%
	en $A^{\otimes 2}$, para todo par $x,y\in X$, por la Proposici\'{o}n~%
	\ref{propo:productotensorialdealgebrasuniversal}, existe un \'{u}nico
	morfismo $\varphi:\,A\tensor A\rightarrow A^{\otimes 2}$ tal que
	\begin{align*}
		\varphi(x\tensor y) & \,=\,\varphi'(x)\,\varphi''(y)\,=\,
			x'\,y''
	\end{align*}
	%
	para todo par $x,y\in X$. En la direcci\'{o}n opuesta, definimos
	$\psi:\,k\{X'\sqcup X''\}\rightarrow A\tensor A$ por
	\begin{align*}
		\psi(x') \,=\,x\tensor 1 & \quad\text{y}\quad
		\psi(y'') \,=\,1\tensor y
		\text{ ,}
	\end{align*}
	%
	si $x'$ es copia de $x$ e $y''$ es copia de $y$. Pero este morfismo de
	\'{a}lgebras se anula en el ideal $\generado{I',I'',X'X''-X''X'}$. Por
	lo tanto, determina un\'{\i}vocamente un morfismo
	$\psi:\,A^{\otimes 2}\rightarrow A\tensor A$. Se puede ver que
	$\varphi$ y $\psi$ son inversos uno de otro.
\end{proof}

Con este \'{u}ltimo resultado, podemos ver que el ``producto de matrices''
$\Delta:\,\MM(2)\rightarrow\MM(2)\tensor\MM(2)$ est\'{a} dado por
\begin{align*}
	\Delta\,\begin{bmatrix} a & b \\ c & d \end{bmatrix} & \,=\,
		\begin{bmatrix} a & b \\ c & d \end{bmatrix}\tensor
		\begin{bmatrix} a & b \\ c & d \end{bmatrix}
	\text{ ,}
\end{align*}
%
es decir,
\begin{align*}
	\Delta(a) \,=\,a\tensor a + b\tensor c & \quad\text{,}\quad
	\Delta(b) \,=\,a\tensor b + b\tensor d \\
	\Delta(c) \,=\,c\tensor a + d\tensor c & \quad\text{y}\quad
	\Delta(d) \,=\,c\tensor b + d\tensor d
	\text{ .}
\end{align*}
%

\begin{obsAlgebraEjemplos}\label{obs:algebraejemplos}
	Sea $H=k[x]$, $k[x,x^{-1}]$, $\GL(2)$ o $\SL(2)$. En la secci\'{o}n
	\ref{subsec:kalgebras:ejemplos}, vimos c\'{o}mo definir un morfismo de
	\'{a}lgebras $\Delta:\,H\rightarrow H^{\otimes 2}$ en cada uno de estos
	casos y afirmamos que el pullback
	\begin{math}
		\pull\Delta:\,\Homalg\big(H^{\otimes 2},A\big)\rightarrow
			\Homalg\big(H,A\big)
	\end{math} permite definir una estructura de grupo en el conjunto de
	morfismos $H\rightarrow A$. La validez de esta afirmaci\'{o}n la
	hab\'{\i}amos visto como consecuencia de ciertas identificaciones
	particulares para cada caso: \eqref{eq:obs:rectaafin},
	\eqref{eq:obs:multiplicativo}. Ahora podemos ver que el argumento es
	v\'{a}lido en general. Supongamos dada un \'{a}lgebra $H$ y un morfismo
	de \'{a}lgebras $\Delta:\,H\rightarrow H^{\otimes 2}$. Entonces, por
	la Proposici\'{o}n~\ref{propo:productotensorialdealgebrascociente},
	\begin{align*}
		\Homalg\big(H^{\otimes 2},A\big) & \,\simeq\,
			\Homalg\big(H\tensor H,A\big)
	\end{align*}
	%
	y, por \eqref{eq:morfismosdesdeelproductotensorialdealgebras}, si $A$
	es conmutativa,
	\begin{align*}
		\Homalg\big(H\tensor H,A\big) & \,\simeq\,
			\Homalg\big(H,A\big)\,\times\,
			\Homalg\big(H,A\big)
		\text{ .}
	\end{align*}
	%
	Por medio de estas biyecciones, obtenemos una funci\'{o}n
	\begin{align*}
		\pull\Delta & \,:\,\Homalg\big(H,A\big)\,\times\,
			\Homalg\big(H,A\big)\,\rightarrow\,\Homalg\big(H,A\big)
	\end{align*}
	%
	dada por
	\begin{align*}
		\pull\Delta(f,g) & \,=\,\mu_A\circ(f\tensor g)\circ\Delta
		\text{ .}
	\end{align*}
	%
	Ahora bien, que esta operaci\'{o}n binaria en el conjunto de morfismos
	sea asociativa, admita un elemento neutro y sea tal que todo morfismo
	posea un inverso con respecto a la misma, depender\'{a} de las
	propiedades del morfismo $\Delta$.
\end{obsAlgebraEjemplos}


\section{Co\'{a}lgebras y bi\'{a}lgebras}
\theoremstyle{plain}
\newtheorem{defCoalgebra}{Definici\'{o}n}[section]
\newtheorem{defBialgebra}[defCoalgebra]{Definici\'{o}n}
\newtheorem{teoIsomorfismoProductoTensorialDeMorfismosDeModulos}[defCoalgebra]%
	{Teorema}
\newtheorem{coroIsomorfismoProductoTensorialDeMorfismosDeModulos}%
	[defCoalgebra]{Corolario}
\newtheorem{propoAlgebraDual}[defCoalgebra]{Proposici\'{o}n}
\newtheorem{propoCoalgebraDual}[defCoalgebra]{Proposici\'{o}n}

\theoremstyle{definition}
\newtheorem{obsCoalgebra}[defCoalgebra]{Observaci\'{o}n}
\newtheorem{ejemploCoalgebra}[defCoalgebra]{Ejemplo}
\newtheorem{ejemploCoalgebraProductoTensorial}[defCoalgebra]{Ejemplo}
\newtheorem{ejemploAlgebraDual}[defCoalgebra]{Ejemplo}
\newtheorem{ejemploCoalgebraDualDeMatrices}[defCoalgebra]{Ejemplo}
\newtheorem{obsPolinomios}[defCoalgebra]{Observaci\'{o}n}
\newtheorem{obsBialgebraDeMatrices}[defCoalgebra]{Observaci\'{o}n}
\newtheorem{ejemploBialgebra}[defCoalgebra]{Ejemplo}

%-------------

\subsection{Co\'{a}lgebras}\label{subsec:coalgebras:coalgebras}

\subsubsection{Definiciones}

En la Definici\'{o}n~\ref{def:algebraalternativa}, dimos una definici\'{o}n de
$k$-\'{a}lgebra en t\'{e}rminos de ciertos diagramas. La idea en la
definici\'{o}n de $k$-co\'{a}lgebra es dar vuelta todas las flechas que
aparecen en aquellos diagramas. Adelant\'{a}ndonos a la secci\'{o}n
\ref{sec:gruposafines}, la noci\'{o}n de co\'{a}lgebra aparece naturalmente,
teniendo en cuenta que el funtor $\Homalg\big(-,-\big)$ es
\emph{contra}variante en el primer lugar: si un grupo af\'{\i}n es un funtor
representable por una $k$-\'{a}lgebra conmutativa, $G=\Homalg\big(H,-\big)$,
junto con (entre otras cosas) una transformaci\'{o}n natural
$m:\,G\times G\xrightarrow\cdot G$ --la multiplicaci\'{o}n en el grupo--,
entonces esta transformaci\'{o}n $m$ deber\'{a} estar inducida por un morfismo
correspondiente $\coproducto:\,H\rightarrow H\tensor H$ (!`en la direcci\'{o}n
opuesta!) y las propiedades de $m$ deber\'{a}n verse reflejadas en propiedades
de $\coproducto$.

\begin{defCoalgebra}\label{def:coalgebra}
	Una $k$-co\'{a}lgebra es un $k$-m\'{o}dulo $C$ junto con morfismos de
	$k$-m\'{o}dulos $\counidad[C]:\,C\rightarrow k$ --la counidad de $C$--
	y $\coproducto[C]:\,C\rightarrow C\tensor C$ --el coproducto en $C$--
	que verifican que los siguientes diagramas conmutan.
	\begin{center}
	\begin{tikzcd}[column sep=large]
		C\tensor C\tensor C & C\tensor C
			\arrow[l,"{\coproducto[C]\tensor\id[C]}"'] \\
		C\tensor C \arrow[u,"{\id[C]\tensor\coproducto[C]}"] &
		C\arrow[l,"{\coproducto[C]}"]\arrow[u,"{\coproducto[C]}"']
	\end{tikzcd}
	\begin{tikzcd}[column sep=large]
		k\tensor C & C\tensor C
			\arrow[l,"{\counidad[C]\tensor\id[C]}"']
			\arrow[r,"{\id[C]\tensor\counidad[C]}"] & C\tensor k \\
		& C \arrow[ul,"\sim"] \arrow[ur,"\sim"']
			\arrow[u,"{\coproducto[C]}"'] &
	\end{tikzcd}
	\end{center}
	Una $k$-co\'{a}lgebra es coconmutativa, si
	\begin{align*}
		\swap\circ\coproducto[C] & \,=\,\coproducto[C]
		\text{ .}
	\end{align*}
	%
	Un morfismo de $k$-co\'{a}lgebras es un morfismo de $k$-m\'{o}dulos
	$f:\,C\rightarrow D$ tal que
	\begin{align*}
		\coproducto[D]\circ f \,=\,(f\tensor f)\circ\coproducto[C]
			& \quad\text{y}\quad
		\counidad[D]\circ f \,=\,\counidad[C]
		\text{ .}
	\end{align*}
\end{defCoalgebra}

\begin{obsCoalgebra}\label{obs:coalgebraopuesta}
	Si $(C,\coproducto[C],\counidad[C])$ es una $k$-co\'{a}lgebra, su
	\emph{co\'{a}lgebra opuesta} es $(C,\coproducto[C]^\opp,\counidad[C])$,
	donde
	\begin{align*}
		\coproducto[C]^\opp & \,=\,\swap\circ\coproducto[C]
		\text{ .}
	\end{align*}
	%
	Dentamos esta co\'{a}lgebra por $C^\copp$. La co\'{a}lgebra $C$ es
	coconmutativa, si $C^\copp=C$, es decir, si el morfismo de m\'{o}dulos
	$\id[C]:\,C\rightarrow C^\copp$ es morfismo de co\'{a}lgebras.
\end{obsCoalgebra}

\subsubsection{Ejemplos}

\begin{ejemploCoalgebra}\label{ejemplo:coalgebraanillodebase}
	El $k$-m\'{o}dulo $k$ es el $k$-m\'{o}dulo libre generado por el
	elemento $1$. Las expresiones
	\begin{align*}
		\coproducto(1) \,=\,1\tensor 1 & \quad\text{y}\quad
			\counidad(1)\,=\,1
	\end{align*}
	%
	determinan un\'{\i}vocamente morfismos de $k$-m\'{o}dulos
	$\coproducto:\,k\rightarrow k\tensor k$ y $\counidad:\,k\rightarrow k$.
	Estos morfismos dan a $k$ una estructura de co\'{a}lgebra que
	denominaremos ``estructura usual de co\'{a}lgebra en $k$''. Nos
	estaremos refiriendo a esta estructura, si usamos la notaci\'{o}n
	$\coproducto[k]$ o $\counidad[k]$. Con esta estructura, dada una
	co\'{a}lgebra $(C,\coproducto[C],\counidad[C])$, el morfismo de
	m\'{o}dulos $\counidad[C]:\,C\rightarrow k$ es morfismo de
	co\'{a}lgebras.
\end{ejemploCoalgebra}

\begin{ejemploCoalgebra}\label{ejemplo:coalgebradeconjunto}
	Sea $X$ un conjunto y sea $k[X]$ el $k$-m\'{o}dulo libre con base $X$.
	Definimos morfismos de m\'{o}dulos
	$\coproducto:\,k[X]\rightarrow k[X]\tensor k[X]$ y
	$\counidad:\,k[X]\rightarrow k$ dando sus valores en los generadores:
	\begin{align*}
		\coproducto(x) \,=\, x\tensor x & \quad\text{y}\quad
			\counidad(x)\,=\,1
		\text{ .}
	\end{align*}
	%
	La terna $(k[X],\coproducto,\counidad)$ es una co\'{a}lgebra: como los
	diagramas en la Definici\'{o}n~\ref{def:coalgebra} son diagramas de
	$k$-m\'{o}dulos, basta notar que, para todo generador $x\in X$, se
	cumple
	\begin{align*}
		(\coproducto\tensor\id)\circ\coproducto(x) & \,=\,
			(x\tensor x)\tensor x \,=\,
			x\tensor (x\tensor x) \,=\,
			(\id\tensor\coproducto)\circ\coproducto(x) \text{ ,} \\
		(\counidad\tensor\id)\circ\coproducto(x) & \,=\,
			1\tensor x \,\sim\, x \quad\text {y} \\
		(\id\tensor\counidad)\circ\coproducto(x) & \,=\,
			x\tensor 1 \,\sim\,x
	\end{align*}
	%
	El Ejemplo~\ref{ejemplo:coalgebraanillodebase} es un caso particular de
	esta construcci\'{o}n.
\end{ejemploCoalgebra}

\begin{ejemploCoalgebra}\label{ejemplo:coalgebradepolinomios}
	Sea $k[x]$ el \'{a}lgebra de polinomios en una indeterminada. Sean
	$\coproducto:\,k[x]\rightarrow k[x]\tensor k[x]$ y
	$\counidad:\,k[x]\rightarrow k$ los morfismos \emph{de \'{a}lgebras}
	dados en el generador $x$ por
	\begin{align*}
		\coproducto(x) \,=\,x\tensor 1 + 1\tensor x
			& \quad\text{y}\quad
		\counidad(x) \,=\,0
		\text{ .}
	\end{align*}
	%
	En particular, $\coproducto$ y $\counidad$ definen, olvid\'{a}ndonos de
	la estructura adicional, morfismos de m\'{o}dulos cuyo dominio es el
	$k$-m\'{o}dulo libre con base en el conjunto $\{1,\,x,\,x^2,\,\dots\}$.
	Por ejemplo,
	\begin{align*}
		\coproducto(1) & \,=\,1\tensor 1 \quad\text{y} \\
		\coproducto(x^2) & \,=\,\coproducto(x)^2 \,=\,
			\big(x\tensor 1 + 1\tensor x\big)^2 \,=\,
			x^2\tensor 1 \,+\, 2\,(x\tensor x) \,+\, 1\tensor x^2
		\text{ .}
	\end{align*}
	%
	Estos morfismos satisfacen
	\begin{align*}
		(\coproducto\tensor\id)\circ\coproducto(x) & \,=\,
			(x\tensor 1)\tensor 1 + x \tensor (1\tensor 1) \,=\,
			x\tensor (1\tensor 1) + (x\tensor 1)\tensor 1 \\
		& \,=\, (\id\tensor\coproducto)\circ\coproducto(x) \text{ ,} \\
		(\counidad\tensor\id)\circ\coproducto(x) & \,=\,
			0\tensor 1 + 1\tensor x \,=\, 1\tensor x
			\quad\text{y} \\
		(\id\tensor\counidad)\circ\coproducto(x) & \,=\,
			x\tensor 1 + 1\tensor 0 \,=\, x\tensor 1
		\text{ .}
	\end{align*}
	%
	Para demostrar que los diagramas de la Definici\'{o}n~%
	\ref{def:coalgebra} conmutan y que $(k[x],\coproducto,\counidad)$
	es una co\'{a}lgebra, necesitamos, en principio, demostrar, al menos,
	que estas igualdades son v\'{a}lidas reemplazando $x$ por cualquier
	otra potencia, ya que los morfismos en la definici\'{o}n de
	co\'{a}lgebra son morfismos de m\'{o}dulos. Pero, por definici\'{o}n,
	$\coproducto$ y $\counidad$ son morfismos de \'{a}lgebras. Esto implica
	que $\id\tensor\coproducto$, $\coproducto\tensor\id$,
	$\counidad\tensor\id$ e $\id\tensor\counidad$ son morfismos de
	\'{a}lgebras, como as\'{\i} tambi\'{e}n las identificaciones
	$k\tensor k[x]\simeq k[x]\simeq k[x]\tensor k$. Asumiendo que esta
	afirmaci\'{o}n es cierta (ver las Observaciones~%
	\ref{obs:productotensorialdealgebrasejemplomodulos} y
	\ref{obs:productotensorialdealgebrasesfuntorial}), basta verificar las
	igualdades
	\begin{align*}
		(\coproducto\tensor\id)\circ\coproducto & \,=\,
			(\id\tensor\coproducto)\circ\coproducto \text{ ,} \\
		(\counidad\tensor\id)\circ\coproducto & \,=\, j
			\quad\text{y} \\
		(\id\tensor\counidad)\circ\coproducto & \,=\, j
		\text{ ,}
	\end{align*}
	%
	para el generador $x$ del \'{a}lgebra $k[x]$ (aqu\'{\i},
	$j:\,k[x]\rightarrow k\tensor k[x]$ es el isomorfismo de \'{a}lgebras
	$x\mapsto 1\tensor x$ y $j:\,k[x]\rightarrow k[x]\tensor k$ es
	$x\mapsto x\tensor 1$). Pero esto ya lo hemos demostrado.
\end{ejemploCoalgebra}

\subsubsection{Producto tensorial de co\'{a}lgebras}

Dadas co\'{a}lgebras $(C,\coproducto[C],\counidad[C])$ y
$(D,\coproducto[D],\counidad[D])$, el producto tensorial de m\'{o}dulos
$C\tensor D$ admite una estructura natural de co\'{a}lgebra de manera que los
morfismos de m\'{o}dulos $p_C:\,C\tensor D\rightarrow C$ y
$p_D:\,C\tensor D\rightarrow D$ dados por
\begin{align*}
	p_C(c\tensor d) \,=\,c\,\counidad[D](d)
		& \quad\text{y}\quad
	p_D(c\tensor d) \,=\,\counidad[C](c)\,d
\end{align*}
%
sean morfismos de co\'{a}lgebras. Definimos
\begin{equation}
	\label{eq:productotensorialdecoalgebras}
	\coproducto[C\tensor D] \,=\,
		(\id[C]\tensor\swap[C\tensor D]\tensor\id[D])\circ
		(\coproducto[C]\tensor\coproducto[D])
		\quad\text{y}\quad
	\counidad[C\tensor D] \,=\,\counidad[C]\tensor\counidad[D]
	\text{ .}
\end{equation}
%
En cuanto a la counidad, estamos identificando $k\tensor k\simeq k$ (como
$k$-m\'{o}dulos):
\begin{align*}
	\counidad[C\tensor D](c\tensor d) & \,=\,
		(\counidad[C]\tensor\counidad[D])(c\tensor d) \,=\,
		\counidad[C](c)\,\counidad[D](d)
\end{align*}
%
(es decir,
\begin{math}
	\counidad[C\tensor D]=
		\producto[k]\circ(\counidad[C]\tensor\counidad[D])
\end{math}). Veamos que, con estas definiciones, $p_C$ y $p_D$ son morfismos de
co\'{a}lgebras: por un lado,
\begin{align*}
	\coproducto[C]\circ p_C(c\tensor d) & \,=\,
		\coproducto[C](c\,\counidad[D](d)) \,=\,
		\coproducto[C](c)\,\counidad[D](d)
\end{align*}
%
y, por otro, si escribimos $\coproducto[C](c)=\sum_{(c)}\,c'\tensor c''$ y
$\coproducto[D](d)=\sum_{(d)}\,d'\tensor d''$, 
\begin{align*}
	& (p_C\tensor p_C)\circ\coproducto[C\tensor D] (c\tensor d) \,=\,
		(p_C\tensor p_C)\Big(\sum_{(c)\,(d)}\,
			c'\tensor d'\tensor c''\tensor d''\Big) \\
	& \qquad\qquad\,=\,
		\sum_{(c)\,(d)}\,c'\,\counidad[D](d')\tensor
			c''\,\counidad[D](d'') \,=\,
		\Big(\sum_{(c)}\,c'\tensor c''\Big)\,\cdot\,
			\counidad[D]\Big(\sum_{(d)}\,d'\counidad[D](d'')
			\Big) \\
	& \qquad\qquad\,=\,
		\coproducto[C](c)\,\counidad[D](d)
	\text{ .}
\end{align*}
%
Entonces $p_C$ respeta coproductos; en cuanto a las counidades,
\begin{align*}
	\counidad[C]\circ p_C(c\tensor d) & \,=\,
		\counidad[C](c\,\counidad[D](d)) \,=\,
		\counidad[C](c)\,\counidad[D](d) \,=\,
		\counidad[C\tensor D](c\tensor d)
	\text{ .}
\end{align*}
%
La verificaci\'{o}n para $p_D$ es an\'{a}loga.

\begin{ejemploCoalgebra}\label{ejemplo:coalgebradeconjuntoproducto}
	Dados conjuntos $X$ e $Y$, el isomorfismo de m\'{o}dulos
	\begin{align*}
		k[X\times Y] & \,\simeq\,k[X]\tensor k[Y]
		\text{ ,}
	\end{align*}
	%
	dado por $(x,y)\mapsto x\tensor y$, es un morfismo de co\'{a}lgebras:
	\begin{align*}
		\coproducto[{k[X\times Y]}](x,y) & \,=\,(x,y)\tensor (x,y)
			\,\mapsto\,(x\tensor y)\tensor (x\tensor y)
			\text{ ,} \\
		\coproducto[{k[X]\tensor k[Y]}](x\tensor y) & \,=\,
			(\id\tensor\swap\tensor\id)
				((x\tensor x)\tensor (y\tensor y))
		\text{ ;}
	\end{align*}
	%
	en cuanto a la counidad,
	\begin{align*}
		\counidad[{k[X\times Y]}](x,y) & \,=\, 1 \,=\,1\cdot 1\,=\,
			(\counidad[{k[X]}]\tensor\counidad[{k[Y]}])(x\tensor y)
		\text{ .}
	\end{align*}
	%
\end{ejemploCoalgebra}

\subsubsection{Dualidad}\label{subsubsec:coalgebras:dualidad}

Sean $U$, $U'$, $V$ y $V'$ cuatro $k$-m\'{o}dulos. Sean $f:\,U\rightarrow U'$ y
$g:\,V\rightarrow V'$ morfismos de m\'{o}dulos. Recordemos que podemos definir
el producto tensorial $f\tensor g:\,U\tensor V\rightarrow U'\tensor V'$ como el
morfismo determinado por
\begin{align*}
	(f\tensor g)(u\tensor v) & \,=\,f(u)\tensor g(v)
	\text{ ,}
\end{align*}
%
en tensores elementales. Esto determina un\'{\i}vocamente un morfismo
\begin{equation}
	\label{eq:productotensorialdemorfismosdemodulos}
	\lambda \,:\,\Hom[k]\big(U,U'\big)\,\tensor\,\Hom[k]\big(V,V'\big)
		\,\rightarrow\,\Hom[k]\big(V\tensor U,U'\tensor V'\big)
	\text{ ,}
\end{equation}
%
de manera que $\lambda(f\tensor g)$ sea el morfismo dado por
$v\tensor u\mapsto f(u)\tensor g(v)$. Notamos que este morfismo incorpora un
intercambio en el orden de $U$ y de $V$. Notamos, tambi\'{e}n, que los grupos
abelianos $\Hom[k]\big(-,-\big)$ son $k$-m\'{o}dulos y el morfismo $\lambda$ es
$k$-lineal.

\begin{teoIsomorfismoProductoTensorialDeMorfismosDeModulos}%
	\label{thm:isomorfismoproductotensorialdemorfismosdemodulos}
	Si, en \eqref{eq:productotensorialdemorfismosdemodulos}, alguno de los
	pares $(U,U')$, $(V,V')$ o $(U,V)$ est\'{a} compuesto por
	$k$-m\'{o}dulos libres f.g., entonces $\lambda$ es un isomorfismo.
\end{teoIsomorfismoProductoTensorialDeMorfismosDeModulos}

\begin{proof}
	Usar que $\prod_i=\bigoplus_i$, si el conjunto de \'{\i}ndices es
	finito y que conmutan con $\Hom[k]$ y con $\tensor$.
\end{proof}

\begin{coroIsomorfismoProductoTensorialDeMorfismosDeModulos}%
	\label{coro:isomorfismoproductotensorialdemorfismosdemodulos}
	\begin{enumerate}
		\item El morfismo de m\'{o}dulos
			\begin{math}
				\lambda:\,\dual U\tensor\dual V\rightarrow
					\dual{(V\tensor U)}
			\end{math} es un isomorfismo, si $U$ o si $V$ son
			libres f.g.
		\item El morfismo de m\'{o}dulos
			\begin{math}
				\lambda:\,V\tensor\dual U\rightarrow
					\Hom[k]\big(U,V\big)
			\end{math} es un isomorfismo, si $U$ o si $V$ son
			libres f.g.
	\end{enumerate}
\end{coroIsomorfismoProductoTensorialDeMorfismosDeModulos}

En el primer caso, $\lambda$ est\'{a} dado por
\begin{align*}
	\lambda(f\tensor g) & \,=\,\big(v\tensor u\mapsto f(u)\,g(v)\big)
\end{align*}
%
y, en el segundo caso, por
\begin{align*}
	\lambda(v\tensor f) & \,=\,\big(u\mapsto f(u)\,v\big)
	\text{ .}
\end{align*}
%

\begin{propoAlgebraDual}\label{propo:algebradual}
	Sea $(C,\coproducto,\counidad)$ una co\'{a}lgebra. Entonces, la terna
	$(\dual C,\dual\coproducto\circ(\lambda\circ\swap),\dual\counidad)$,
	donde $\lambda:\,\dual C\tensor\dual C\rightarrow\dual{(C\tensor C)}$
	es el morfismo \eqref{eq:productotensorialdemorfismosdemodulos}, es un
	\'{a}lgebra.
\end{propoAlgebraDual}

\begin{proof}
	El primer diagrama de la Definici\'{o}n~\ref{def:coalgebra} induce
	\begin{center}
		\begin{tikzcd}[column sep=large]
			\dual C\tensor\dual C\tensor\dual C
				\arrow[r,"{\lambda\circ\swap\tensor\id}"]
				\arrow[d,"{\id\tensor\lambda\circ\swap}"'] &
			\dual{(C\tensor C)}\tensor\dual C
				\arrow[r,dashed,
					"{\dual\coproducto\tensor\dual\id}"]
				\arrow[d,"{\lambda\circ\swap}"'] &
			\dual C\tensor\dual C
				\arrow[d,"{\lambda\circ\swap}"] \\
			\dual C\tensor\dual{(C\tensor C)}
				\arrow[r,"{\lambda\circ\swap}"]
				\arrow[d,dashed,
					"{\dual\id\tensor\dual\coproducto}"'] &
			\dual{\big(C\tensor C\tensor C\big)}
				\arrow[r,"\dual{(\coproducto\tensor\id)}"]
				\arrow[d,"\dual{(\id\tensor\coproducto)}"'] &
			\dual{\big(C\tensor C\big)}
				\arrow[d,"\dual\coproducto"] \\
			\dual C\tensor\dual C
				\arrow[r,"{\lambda\circ\swap}"'] &
			\dual{\big(C\tensor C\big)}
				\arrow[r,"\dual\coproducto"'] &
			\dual C
		\end{tikzcd}
	\end{center}
	Es decir, $\dual\coproducto\circ(\lambda\circ\swap)$ es asociativo. De
	manera an\'{a}loga, mediante dualizando el diagrama de la counidad, se
	deduce que $\dual\counidad$ es una unidad para este producto.
\end{proof}

\begin{propoCoalgebraDual}\label{propo:coalgebradual}
	Sea $(A,\producto,\unidad)$ un \'{a}lgebra. Si $A$ es libre y f.g. como
	$k$-m\'{o}dulo, entonces la terna
	$(\dual A,(\lambda\circ\swap)^{-1}\circ\dual\producto,\dual\unidad)$ es
	una co\'{a}lgebra.
\end{propoCoalgebraDual}

\begin{proof}
	La demostraci\'{o}n es dual a la de la Proposici\'{o}n~%
	\ref{propo:algebradual}. Para poder definir el coproducto, necesitamos
	que $\lambda$ sea un isomorfismo, lo cual es cierto, bajo las
	hip\'{o}tesis del enunciado.
\end{proof}

\subsubsection{Ejemplos}\label{subsubsec:coalgebras:dualidad:ejemplos}

\begin{ejemploAlgebraDual}\label{ejemplo:algebradual}
	Sea $C$ la co\'{a}lgebra de un conjunto $X$. El \'{a}lgebra dual
	$\dual C$ se identifica con el \'{a}lgebra de funciones
	$X\rightarrow k$. Toda funcional $f:\,C\rightarrow k$ est\'{a}
	determinada por sus valores en $X$. Dadas $f,f_1\in\dual C$, el
	producto $\producto(f\tensor f_1)$ es la funcional determinada por
	\begin{align*}
		\producto(f\tensor f_1) (x) & \,=\,
			(\lambda\circ\swap)(f\tensor f_1)(\coproducto(x)) \,=\,
			f(x)\,f_1(x)
		\text{ .}
	\end{align*}
	%
	(El orden del producto del lado derecho, si bien superfluo, es el
	correcto). En cuanto a la unidad, el isomorfismo $\dual k\simeq k$
	est\'{a} dado por identificar $\id[k]$ con $1$. Entonces la unidad de
	$\dual C$ es la funci\'{o}n
	\begin{align*}
		\unidad(1)(x) & \,=\,\id[k](\counidad(x))\,=\,1
		\text{ .}
	\end{align*}
	%
\end{ejemploAlgebraDual}

\begin{ejemploCoalgebraDualDeMatrices}\label{ejemplo:coalgebradualdematrices}
	Sea $A=\MM[n\times n](k)$ el \'{a}lgebra de matrices cuadradas con
	coeficientes en el anillo conmutativo $k$, con el producto e identidad
	usuales. Como $k$-m\'{o}dulo, $A$ es libre con base las matrices
	$E^{ij}$. Sea $\{x_{ij}\}_{i,j}$ la base dual en $\dual A$. La
	estructura de co\'{a}lgebra dual est\'{a} dada por los morfismos
	\begin{align*}
		\coproducto(x_{ij}) \,=\,\sum_{k=1}^n\,x_{ik}\tensor x_{kj}
			& \quad\text{y}\quad
		\counidad(x_{ij}) \,=\,\delta_{ij}
		\text{ .}
	\end{align*}
	%
	Notamos que
	\begin{math}
		\dual\producto(x_{ij})(E^{pq}\tensor E^{rs})=
			x_{ij}(E^{pq}\,E^{rs})=\delta^{qr}\,x_{ij}(E^{ps})=
			\delta^{qr}\delta_{ip}\delta_{js}
	\end{math} y que esto coincide con
	\begin{align*}
		\lambda\circ\swap\Big(\sum_{k=1}^n\,x_{ik}\tensor x_{kj}\Big)
			(E^{pq}\tensor E^{rs}) & \,=\,
			\sum_{k=1}^n\,x_{ik}(E^{pq})\,x_{kj}(E^{rs})
		\text{ .}
	\end{align*}
	%
	En cuanto a la counidad, $\counidad(x_{ij})\in \dual k\simeq k$ y el
	isomorfismo est\'{a} dado por identificar $f\in\dual k$ con $f(1)$.
	Entonces, como $\unidad(1)=I$, la matriz identidad,
	\begin{align*}
		\counidad(x_{ij})(1) & \,=\,x_{ij}(I) \,=\,\delta_{ij}
		\text{ .}
	\end{align*}
	%
\end{ejemploCoalgebraDualDeMatrices}

\subsection{Bi\'{a}lgebras}\label{subsec:coalgebras:bialgebras}

El $k$-m\'{o}dulo $k$ tiene estructura de \'{a}lgebra, dada por el producto y
la unidad del anillo $k$, y  estructura de co\'{a}lgebra como en el Ejemplo~%
\ref{ejemplo:coalgebraanillodebase}. Denotamos, por el momento, estas
estructuras por $(k,\producto,\unidad)$ y por $(k,\coproducto,\counidad)$,
respectivamente. Entonces podemos verificar que se cumplen las siguientes
igualdades:
\begin{align*}
	(\producto\tensor\producto)\circ(\id\tensor\swap\tensor\id)\circ
		(\coproducto\tensor\coproducto)(1\tensor 1) & \,=\,
		1\tensor 1 \,=\,
		\coproducto\circ\producto(1\tensor 1) \text{ ,} \\
	\producto\circ(\counidad\tensor\counidad) (1\tensor 1) & \,=\,
		1 \,=\,\counidad\circ\producto(1\tensor 1) \text{ ,} \\
	(\unidad\tensor\unidad)\circ\coproducto(1) & \,=\, 1\tensor 1\,=\,
		\coproducto\circ\unidad(1) \text{ ,} \\
	\counidad\circ\unidad(1) \,=\,1\,=\,\counidad(1) & \quad\text{y}\quad
	\unidad\circ\counidad(1) \,=\,1\,=\,\unidad(1)
	\text{ .}
\end{align*}
%
Las primeras dos ecuaciones implican que $\producto:\,k\tensor k\rightarrow k$
es morfismo de co\'{a}lgebras. An\'{a}logamente, la primera y la tercera
implican que $\coproducto:\,k\rightarrow k\tensor k$ es morfismo de
\'{a}lgebras. Tambi\'{e}n se comprueba que $\unidad$ es morfismo de
co\'{a}lgebras y que $\counidad$ es morfismo de \'{a}lgebras. Es decir,
$(k,\producto,\unidad)$ y $(k,\coproducto,\counidad)$ son compatibles y
forman lo que se llama una bi\'{a}lgebra.

\subsubsection{Definiciones}

\begin{defBialgebra}\label{def:bialgebra}
	Sea $B$ un $k$-m\'{o}dulo con estructuras de \'{a}lgebra
	$(B,\producto,\unidad)$ y de co\'{a}lgebra $(B,\coproducto,\counidad)$.
	Entonces $B$ se dice \emph{$k$-bi\'{a}lgebra}, si se cumple cualquiera
	de las dos condiciones equivalentes siguientes:
	\begin{itemize}
		\item $\producto$ y $\unidad$ son morfismos de co\'{a}lgebras;
		\item $\coproducto$ y $\counidad$ son morfismos de
			\'{a}lgebras.
	\end{itemize}
	%
	Un morfismo de bi\'{a}lgebras es un morfismo de m\'{o}dulos que es
	morfismo de \'{a}lgebras y co\'{a}lgebras.
\end{defBialgebra}

Si $(B,\producto,\unidad,\coproducto,\counidad)$ es una bi\'{a}lgebra,
\begin{align*}
	B^\opp & \,:=\,(B,\producto^\opp,\unidad,\coproducto,\counidad)
		\text{ ,} \\
	B^\copp & \,:=\,(B,\producto,\unidad,\coproducto^\opp,\counidad)
		\quad\text{y} \\
	B^{\opp\,\copp} & \,:=\,
		(B,\producto^\opp,\unidad,\coproducto^\opp,\counidad)
\end{align*}
%
son bi\'{a}lgebras. Por ejemplo, si queremos ver que $B^\opp$ es bi\'{a}lgebra,
tenemos que verificar que $\producto^\opp:\,B\tensor B\rightarrow B$ es
morfismo de co\'{a}lgebras, es decir, que se cumple
\begin{align*}
	(\producto^\opp\tensor\producto^\opp)\circ\coproducto[\tensor] & \,=\,
		\coproducto\circ\producto^\opp \quad\text{y} \\
	\counidad\circ\producto^\opp & \,=\,\counidad[\tensor]
	\text{ ,}
\end{align*}
%
donde
\begin{math}
	\coproducto[\tensor]=(\id\tensor\swap\tensor\id)\circ
		(\coproducto\tensor\coproducto)
\end{math}. El lado derecho de la primera igualdad es igual a
\begin{align*}
	\coproducto\circ\producto\circ\swap & \,=\,
		(\producto\tensor\producto)\circ\coproducto[\tensor]\circ\swap
	\text{ ,}
\end{align*}
%
porque $\producto$ es morfismo de co\'{a}lgebras, mientras que el lado
izquierdo es igual a
\begin{align*}
	(\producto\tensor\producto)\circ(\swap\tensor\swap)\circ
		\coproducto[\tensor]
	\text{ .}
\end{align*}
%
Ser\'{a} suficiente demostrar que
\begin{align*}
	(\swap\tensor\swap)\circ(\id\tensor\swap\tensor\id)\circ
		(\coproducto\tensor\coproducto) & \,=\,
	(\id\tensor\swap\tensor\id)\circ(\coproducto\tensor\coproducto)\circ
		\swap
	\text{ .}
\end{align*}
%
Evaluando el lado derecho en un tensor elemental $x\tensor y$, se obtiene
\begin{align*}
	& x\tensor y \,\mapsto\, y\tensor x\,\mapsto\,
		\Big(\sum_{(y)}\,y'\tensor y''\Big)\tensor
			\Big(\sum_{(x)}\,x'\tensor x''\Big)\,\mapsto\,
		\sum_{(y)\,(x)}\,y'\tensor x'\tensor y''\tensor x''
	\text{ ,}
\end{align*}
%
pero si evaluamos el lado izquierdo,
\begin{align*}
	& x\tensor y\,\mapsto\, \Big(\sum_{(x)}\,x'\tensor x''\Big)\tensor
		\Big(\sum_{(y)}\,y'\tensor y''\Big)\,\mapsto\,
		\sum_{(x)\,(y)}\,x'\tensor y'\tensor x''\tensor y''\\
	& \qquad\qquad\,\mapsto\,
		\sum_{(x)\,(y)}\,y'\tensor x'\tensor y''\tensor x''
	\text{ .}
\end{align*}
%
Para ver que $\producto^\opp$ respeta la counidad, como $k$ es conmutativo, se
comprueba que
\begin{align*}
	\counidad\circ\producto^\opp & \,=\,\counidad[\tensor]\circ\swap\,=\,
		\producto[k]\circ(\counidad\tensor\counidad)\circ
			\swap[B\tensor B] \,=\,
		\producto[k]\circ\swap[k\tensor k]\circ
			(\counidad\tensor\counidad) \\
	& \,=\, \producto[k]\circ(\counidad\tensor\counidad)\,=\,
		\counidad[\tensor]
	\text{ .}
\end{align*}
%

\subsubsection{Ejemplos}

\begin{ejemploBialgebra}\label{ejemplo:bialgebramatrices}
	Sea $\MM(m)=k[\lista[11]{x}{mm}]$ el \'{a}lgebra de polinomios en $m^2$
	variables. Si definimos
	\begin{align*}
		\coproducto(x_{ij}) \,=\,\sum_k\,x_{ik}\tensor x_{kj}
			& \quad\text{y}\quad
		\counidad(x_{ij}) \,=\,\delta_{ij}
		\text{ ,}
	\end{align*}
	%
	entonces $\coproducto$ y $\counidad$ se extienden de manera \'{u}nica
	como \emph{morfismos de \'{a}lgebras}
	$\coproducto:\,\MM(m)\rightarrow\MM(m)\tensor\MM(m)$ y
	$\counidad:\,\MM(m)\rightarrow k$. Entonces, por c\'{o}mo fueron
	definidos, $\MM(m)$, junto con $\coproducto$ y $\counidad$ es una
	bi\'{a}lgebra (lo \'{u}nico que hay que verificar es que
	$\coproducto$ y $\counidad$ dan una estructura de co\'{a}lgebra, la
	compatibilidad con el \'{a}lgebra polinomial es autom\'{a}tica).
\end{ejemploBialgebra}

\begin{ejemploBialgebra}\label{ejemplo:bialgebramonoide}
	Sea $X$ un monoide con producto $\producto:\,X\times X\rightarrow X$ y
	elemento neutro $e\in X$ y sea $k[X]$ la co\'{a}lgebra en el conjunto
	subyacente a $X$ (c.f. el Ejemplo~\ref{ejemplo:coalgebradeconjunto}).
	Le damos al $k$-m\'{o}dulo $k[X]$ (libre en $X$) una estructura de
	\'{a}lgebra con producto y unidad
	\begin{align*}
		x\tensor y\,\mapsto\,\producto(x,y) & \quad\text{y}\quad
			\unidad(1_k) \,=\,e\in X
		\text{ .}
	\end{align*}
	%
	Contamos, entonces, con estructuras de \'{a}lgebra y de co\'{a}lgebra
	en $k[X]$. Pero, como $\producto(x\tensor y)=\producto(x,y)\in X$, si
	$x,y\in X$, deducimos que
	\begin{align*}
		\coproducto(\producto(x,y)) & \,=\,
			\producto(x,y)\tensor\producto(x,y) \,=\,
			\producto[\tensor]((x\tensor x)\tensor (y\tensor y))
				\,=\,
			\producto[\tensor](\coproducto(x)\tensor\coproducto(y))
				\text{ ,} \\
		\counidad(\producto(x,y)) & \,=\,1\,=\,
			\producto[k](\counidad(x)\tensor\counidad(y))
				\text{ ,} \\
		\coproducto(\unidad(1)) & \,=\,\coproducto(e)\,=\,e\tensor e
			\,=\,\unidad\tensor\unidad(1) \text{ ,} \\
		\counidad\circ\unidad(1) & \,=\,\counidad(e) \,=\,1 \,=\,
			\unidad[k](1)
		\text{ ,}
	\end{align*}
	%
	de lo que se deduce que $\coproducto$ y $\counidad$ son morfismos de
	\'{a}lgebras.
\end{ejemploBialgebra}


\section{\'{A}lgebras de Hopf}
\theoremstyle{plain}
\newtheorem{defAntipoda}{Definici\'{o}n}[section]
\newtheorem{obsHopf}[defAntipoda]{Observaci\'{o}n}
\newtheorem{teoGrupoDeMorfismos}[defAntipoda]{Teorema}

%-------------

\subsection{La ant\'{\i}poda}

\begin{frame}{Convoluci\'{o}n y ant\'{\i}poda}
	Sean $(A,\mu,\eta)$, $(C,\Delta,\varepsilon)$. La
	\emph{convoluci\'{o}n} de $f,g\in\Hom[k](C,A)$, es
	\begin{align*}
		f \convol g & \,:=\,\mu\circ(f\tensor g)\circ\Delta
			\,\in\,\Hom[k](C,A)
		\text{ .}
	\end{align*}
	%
	Si $\Delta(x)=\sum_i\,x_i'\tensor x_i''$,
	$f \convol g(x) =\sum_i\,f(x_i')\,g(x_i'')$
	\begin{defAntipoda}\label{def:antipoda}
		Una \emph{ant\'{\i}poda} en $(H,\mu,\eta,\Delta,\varepsilon)$
		es $S\in\Endo[k](H)$ tal que
		\begin{align*}
			S\convol\id[H] & \,=\,\id[H]\convol S \,=\,
				\eta\circ\varepsilon
			\text{ .}
		\end{align*}
		%
		\begin{math}
			\sum_i\,x_i'S(x_i'')=\varepsilon(x)\cdot 1=
				\sum_i\,S(x_i')x_i''
		\end{math}. Si existe, es \'{u}nica.
	\end{defAntipoda}
\end{frame}

\begin{frame}{Definici\'{o}n}
	Un \emph{\'{a}lgebra de Hopf} es una bi\'{a}lgebra $H$ con
	ant\'{\i}poda. Un morfismo de \'{a}lgebras de Hopf es un morfismo de
	bi\'{a}lgebras.
	\begin{obsHopf}\label{obs:hopf}
		\begin{math}
			S:\,H\rightarrow H^{\opp\,\copp}=
				(H,\mu^\opp,\eta,\Delta^\opp,\varepsilon)
		\end{math} es morfismo de bi\'{a}lgebras.
		Si $H=k\{X\}/I$ es bi\'{a}lgebra, dado un morfismo de
		\'{a}lgebras $S:\,H\rightarrow H^\opp$, basta verificar la
		condici\'{o}n de ant\'{\i}poda en $X$.
	\end{obsHopf}
\end{frame}

\begin{frame}{El \'{a}lgebra de un grupo}
	Sean $G$ un grupo, $k[G]$ la bi\'{a}lgebra del monoide.
	\begin{align*}
		S(g) & \,=\,g^{-1}
	\end{align*}
	%
	$g\in G$, define una ant\'{\i}poda: $\Delta(g)=g\tensor g$ y
	$\varepsilon(g)=1$. Rec\'{\i}procamente, si $G$ es monoide y
	$S:\,k[G]\rightarrow k[G]$ es ant\'{\i}poda,
	\begin{align*}
		g\,S(g) & \,=\,S(g)\,g \,=\,\varepsilon(g)\,1\,=\,1
	\end{align*}
	%
	implica $S(g)\in G$ y es inverso de $g$.
\end{frame}

\subsection{Ejemplos}

\begin{frame}{$\GL(2)$ y $\SL(2)$}
	$\GL(2)$ y $\SL(2)$ son bi\'{a}lgebras conmutativas con
	$\Delta,\varepsilon$ dados por
	\begin{align*}
		\Delta\,\begin{bmatrix} a & b \\ c & d \end{bmatrix} \,=\,
			\begin{bmatrix} a & b \\ c & d \end{bmatrix} \tensor
			\begin{bmatrix} a & b \\ c & d \end{bmatrix}
			& \quad\text{,}\quad
			\Delta(t)\,=\,t\tensor t \text{ ,} \\
		\varepsilon\,\begin{bmatrix} a & b \\ c & d \end{bmatrix} \,=\,
			\begin{bmatrix} 1 & \\ & 1 \end{bmatrix}
			& \quad\text{,}\quad
			\varepsilon(t) \,=\,1
		\text{ .}
	\end{align*}
	%
	$\Delta$ no es coconmutativa:
	\begin{align*}
		\Delta(a) & \,=\,a\tensor a+b\tensor c\,\not=\,
			a\tensor a+c\tensor b \,=\,\swap\circ\Delta(a)
		\text{ .}
	\end{align*}
	%
	La ant\'{\i}poda est\'{a} dada por
	\begin{align*}
		S\,\begin{bmatrix} a & b \\ c & d \end{bmatrix} \,=\,
			(ad-bc)^{-1}\,
			\begin{bmatrix} d & -b \\ -c & a \end{bmatrix}
			& \quad\text{,}\quad
			S(t) \,=\,t^{-1}
		\text{ .}
	\end{align*}
	%
\end{frame}

\begin{frame}{El grupo de un \'{a}lgebra}
	Dada $(H,\Delta,\varepsilon)$,
	\begin{align*}
		\grouplike H & \,:=\,\big\{x\in H\,:\,x\not=0,\,
			\Delta(x)=x\tensor x\big\}
		\text{ .}
	\end{align*}
	%
	Si $H$ es bi\'{a}lgebra, es monoide con unidad $\Delta(1)=1\tensor 1$ y
	\begin{align*}
		\Delta(xy) & \,=\,\Delta(x)\,\Delta(y)\,=\,
			(x\tensor x)\,(y\tensor y) \,=\,xy\tensor xy
		\text{ .}
	\end{align*}
	%
	Si $H$ es de Hopf, $x\mapsto S(x)$ define un inverso en
	$\grouplike H$:
	\begin{align*}
		\swap\circ(S\tensor S)\circ\Delta & \,=\,
			\Delta\circ S
		\text{ .}
	\end{align*}
	%
	Si $H=k[G]$, entonces $\grouplike{k[G]}=G$.
\end{frame}

\begin{frame}{El grupo $\Homalg\big(H,A\big)$ (cont.)}
	\begin{teoGrupoDeMorfismos}\label{thm:grupodemorfismos}
		Sean $H$ un \'{a}lgebra de Hopf y $A$ un \'{a}lgebra
		conmutativa. Los conjuntos $\Homalg\big(H,A\big)$ son grupos
		con la convoluci\'{o}n heredada de $\Hom[k](H,A)$. El inverso
		de $\psi:\,H\rightarrow A$ est\'{a} dado por $\psi\circ S$.
	\end{teoGrupoDeMorfismos}
	Comprobar que
	\begin{itemize}
		% \item
			% \begin{math}
				% (\psi\convol\varphi)\circ\mu_H =
				% \mu_A\circ((\psi\convol\varphi)\tensor
					% (\psi\convol\varphi))
			% \end{math},
		% \item
			% \begin{math}
				% (\psi\convol\varphi)\circ\eta_H =\eta_A
			% \end{math},
		\item $\psi\convol\varphi\in\Homalg\big(H,A\big)$,
		\item $c =\eta_A\circ\varepsilon_H\in\Homalg\big(H,A\big)$ es
			unidad,
		\item $\psi\circ S\in\Homalg\big(H,A\big)$ es inverso.
	\end{itemize}
	%
\end{frame}

\begin{frame}{El grupo $\Homalg\big(H,A\big)$ (cont.)}
	Si $\psi,\varphi\in\Homalg\big(H,A\big)$, como $H$ es bi\'{a}lgebra,
	\begin{align*}
		(\psi\convol\varphi)\circ\mu_H & \,=\
			\mu_A\circ (\psi\tensor\varphi)\circ\Delta_H\circ\mu_H
				\\
		& \,=\,	\mu_A\,(\psi\tensor\varphi)\,(\mu_H\tensor\mu_H)\,
			(\id[H]\tensor\swap[H]\tensor\id[H])\,
			(\Delta_H\tensor\Delta_H) \\
		& \,=\,\mu_A\,((\mu_A\,(\psi\tensor\varphi)\,\Delta_H)\tensor
			(\mu_A\,(\psi\tensor\varphi)\,\Delta_H)) \\
		& \,=\,\mu_A((\psi\convol\varphi)\tensor(\psi\convol\varphi))
		\text{ .}
	\end{align*}
	%
	Si $c=\eta_A\circ\varepsilon_H$,
	\begin{align*}
		\psi\convol c & \,=\,\mu_A\,(\psi\tensor\eta_A\varepsilon_H)\,
								\Delta_H \\
		& \,=\,\mu_A\,(\id[A]\tensor\eta_A)\,(\psi\tensor\id[k])\,
			(\id[H]\tensor\varepsilon_H)\,\Delta_H \\
		& \,=\,\psi\tensor\id[k]\,=\,\psi
		\text{ .}
	\end{align*}
	%
\end{frame}

\begin{frame}{El grupo $\Homalg\big(H,A\big)$ (cont.)}
	$G=(\Homalg\big(H,A\big),\convol,\eta_A\circ\varepsilon_H)$ es un
	monoide y podemos definir $(k[G],\mu,\eta,\Delta,\varepsilon)$. Sea
	$S_H$ la ant\'{\i}poda en $H$ y sea
	$S(\psi)=\psi\circ S_H\in\Hom[k](H,A)$. Se verifica que
	\begin{align*}
		\mu\circ(\id\tensor S)\circ\Delta & \,=\,
			\eta\circ\varepsilon\,=\,
			\mu\circ(S\tensor\id)\circ\Delta
	\end{align*}
	%
	Por ejemplo, $\eta\circ\varepsilon(\psi)=\eta(1)=\eta_A\varepsilon_H$ y
	\begin{align*}
		\mu\circ(\id\tensor S)\circ\Delta(\psi) & \,=\,
			\mu(\psi\tensor S(\psi)) \,=\,\psi\convol S(\psi) \\
		& \,=\, \mu_A\,(\psi\tensor\psi)\,(\id[H]\tensor S_H)\,
								\Delta_H \\
		& \,=\,\psi\,\mu_H\,(\id[H]\tensor S_H)\,\Delta_H \,=\,
			(\psi\,\eta_H)\,\varepsilon_H \\
		& \,=\,\eta_A\varepsilon_H
		\text{ .}
	\end{align*}
	%
\end{frame}

\begin{frame}{El grupo $\Homalg\big(H,A\big)$ (cont.)}
	Resta verificar que $S(\psi)\in\Homalg\big(H,A\big)$:
	\begin{align*}
		(\psi\,S_H)\,\mu_H & \,=\,\psi\,(\mu_H\,\swap[H])\,
			(S_H\tensor S_H) \,=\,
			\mu_A\,(\psi\tensor\psi)\,\swap[H]\,(S_H\tensor S_H) \\
		& \,=\,(\mu_A\,\swap[A])\,((\psi\,S_H)\tensor (\psi\,S_H))
	\end{align*}
	%
	Si $\mu_A\circ\swap[A]=\mu_A$, entonces $\psi\circ S_H$ respeta
	productos. En cuanto a la unidad,
	\begin{align*}
		(\psi\,S_H)\,\eta_H & \,=\,\psi\,\eta_H\,=\,\eta_H
		\text{ .}
	\end{align*}
\end{frame}



\section{Grupos afines}
\theoremstyle{plain}
\newtheorem{teoEquivalencia}{Teorema}[section]
\newtheorem{coroGrupoDeMorfismos}[teoEquivalencia]{Corolario}
\newtheorem{propoYoneda}[teoEquivalencia]{Proposici\'{o}n}
\newtheorem{teoMorfismoDeGrupos}[teoEquivalencia]{Teorema}

\theoremstyle{definition}
\newtheorem{obsGrupoDeMorfismos}[teoEquivalencia]{Observaci\'{o}n}
\newtheorem{obsProductosEnGrupos}[teoEquivalencia]{Observaci\'{o}n}
\newtheorem{obsRepresentabilidad}[teoEquivalencia]{Observaci\'{o}n}

%-------------

\subsection{Recapitulaci\'{o}n}\label{subsec:gruposafines:recap}

Una manera un poco m\'{a}s clara de hacer estas cuentas es usar diagramas:
el coproducto $\coproducto[H]:\,H\rightarrow H\tensor H$ induce
\begin{center}
	\begin{tikzcd}
		H\tensor H &
		\Homalg\big(H\tensor H,A\big)
			\arrow[d,"{\pull{\coproducto[H]}}"'] &
		\Homalg\big(H,A\big)\,\times\,
			\Homalg\big(H,A\big) \arrow[l,"\sim"'] \\
		H \arrow[u,"{\coproducto[H]}"] &
		\Homalg\big(H,A\big) &
	\end{tikzcd}
\end{center}
el isomorfismo horizontal est\'{a} dado por
$(\psi,\phi)\mapsto\producto[A]\,(\psi\tensor\phi)$ y el morfismo vertical es
precomponer con $\coproducto[H]$, mostrando que el producto es la
convoluci\'{o}n; la counidad $\counidad[H]:\,H\rightarrow k$ induce
\begin{center}
	\begin{tikzcd}
		k &
		\Homalg\big(k,A\big)\,=\,\big\{\unidad[A]\big\}
			\arrow[d,"{\pull{\counidad[H]}}"'] &
		\big\{1\big\} \arrow[l,"\sim"'] \\
		H \arrow[u,"{\counidad[H]}"] &
		\Homalg\big(H,A\big) &
	\end{tikzcd}
\end{center}
el isomorfismo horizontal es $1\mapsto\unidad[A]$ y el morismo vertical es
precomponer con $\counidad[H]$, mostrando que el elemento neutro es
$\unidad[A]\,\counidad[H]$; la ant\'{\i}poda $S_H:\,H\rightarrow H$ induce
\begin{center}
	\begin{tikzcd}
		H & \Homalg\big(H,A\big) \arrow[d,"{\pull S_H}"'] \\
		H \arrow[u,"S_H"] &
		\Homalg\big(H,A\big)
	\end{tikzcd}
\end{center}
es decir, el inverso es $\pull S_H(\psi)=\psi\,S_H$. Para que $\pull S_H$
est\'{e} bien definido, se usa la conmutatividad de $A$. La estructura de
\'{a}lgebra de $H$ define la base $\Homalg\big(H,A\big)$, la estructura de
co\'{a}lgebra (bi\'{a}lgebra) da lugar al monoide
$(\Homalg\big(H,A\big),\convol,c)$ y la ant\'{\i}poda, junto con la
conmutatividad de $A$, permiten dar una noci\'{o}n de inverso. Los axiomas de
monoide/grupo son consecuencia de los axiomas de coasociatividad, counidad y de
la definici\'{o}n de ant\'{\i}poda.

Sean $G_A=\Homalg\big(H,A\big)$,
$G_A^{\otimes i}=\Homalg\big(H^{\otimes i},A\big)$ para $i\geq 1$. Entonces, la
asociatividad del producto tensorial implica que el diagrama siguiente
conmuta con todas las flechas isomorfismos:
\begin{center}
	\begin{tikzcd}
		& G\times (G\times G) \arrow[r,
			"\id\times({\producto[A]}\circ\tensor)"] &
		G\times G^{\otimes 2}
			\arrow[dr,"{\producto[A]}\circ\tensor"] & \\
		G\times G\times G \arrow[ur,"\sim"]
			\arrow[dr,"\sim"'] & & &
		G^{\otimes 3} \\
		& (G\times G)\times G\arrow[r,
			"({\producto[A]}\circ\tensor)\times\id"'] &
		G^{\otimes 2}\times G
			\arrow[ur,"{\producto[A]}\circ\tensor"'] &
	\end{tikzcd}
\end{center}
Ahora, el diagrama de coasociatividad para $\coproducto[H]$ induce un diagrama
para $G_A$ que se interpreta como la asociatividad del producto en $G_A$:
\begin{center}
	\begin{tikzcd}
		H\tensor H\tensor H &
			H\tensor H
				\arrow[l,"\id\tensor\Delta"'] \\
		H\tensor H \arrow[u,"\Delta\tensor\id"] &
			H \arrow[u,"\Delta"']
				\arrow[l,"\Delta"]
	\end{tikzcd}
	$\rightsquigarrow$
	\begin{tikzcd}
		G_A^{\otimes 3} \arrow[r,"\pull{(\id\tensor\Delta)}"]
			\arrow[d,"\pull{(\Delta\tensor\id)}"'] &
			G_A^{\otimes 2} \arrow[d,"\pull\Delta"] \\
		G_A^{\otimes 2} \arrow[r,"\pull\Delta"'] &
			G_A
	\end{tikzcd}
\end{center}
Expl\'{\i}citamente, $f\convol (g\convol h)=(f\convol g)\convol h$. De manera
similar, el isomorfismo $k\tensor H\simeq H$ induce un isomorfismo
\begin{align*}
	G_A & \,=\,\Homalg\big(H,A\big)
		\,\simeq\,\Homalg\big(k\tensor H,A\big) \\
	& \,\simeq\,\Homalg\big(k,A\big)\times\Homalg\big(H,A\big)
		\,=\,\{\unidad[A]\}\times G_A
\end{align*}
%
y, an\'{a}logamente, $H\tensor k\simeq H$ induce
$G_A\simeq G_A\times \{\unidad[A]\}$. As\'{\i},
\begin{center}
	\begin{tikzcd}
		k\tensor H &
		H\tensor H \arrow[l,"\counidad\tensor\id"']
			\arrow[r,"\id\tensor\counidad"] &
		H\tensor k \\
		& H \arrow[u,"\coproducto"'] \arrow[ul,"\sim"]
			\arrow[ur,"\sim"']
	\end{tikzcd}
	$\rightsquigarrow$
	\begin{tikzcd}
		\{\unidad[A]\}\times G_A
			\arrow[r,"\pull{(\counidad\tensor\id)}"]
			\arrow[dr, "\sim"'] &
		G_A^{\otimes 2} \arrow[d,"\pull\coproducto"] &
		G_A\times\{\unidad[A]\}
			\arrow[l,"\pull{(\id\tensor\counidad)}"']
			\arrow[dl,"\sim"] \\
		& G_A &
	\end{tikzcd}
\end{center}
Observamos, entonces que el elemento neutro de $G_A$ est\'{a} dado por
$\pull\counidad(\unidad[A])=\unidad[A]\,\counidad$. Por \'{u}ltimo, para la
ant\'{\i}poda tenemos diagramas conmutativos
\begin{center}
	\begin{tikzcd}
		H\tensor H \arrow[r,"\producto"] &
		H &
		H\tensor H \arrow[l,"\producto"'] \\
		H\tensor H \arrow[u,"S\tensor\id"] &
		H \arrow[l,"\coproducto"] \arrow[r,"\coproducto"']
			\arrow[u,"\unidad\,\counidad"'] &
		H\tensor H \arrow[u,"\id\tensor S"']
	\end{tikzcd}
	$\rightsquigarrow$
	\begin{tikzcd}
		G_A^{\otimes 2} \arrow[d,"\pull{(S\tensor\id)}"'] &
		G_A \arrow[l,"\pull\producto"']
			\arrow[d,"\pull{(\unidad\,\counidad)}"]
			\arrow[r,"\pull\producto"] &
		G_A^{\otimes 2} \arrow[d,"\pull{(\id\tensor S)}"] \\
		G_A^{\otimes 2} \arrow[r,"\pull\coproducto"'] &
		G_A &
		G_A^{\otimes 2} \arrow[l,"\pull\coproducto"]
	\end{tikzcd}
\end{center}
Si $f\in G_A$, entonces
\begin{math}
	\pull{(\unidad\,\counidad)}\,f=f\circ(\unidad\,\counidad)=
		\unidad[A]\,\counidad
\end{math}, que es el elemento neutro de $G_A$; la aplicaci\'{o}n
$\pull\producto:\,G_A\rightarrow G_A^{\otimes 2}$ est\'{a} dada por
$f\mapsto f\circ\producto=\producto[A]\circ(f\tensor f)$ y, componiendo con el
isomorfismo $G_A^{\otimes 2}\simeq G_A\times G_A$ se obtiene la diagonal
$\diag:\,f\mapsto (f,f)$; v\'{\i}a este mismo isomorfismo,
$\pull{(S\tensor\id)}$ se corresponde con $\pull S\times\id[G_A]$,
$\pull{(\id\tensor S)}$ con $\id[G_A]\times\pull S$ y $\pull\coproducto$ con
$(f,g)\mapsto f\convol g$, el producto en $G_A$. En definitiva, el diagrama
siguiente conmuta, mostrando que $f\mapsto f\circ S$ es el inverso en $G_A$:
\begin{center}
	\begin{tikzcd}
		G_A\times G_A \arrow[d,"\pull S\times\id"'] &
		G_A \arrow[l,"\diag"'] \arrow[d,"1"]
			\arrow[r,"\diag"] &
		G_A\times G_A \arrow[d,"\id\times\pull S"] \\
		G_A\times G_A \arrow[r,"\convol"'] & G_A &
		G_A\times G_A \arrow[l,"\convol"]
	\end{tikzcd}
\end{center}

\subsection{El grupo $\Homalg\big(H,-\big)$}%
	\label{subsec:gruposafines:elgrupodemorfismos}

\subsubsection{El funtor $\Homalg\big(H,-\big)$}
Para cada \'{a}lgebra de Hopf $H$ y cada \'{a}lgebra conmutativa $A$,
obtenemos un grupo en $G_A=\Homalg\big(H,A\big)$.%
\footnote{
	Si $H$ es bi\'{a}lgebra y $A$ no necesariamente es conmutativa,
	entonces se obtiene un monoide.
}
Supongamos que tenemos, adem\'{a}s, un morfismo $\varphi:\,A\rightarrow B$ de
$k$-\'{a}lgebras. Como $\Homalg\big(H,-\big):\,\Alg[k]\rightarrow\Set$ es
funtor, $\varphi$ induce una funci\'{o}n $\push\varphi:\,G_A\rightarrow G_B$,
dada por $f\mapsto \varphi\circ f$. Similarmente, se obtiene una
$\push\varphi:\,G_A^{\otimes 2}\rightarrow G_B^{\otimes 2}$. Notamos que
\begin{center}
	\begin{tikzcd}
		G_A^{\otimes 2} \arrow[d,"\push\varphi"']
			\arrow[r,"\pull{\coproducto[A]}"] &
		G_A \arrow[d,"\push\varphi"] \\
		G_B^{\otimes 2} \arrow[r,"\pull{\coproducto[B]}"'] &
		G_B
	\end{tikzcd}
\end{center}
conmuta. Aqu\'{\i} $\pull{\coproducto[A]}$ y $\pull{\coproducto[B]}$ denotan
precomposici\'{o}n con $\coproducto$ en $G_A$ y en $G_B$, respectivamente.
Entonces ambos caminos son iguales: ambos son componer a derecha con
$\coproducto$ y componer a izquierda con $\varphi$. En t\'{e}rminos de la
convoluci\'{o}n,
\begin{align*}
	\push\varphi(f\convol g) & \,=\,
		\varphi\circ\producto[A]\circ(f\tensor g)\circ\coproducto
		\,=\,\producto[B]\circ(\varphi\tensor\varphi)\circ
			(f\tensor g)\circ\coproducto
		\,=\,\push\varphi(f)\convol\push\varphi(g)
	\text{ .}
\end{align*}
%
Es decir, $\push\varphi$ es morfismo de grupos. Dicho de otra manera, el funtor
$\Homalg\big(H,-\big)$ se factoriza por la categor\'{\i}a de grupos.

\begin{coroGrupoDeMorfismos}\label{coro:grupodemorfismos}
	La aplicaci\'{o}n que a un \'{a}lgebra conmutativa $A$ le asigna el
	grupo dado por el conjunto $\Homalg\big(H,A\big)$ junto con la
	estructura definida en el Teorema~\ref{thm:grupodemorfismos} y que a un
	morfismo $\varphi:\,A\rightarrow B$ le asigna $\push\varphi$ determina
	un funtor $G:\,\CommAlg[k]\rightarrow\Grp$. Este funtor verifica
	\begin{align*}
		U\circ G & \,=\,\Homalg\big(H,-\big)
		\text{ ,}
	\end{align*}
	%
	donde $U:\,\Grp\rightarrow\Set$ denota el funtor olvido.
\end{coroGrupoDeMorfismos}

\begin{obsGrupoDeMorfismos}\label{obs:grupodemorfismos}
	Los diagramas conmutativos que expresan que $\Homalg\big(H,-\big)$ se
	factoriza por $\Grp$, es decir, que las $\push\varphi$ son morfismos de
	grupos, son los mismos diagramas que expresan la naturalidad de
	\begin{align*}
		\pull\coproducto & \,:\,\Homalg\big(H,-\big)
			\,\xrightarrow{\cdot}\,
			\Homalg\big(H\tensor H,-\big)
		\text{ .}
	\end{align*}
	%
\end{obsGrupoDeMorfismos}

\subsubsection{Grupos en $\CommAlg[k]\rightarrow\Set$}
En la categor\'{\i}a $\CommAlg[k]\rightarrow\Set$ existen productos y objetos
terminales. Denotamos por $X\times Y$ el producto de los conjuntos $X$ e $Y$ y
por $1$ el conjunto con un \'{u}nico elemento, el objeto terminal en $\Set$.
Entonces, dados $F,F':\,\CommAlg[k]\rightarrow\Set$, definimos un nuevo funtor
\begin{align*}
	(F\times F')(\varphi) & \,=\,F(\varphi)\times F'(\varphi):\,
		F(A)\times F'(A)\rightarrow F(B)\times F'(B)
\end{align*}
%
al que llamamos \emph{producto de $F$ con $F'$}; definimos, tambi\'{e}n el
funtor $\mathsf{1}$ dado por $A\mapsto 1$ en objetos y por
$\varphi\mapsto\id[1]$ en morfismos. Notemos que existe un isomorfismo natural
\begin{align*}
	\mathsf 1 & \,\simeq\,\Homalg\big(k,-\big)	
\end{align*}
%
Para cada $F:\,\CommAlg[k]\rightarrow\Set$ existe una \'{u}nica
transformaci\'{o}n natural $F\xrightarrow{\cdot}\mathsf{1}$; en cada objeto $A$
est\'{a} dada por la \'{u}nica funci\'{o}n
\begin{align*}
	t & \,:\, F(A)\,\xrightarrow{\cdot}\,\mathsf 1(A)
	\text{ .}
\end{align*}
%
Esto nos permite definir \emph{grupos en $\CommAlg[k]\rightarrow\Set$}.

\begin{obsProductosEnGrupos}\label{obs:productosengrupos}
	Tambi\'{e}n existen productos y objetos terminales en la categor\'{\i}a
	de funtores en $\Grp$, pero no los necesitaremos para definir grupos
	afines.
\end{obsProductosEnGrupos}

\begin{coroGrupoDeMorfismos}\label{coro:grupodemorfismos:grupoenlacategoria}
	Sea $G:\,\CommAlg[k]\rightarrow\Grp$ el funtor del Corolario~%
	\ref{coro:grupodemorfismos}. Los morfismos
	$\coproducto:\,H\rightarrow H\tensor H$, $\counidad:\,H\rightarrow k$ y
	$S:\,H\rightarrow H$ determinan transformaciones naturales
	\begin{align*}
		\pull\coproducto\,:\,
			UG\times UG\,\xrightarrow{\cdot}\,UG
		& \quad\text{,}\quad
		\pull\counidad\,:\,\mathsf 1\,\xrightarrow{\cdot}\,UG
		\quad\text{y}\quad
		\pull S\,:\,UG\xrightarrow{\cdot}\,UG
	\end{align*}
	%
	que cumplen
	\begin{equation}
		\label{eq:grupodemorfismos}
		\begin{aligned}
			\pull\Delta\circ(\id\times\pull\Delta) & \,=\,
				\pull\Delta\circ(\pull\Delta\times\id) \\
			\pull\Delta\circ((\pull\varepsilon\circ t)\times\id)
				\circ\diag & \,=\,\id
				\,=\, \pull\Delta\circ
					(\id\times (\pull\varepsilon\circ t))
					\circ\diag \\
			\pull\Delta\circ(\id\times\pull S)\circ\diag & \,=\,
				\pull\varepsilon\circ t
				\,=\, \pull\Delta\circ(\pull S\times\id)
					\circ\diag
		\end{aligned}
	\end{equation}
	%
	donde $\id=\id[UG]$ y $\diag:\,UG\xrightarrow\cdot UG\times UG$
	es la transformaci\'{o}n diagonal.
\end{coroGrupoDeMorfismos}

\begin{obsProductosEnGrupos}\label{obs:productosengrupos:objetoterminal}
	En realidad, sabemos un poco m\'{a}s. La t.n. $\pull\counidad$ debe
	provenir de la \'{u}nica flecha $\mathsf 1\xrightarrow\cdot G$,
	donde $\mathsf 1$ es el objeto nulo (objeto inicial y final) de la
	categor\'{\i}a.
\end{obsProductosEnGrupos}

\subsection{Equivalencia con grupos afines}%
	\label{subsec:gruposafines:equivalencia}

Sean $G,G':\,\CommAlg[k]\rightarrow\Grp$ dos funtores. Denotamos por
$U:\,\Grp\rightarrow\Set$ el funtor olvido. En este contexto nos hacemos dos
preguntas:
\begin{itemize}
	\item dada $\tilde\tau:\,UG\xrightarrow\cdot UG'$, ?`existe
		$\tau:\,G\xrightarrow\cdot G'$ tal que $U\tau=\tilde\tau$?
	\item dadas $\tau_1,\tau_2:\,G\xrightarrow\cdot G'$ tales que
		$U\tau_1=U\tau_2$, ?`vale que $\tau_1=\tau_2$?
\end{itemize}
%
En cuanto a la segunda pregunta, como $U$ es fiel, la conmutatividad del
diagrama de la izquierda implica la conmutatividad del de la derecha:
\begin{center}
	\begin{tikzcd}
		U(G(A)) \arrow[r,"U(\tau_{1A})"] \arrow[d,equal] &
		U(G'(A)) \arrow[d,equal] \\
		U(G(A)) \arrow[r,"U(\tau_{2A})"'] &
		U(G'(A))
	\end{tikzcd}
	\begin{tikzcd}
		G(A) \arrow[r,"\tau_{1A}"] \arrow[d,equal] &
		G'(A) \arrow[d,equal] \\
		G(A) \arrow[r,"\tau_{2A}"'] &
		G'(A)
	\end{tikzcd}
\end{center}
En cuanto a la primera pregunta, que $\tilde\tau$ sea igual a $U\tau$ es
equivalente, por fidelidad de $U$ a que, para cada objeto $A$,
$\tilde\tau_A:\,UG(A)\rightarrow UG'(A)$ sea morfismo de grupos.

Para cada \'{a}lgebra $A$, el objeto $G(A)$ es un grupo. En particular, existe
una \emph{funci\'{o}n} $m_A^G:\,U(G(A))\times U(G(A))\rightarrow U(G(A))$ para
la cual se verifican los axiomas de grupos. Que $\tilde\tau_A$ sea morfismo de
grupos de $G(A)$ en $G'(A)$ significa que existe un cuadrado conmutativo
\begin{center}
	\begin{tikzcd}
		UG(A)\times UG(A) \arrow[r,"m_A^G"]
			\arrow[d,"\tilde\tau_A\times\tilde\tau_A"'] &
		UG(A) \arrow[d,"\tilde\tau_A"] \\
		UG'(A) \times UG'(A) \arrow[r,"m_A^{G'}"'] &
		UG'(A)
	\end{tikzcd}
\end{center}
Pero esto querr\'{\i}a decir que, en cierto sentido, la multiplicaci\'{o}n
deber\'{\i}a ser natural en los funtores $G$ y $G'$.

Supongamos, entonces que $G$ y $G'$ son representables, en tanto existen
\'{a}lgebras de Hopf $H$ y $H'$ tales que%
\footnote{
	En cuanto a por qu\'{e} deben ser de Hopf, ver la Observaci\'{o}n~%
	\ref{obs:representabilidad}.
	}
\begin{align*}
	U\circ G \,=\,\Homalg\big(H,-\big) & \quad\text{y}\quad
	U\circ G' \,=\,\Homalg\big(H',-\big)
	\text{ .}
\end{align*}
%

\begin{propoYoneda}\label{propo:yoneda}
	La aplicaci\'{o}n $\phi\mapsto(\pull\phi:\,f\mapsto f\circ\phi)$
	determina una biyecci\'{o}n
	\begin{equation}
		\label{eq:yoneda}
		\Homalg\big(H',H\big)\,=\,U\circ G'(H) \,\simeq\,
			\Nat(U\circ G,U\circ G')
	\end{equation}
	%
\end{propoYoneda}

Los morfismos de la Proposici\'{o}n~\ref{propo:yoneda} son, \emph{a priori},
morfismos de \'{a}lgebras, no necesariamente de \'{a}lgebras de Hopf, ni de
bi\'{a}lgebras. Volviendo a las preguntas anteriores, dada
$\phi:\,H'\rightarrow H$, ?`existe una t.n. $\tau:\,G\xrightarrow\cdot G'$ tal
que $U\tau=\pull\phi$? Como ya hemos mencionado, esto significa que
$\pull\phi:\,UG(A)\rightarrow UG'(A)$ es morfismo de grupos, para cada $A$. La
estructura de grupo en estos conjuntos est\'{a} dada por el Teorema~%
\ref{thm:grupodemorfismos}. En particular, $\pull\phi$ es morfismo de grupos,
si y s\'{o}lo si el diagrama siguiente conmuta:
\begin{center}
	\begin{tikzcd}
		UG(A)\times UG(A) \arrow[r,"\pull\phi\times\pull\phi"]
			\arrow[d,"\pull{\coproducto[H]}"'] &
		UG'(A)\times UG'(A) \arrow[d,"\pull{\coproducto[H']}"] \\
		UG(A) \arrow[r,"\pull\phi"'] & UG'(A)
	\end{tikzcd}
\end{center}
o, equivalentemente, v\'{\i}a el isomorfismo
\begin{equation}
	\label{eq:isomorfismoproducto}
	\Homalg\big(H,A\big)\times\Homalg\big(H,A\big) \,\simeq\,
		\Homalg\big(H\tensor H,A\big)
\end{equation}
%
y el isomorfismo an\'{a}logo para $H'$, si y s\'{o}lo si el diagrama
\begin{center}
	\begin{tikzcd}
		\Homalg\big(H\tensor H,A\big)
			\arrow[r,"\pull{(\phi\tensor\phi)}"]
			\arrow[d,"\pull{\coproducto[H]}"'] &
		\Homalg\big(H'\tensor H',A\big)
			\arrow[d,"\pull{\coproducto[H']}"] \\
		\Homalg\big(H,A\big) \arrow[r,"\pull\phi"'] &
		\Homalg\big(H',A\big)
	\end{tikzcd}
\end{center}
conmuta. Es decir, para todo par $f,g:\,H\rightarrow A$, debe ser
\begin{align*}
	\producto[A]\,(f\tensor g)\,(\phi\tensor\phi)\,\coproducto[H'] & \,=\,
		\producto[A]\,(f\tensor g)\,\coproducto[H]\,\phi
	\text{ .}
\end{align*}
%
Necesitamos que estos diagramas conmuten \emph{para toda $A$}. Pero la
conmutatividad para toda \'{a}lgebra $A$ equivale a la conmutatividad de un
\'{u}nico diagrama: tomando $A=H$ y evaluando en el par $(\id[H],\id[H])$,
deducimos que
\begin{center}
	\begin{tikzcd}
		H\tensor H &
		H'\tensor H' \arrow[l,"\phi\tensor\phi"'] \\
		H \arrow[u,"{\coproducto[H]}"] &
		H' \arrow[u,"{\coproducto[H']}"'] \arrow[l,"\phi"]
	\end{tikzcd}
\end{center}
debe conmutar. Pero esto quiere decir, exactamente, que
$\phi:\,H'\rightarrow H$ es morfismos de co\'{a}lgebras, tambi\'{e}n.%
\footnote{
	En realidad, resta ver que respeta la counidad, pero esto se deduce
	haciendo un razonamiento an\'{a}logo con los diagramas que involucran
	$\pull{\counidad[H]}$ y $\pull{\counidad[H']}$ y el objeto terminal.
	Si los diagramas
	\begin{center}
		\begin{tikzcd}[ampersand replacement=\&]
			\mathsf 1(A) \arrow[r,equal]
				\arrow[d,"\pull{\counidad[H]}"'] \&
			\mathsf 1(A) \arrow[d,"\pull{\counidad[H']}"] \\
			UG(A) \arrow[r,"\pull\phi"'] \&
			UG'(A)
		\end{tikzcd}
	\end{center}
	conmutan para toda \'{a}lgebra $A$, en particular, conmutan para $A=H$
	y, evaluando en $\unidad[H]$ --el \'{u}nico elemento de
	$\mathsf 1(H)$--, se deduce que, para $x\in H'$,
	\begin{align*}
		\unidad[H]\,(\counidad[H]\,\phi) & \,=\,
			\pull\phi\,\pull{\counidad[H]}(\unidad[H]) \,=\,
			\pull{\counidad[H']}(\unidad[H]) \,=\,
			\unidad[H]\,\counidad[H']
		\text{ .}
	\end{align*}
	%
	Pero entonces, componiendo con $\counidad[H]$ a izquierda, podemos
	``cancelar'' y obtener $\counidad[H]\,\phi=\counidad[H']$.
}
En particular, $\phi$ es morfismo de bi\'{a}lgebras y, por lo tanto, de
\'{a}lgebras de Hopf. Rec\'{\i}procamente, si $\phi$ es de \'{a}lgebras de
Hopf, entonces el \'{u}ltimo diagrama conmuta y, aplicando el funtor
$\Homalg\big(-,A\big)$ se obtiene el ante\'{u}ltimo diagrama, lo que muestra
que, en ese caso, $\pull\phi:\,UG(A)\rightarrow UG'(A)$ es morfismo de grupos
para toda $A$.

\begin{teoMorfismoDeGrupos}\label{thm:morfismodegrupos}
	Dadas \'{a}lgebras de Hopf $H,H'$, dado un morfismo de \'{a}lgebras
	$\phi:\,H'\rightarrow H$, la transformaci\'{o}n natural
	\begin{math}
		\pull\phi:\,\Homalg\big(H,-\big)\xrightarrow\cdot
			\Homalg\big(H',-\big)
	\end{math} es morfismo de grupos, si y s\'{o}lo si $\phi$ es morfismo
	de \'{a}lgebras de Hopf.
\end{teoMorfismoDeGrupos}

\begin{teoEquivalencia}\label{thm:equivalencia}
	La aplicaci\'{o}n
	\begin{align*}
		H\,\mapsto\,\Homalg\big(H,-\big) &\quad\text{,}\quad
			\phi\,\mapsto\,\pull\phi
	\end{align*}
	%
	define un funtor contravariante fiel y pleno de la categor\'{\i}a de
	\'{a}lgebras de Hopf en la categor\'{\i}a de grupos afines,
	$(G,m,u,\sigma)$ donde
	\begin{itemize}
		\item $G:\,\CommAlg[k]\rightarrow\Grp$ es funtor,
		\item $UG$ es representable: existe $H$ tal que
			$UG\simeq\Homalg\big(H,-\big)$,
		\item $m,u,\sigma$ son transformaciones naturales que hacen de
			$UG$ un grupo en $\Set^{\CommAlg[k]}$.
	\end{itemize}
	%
\end{teoEquivalencia}

\begin{obsRepresentabilidad}\label{obs:representabilidad}
	Sea $(G,m,u,\sigma)$ un grupo af\'{\i}n y sea $R$ un \'{a}lgebra
	conmutativa que lo representa. Entonces $R$ admite una estructura de
	\'{a}lgebra ed Hopf. Por definici\'{o}n, $m$ define, componiendo con el
	isomorfismo \eqref{eq:isomorfismoproducto}, una t.n.
	\begin{align*}
		m & \,:\,\Homalg\big(R\tensor R,-\big) \,\xrightarrow\cdot\,
			\Homalg\big(R,-\big)
		\text{ .}
	\end{align*}
	%
	Por la Proposici\'{o}n~\ref{propo:yoneda}, existe un morfismo de
	\'{a}lgebras $\coproducto:\,R\rightarrow R\tensor R$ tal que
	$\pull\coproducto=m$. Expl\'{\i}citamente, siguiendo la
	demostraci\'{o}n de la Proposici\'{o}n~\ref{propo:yoneda}, definimos
	\begin{align*}
		\coproducto & \,:=\, m_{R\tensor R}(\id[R\tensor R])\,\in\,
			\Homalg\big(R,R\tensor R\big)
		\text{ .}
	\end{align*}
	%
	Para cada $k$-\'{a}lgebra $A$, existe una funci\'{o}n
	$m_A:\,\Homalg\big(R\tensor R,A\big)\rightarrow\Homalg\big(R,A\big)$.
	Dado $f:\,R\tensor R\rightarrow A$, ?`qu\'{e} morfismo es $m_A(f)$?
	?`Qu\'{e} funci\'{o}n es $m_A$? Por naturalidad,
	\begin{center}
		\begin{tikzcd}
			\Homalg\big(R\tensor R,R\tensor R\big)
				\arrow[r,"m_{R\tensor R}"]
				\arrow[d,"\push f"'] &
			\Homalg\big(R,R\tensor R\big)
				\arrow[d,"\push f"] \\
			\Homalg\big(R\tensor R,A\big) \arrow[r,"m_A"'] &
			\Homalg\big(R,A\big)
		\end{tikzcd}
	\end{center}
	conmuta y
	\begin{align*}
		f\circ\coproducto & \,=\,
			\push f\circ m_{R\tensor R}(\id[R\tensor R])
			\,=\,m_A\circ\push f(\id[R\tensor R])
			\,=\,m_A(f)
		\text{ .}
	\end{align*}
	%
	Entonces $\pull\coproducto=m$. Hay que ver que $\coproducto$ es
	coproducto, pero esto se deduce de aplicar la correspondencia de la
	Proposici\'{o}n~\ref{propo:yoneda} a las transformaciones naturales que
	aparecen en los diagramas que hacen de $m$ un producto.
\end{obsRepresentabilidad}


\end{document}
