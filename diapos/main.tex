
\documentclass{beamer}
\mode<presentation>
{
  \usetheme{Singapore}

  \setbeamercovered{transparent}
}

\usepackage[utf8]{inputenc}
\usepackage{tikz-cd}
\usetikzlibrary{matrix}
\usetikzlibrary{decorations.pathmorphing}
\usepackage{nombres}
\usepackage{abreviaciones}

\title{\'{A}lgebras de Hopf y grupos afines}
\subtitle{}
\author{}
\institute{
	% Departamento de Matem\'{a}tica\\
	% Facultad de Ciencias Exactas y Naturales\\
	% Universidad de Buenos Aires
}
\date{}
\subject{Talks}

% \pgfdeclareimage[height=0.5cm]{university-logo}{university-logo-filename}
% \logo{\pgfuseimage{university-logo}}

\AtBeginSubsection[]
{
  \begin{frame}<beamer>{Contenidos}
    \tableofcontents[currentsection,currentsubsection]
  \end{frame}
}

% \beamerdefaultoverlayspecification{<+->}

\setbeamertemplate{itemize subitem}{$\circ$}

\newcounter{saveenumi}
\newcommand{\seti}{\setcounter{saveenumi}{\value{enumi}}}
\newcommand{\conti}{\setcounter{enumi}{\value{saveenumi}}}
\resetcounteronoverlays{saveenumi}

\begin{document}

\begin{frame}
  \titlepage
\end{frame}

\begin{frame}{Contenidos}
  \tableofcontents
  %% You might wish to add the option [pausesections]
  % Empezamos recordando la definici\'{o}n de $k$-\'{a}lgebra y algunas construcciones,
  % como el \'{a}lgebra libre, y vemos c\'{o}mo definir un grupo a partir del conjunto de
  % morfismos de $k$-\'{a}lgebras en algunos casos particulares/.
  %
  % A continuaci\'{o}n introducimos las definiciones de co\'{a}lgebra y bi\'{a}lgebra y
  % damos algunos ejemeplos.
  %
  % Seguimos, luego, con la noci\'{o}n de convoluci\'{o}n, ant\'{\i}poda y \'{a}lgebra de
  % Hopf, terminando esta parte definiendo un grupo a partir de un \'{a}lgebra de Hopf y
  % un \'{a}lgebra conmutativa.
  %
  % Por \'{u}ltimo, veremos c\'{o}mo estas estructuras describen una ley de grupo de
  % manera universal, es decir, independientemente del \'{a}lgebra conmutativa. Esto
  % permitir\'{a} asociarle a un \'{a}lgebra de Hopf (conmutativa) un esquma af\'{\i}n en
  % grupos.
  %
\end{frame}

\section{$k$-\'{a}lgebras}
\theoremstyle{plain}
\newtheorem{defAlgebra}{Definici\'{o}n}[section]
\newtheorem{ejemploLibre}[defAlgebra]{Ejemplos}
\newtheorem{propoLibre}[defAlgebra]{Proposici\'{o}n}
\newtheorem{ejemploCocienteDeLibre}[defAlgebra]{Ejemplo}
\newtheorem{propoRectaAfin}[defAlgebra]{Proposici\'{o}n}
\newtheorem{propoMultiplicativo}[defAlgebra]{Proposici\'{o}n}
\newtheorem{coroMultiplicativo}[defAlgebra]{Corolario}
\newtheorem{propoProductoDeMatrices}[defAlgebra]{Proposici\'{o}n}
\newtheorem{obsProductoDeMatrices}[defAlgebra]{Observaci\'{o}n}
\newtheorem{propoLinealGeneral}[defAlgebra]{Proposici\'{o}n}
\newtheorem{propoProductoTensorial}[defAlgebra]{Proposici\'{o}n}
\newtheorem{propoProductoDeCociente}[defAlgebra]{Proposici\'{o}n}
\newtheorem{obsRectaAfinAbeliana}[defAlgebra]{Observaci\'{o}n}

%------------


\subsection{Definiciones}

\begin{frame}{Definici\'{o}n y algunas construcciones}
	$k$: anillo conmutativo (con unidad).

	\begin{defAlgebra}\label{def:algebra}
		Una $k$-\'{a}lgebra es un anillo $A$ junto con un morfismo de
		anillos $\eta_A:\,k\rightarrow A$.
		\begin{align*}
			\Homalg(A,B) & \,=\,
				\Big\{f:\,A\rightarrow B
				\text{ de anillos, } f\circ\eta_A=\eta_B
				\Big\}
		\end{align*}
		%
	\end{defAlgebra}
	%
	$A$ es $k$-m\'{o}dulo con $(\lambda,a)\mapsto\eta_A(\lambda)\,a$ y
	$\mu_A:\,A\times A\rightarrow A$ es $k$-bilineal.
\end{frame}

\begin{frame}{El \'{a}lgebra libre}
	Dado un conjunto $X$, una \emph{palabra} en $X$ es $x_1\,\cdots\,x_n$ o
	$\varnothing$. Definimos $k\{X\}$, el $k$-m\'{o}dulo libre en las
	palabras en $X$. Es un \'{a}lgebra con
	\begin{align*}
		(x_{i_1}\,\cdots\,x_{i_n})\,(x_{i_{n+1}}\,\cdots\,x_{i_m})
			& \,=\,x_{i_1}\,\cdots\,x_{i_n}\,x_{i_{n+1}}\,\cdots\,
				x_{i_m}
		\text{ .}
	\end{align*}
	%
	\begin{ejemploLibre}\label{ejemplo:libre}
		\begin{itemize}
			\item $k\{x\}=k[x]$, polinomios en una variable;
			\item $k\{x,y\}\not=k[x,y]$, pues $xy\not=yx$.
		\end{itemize}
	\end{ejemploLibre}
	\begin{propoLibre}\label{propo:libre}
		Dados un conjunto $X$, una $k$-\'{a}lgebra $A$ y una
		\emph{funci\'{o}n} $f:\,X\rightarrow A$, existe un \'{u}nico
		morfismo $\tilde f:\,k\{X\}\rightarrow A$ tal que
		$\tilde f(x)=f(x)$, si $x\in X$.
	\end{propoLibre}
\end{frame}

\begin{frame}{El \'{a}lgebra libre (cont.)}
	Existe una biyecci\'{o}n natural
	\begin{align*}
		\Homalg\big(k\{X\},A\big) & \,\simeq\,
			\Hom[\Set]\big(X,\olvido A\big)
		\text{ .}
	\end{align*}
	%
	En particular,
	\begin{math}
		\Homalg\big(k\{x,y\},A\big)=A^2
	\end{math}, v\'{\i}a $f\mapsto (f(x),f(y))$.

	Un poco m\'{a}s en general,
	\begin{align*}
		\Homalg\big(k\{X\}/I,A\big) & \,\simeq\,
			\Big\{f\in\Hom[\Set]\big(X,\olvido A\big)\,:\,
				\tilde f(I)=0\Big\}
		\text{ .}
	\end{align*}
	%
	\begin{ejemploCocienteDeLibre}\label{ejemplo:cocientedelibre}
		Para el \'{a}lgebra $k[x,y]\simeq k\{x,y\}/\generado{xy-yx}$,
		\begin{align*}
			\Homalg\big(k[x,y],A\big) & \,\simeq\,
				\big\{(a,b)\in A^2\,:\,ab=ba\big\}
			\text{ .}
		\end{align*}
		%
	\end{ejemploCocienteDeLibre}
\end{frame}

\subsection{Ejemplos}

\begin{frame}{La recta y el plano afines}
	Asumimos $A$ conmutativa. Existen biyecciones
	\begin{align*}
		\Homalg\big(k[x],A\big) & \,\simeq\,A \text{ ,} \\
		\Homalg\big(k[x',x''],A\big) & \,\simeq\,A^2 \quad\text{y} \\
		\Homalg\big(k,A\big) & \,\simeq\,\{0\} \quad
			\text{(\phantom)}=\{\eta_A\}\text{\phantom()}
		\text{ .}
	\end{align*}
	%
	Queremos expresar las leyes de grupo abeliano de $A$ de manera
	universal:
	\begin{align*}
		+ \,:\,A\,\times\,A\rightarrow A & \quad\text{,}\quad
		0 \,:\,\{0\}\,\rightarrow\,A \quad\text{y}\quad
		- \,:\,A\,\rightarrow\,A
		\text{ .}
	\end{align*}
	%
\end{frame}

\begin{frame}{La recta y el plano afines (cont.)}
	\begin{propoRectaAfin}\label{propo:rectaafin}
		V\'{\i}a las biyecciones, los morfismos
		$\Delta:\,k[x]\rightarrow k[x',x'']$,
		$\varepsilon:\,k[x]\rightarrow k$ y $S:\,k[x]\rightarrow k[x]$,
		determinados por
		\begin{align*}
			\Delta(x) \,=\,x'+x'' & \quad\text{,}\quad
				\varepsilon(x)\,=\,0 \quad\text{y}\quad
				S(x)\,=\,-x
			\text{ ,}
		\end{align*}
		%
		se corresponden con $+$, $0$ y $-$, respectivamente.
	\end{propoRectaAfin}
	\begin{proof}
		$\Delta$ induce
		\begin{math}
			\pull\Delta:\,\Homalg\big(k[x',x''],A\big)\rightarrow
				\Homalg\big(k[x],A\big)
		\end{math} dada por $\pull\Delta(f)=f\circ\Delta$. Vale
		\begin{align*}
			\big(\pull\Delta f\big)(x) & \,=\,f(x'+x'')\,=\,
				f(x')+f(x'')
			\text{ .}
		\end{align*}
		%
	\end{proof}
\end{frame}

\begin{frame}{La recta y el plano afines (cont.)}
	Dados $f,g:\,k[x]\rightarrow A$, definimos la ``suma''
	$\pull\Delta(f,g):\,k[x]\rightarrow A$: si $f(x)=a$ y $g(x)=b$,
	\begin{align*}
		\pull\Delta(f,g)(x) & \,=\,a+b
		\text{ .}
	\end{align*}
	%
	La suma satisface:
	\begin{itemize}
		\item
			\begin{math}
				\pull\Delta(f,\pull\Delta(g,h))=
					\pull\Delta(\pull\Delta(f,g),h)
			\end{math};
		\item
			\begin{math}
				\pull\Delta(f,\pull\varepsilon(\eta_A))=
					\pull\Delta(\pull\varepsilon(\eta_A),f)
					=f
			\end{math};
		\item
			\begin{math}
				\pull\Delta(f,\pull S(f))=
					\pull\Delta(\pull S(f),f)=
					\pull\varepsilon(\eta_A)
			\end{math};
		\item
			\begin{math}
				\pull\Delta(f,g)=\pull\Delta(g,f)
			\end{math}.
	\end{itemize}
	%
	\begin{align*}
		\pull\Delta(f,\pull\Delta(g,h))(x) & \,=\,
			f(x)+g(x)+h(x)\,=\,
			\pull\Delta(\pull\Delta(f,g),h)(x)
		\text{ .}
	\end{align*}
	%
\end{frame}

\begin{frame}{La recta y el plano afines (cont.)}
	Todo morfismo de $k$-\'{a}lgebras $\varphi:\,A\rightarrow B$ induce
	$\push\varphi(f)=\varphi\circ f$. Esta \emph{funci\'{o}n} es morfismo
	de grupos:
	\begin{center}
		\begin{tikzcd}[column sep=small,
			ampersand replacement=\&]
			\Homalg\big(k[x',x''],A\big)
				\arrow[r,"\pull\Delta"]
				\arrow[d,"\push\varphi"'] \&
			\Homalg\big(k[x],A\big)
				\arrow[d,"\push\varphi"] \\
			\Homalg\big(k[x',x''],B\big)
				\arrow[r,"\pull\Delta"'] \&
			\Homalg\big(k[x],B\big)
		\end{tikzcd}
	\end{center}
	\begin{math}
		\pull\Delta:\,\Homalg\big(k[x',x''],-\big)\xrightarrow{\cdot}
			\Homalg\big(k[x],-\big)
	\end{math} es una transformaci\'{o}n natural. $\pull\varepsilon$ y
	$\pull S$ tambi\'{e}n.
\end{frame}

\begin{frame}{La recta y el plano afines (cont.)}
	% $\Homalg\big(k[x],-\big):\,\AlgCom[k]\rightarrow\Set$ se factoriza
	% por $\Grp$:
	Existe $G:\,\CommAlg[k]\rightarrow\Grp$ tal que
	\begin{align*}
		U\circ G & \,=\,\Homalg\big(k[x],-\big)
		\text{ .}
	\end{align*}
	%
	La funci\'{o}n $(f\mapsto f(x)):\,\Homalg\big(k[x],A\big)\rightarrow A$
	es un isomorfismo de grupos $\tau_A:\,G(A)\rightarrow (A,+)$ una t.n.:
	\begin{align*}
		\tau_B\circ\push\varphi(f) & \,=\,(\varphi\circ f)(x)\,=\,
			\varphi(f(x))\,=\,\push\varphi\circ\tau_A(f)
	\end{align*}
	%
	Si $(-,+):\,\CommAlg[k]\rightarrow\Grp$ el grupo aditivo subyacente,
	\begin{align*}
		\tau & \,:\,G\,\xrightarrow{\cdot}\,(-,+)
	\end{align*}
	%
	es un isomorfismo natural.
\end{frame}

\begin{frame}{El grupo multiplicativo}
	$A^\times:\,\CommAlg[k]\rightarrow\Grp$ el grupo multiplicativo
	subyacente.
	\begin{align*}
		\times \,:\,A^\times\,\times\,A^\times\rightarrow A^\times
			& \quad\text{,}\quad
		1 \,:\,\{1\}\,\rightarrow\,A^\times \quad\text{y}\quad
		\null^{-1}\,:\,A^\times\,\rightarrow\,A^\times
		\text{ .}
	\end{align*}
	%
	Existen biyecciones ($\tau_A:\,f\mapsto f(\bar x)$)
	\begin{align*}
		\Homalg\big(k[x,x^{-1}],A\big) & \,\simeq\,A^\times \\
		\Homalg\big(k[x',x'',x'\null^{-1},x''\null^{-1}],A\big)
			& \,\simeq\,A^\times\times A^\times
		\text{ ,}
	\end{align*}
	%
	donde
	\begin{align*}
		k[x,x^{-1}] & \,:=\,k[x,y]/\generado{xy-1} \text{ ,} \\
		k[x',x'',x'\null^{-1},x''\null^{-1}] & \,:=\,
			k[x',y',x'',y'']/\generado{x'y'-1,x''y''-1}
		\text{ .}
	\end{align*}
	%
\end{frame}

\begin{frame}{El grupo multiplicativo (cont.)}
	\begin{propoMultiplicativo}\label{propo:multiplicativo}
		Los morfismos
		\begin{math}
			\Delta:\,k[x,x^{-1}]\rightarrow
				k[x',x'',x'\null^{-1},x''\null^{-1}]
		\end{math},
		\begin{math}
			\varepsilon:\,k[x,x^{-1}]\rightarrow k
		\end{math} y
		\begin{math}
			S:\,k[x,x^{-1}]\rightarrow k[x,x^{-1}]
		\end{math} determinados por
		\begin{align*}
			\Delta(x) \,=\,x'x'' & \quad\text{,}\quad
				\varepsilon(x)\,=\,1 \quad\text{y}\quad
				S(x)\,=\,x^{-1}
		\end{align*}
		%
		se corresponden con $\times$, $1$ y $\null^{-1}$.
	\end{propoMultiplicativo}
	\begin{coroMultiplicativo}\label{coro:multiplicativo}
		$\pull\Delta,\pull\varepsilon,\pull S$ son t.n. de funtores de
		tipo $\CommAlg[k]\rightarrow\Set$ y existe
		$G:\,\CommAlg[k]\rightarrow\Grp$ tal que
		\begin{align*}
			U\circ G & \,=\,\Homalg\big(k[x,x^{-1}],-\big)
			\text{ .}
		\end{align*}
		%
		Las funciones $\tau_A$ inducen un isomorfismo natural
		$\tau:\,G\xrightarrow{\cdot}(-,\times)$.
	\end{coroMultiplicativo}
\end{frame}

\begin{frame}{Producto de matrices}
	Sea $\MM(2)=k[a,b,c,d]$.
	\begin{math}
		f\mapsto f\big(
			\left[\begin{smallmatrix}
				a & b \\ c & d
			\end{smallmatrix}\right]\big)=
			\left[\begin{smallmatrix}
				f(a) & f(b) \\ f(c) & f(d)
			\end{smallmatrix}\right]
	\end{math} induce
	\begin{align*}
		\Homalg\big(\MM(2),A\big) & \,\simeq\,A^4\,=\,
			\MM[2\times 2](A)
		\text{ .}
	\end{align*}
	%
	Duplicamos las variables:
	$\MM(2)^{\otimes 2}=k[a',b',c',d',a'',b'',c'',d'']$ y buscamos
	$\Delta:\,\MM(2)\rightarrow\MM(2)^{\otimes 2}$ tal que
	\begin{center}
		\begin{tikzcd}[column sep=small,ampersand replacement=\&]
			\Homalg\big(\MM(2)^{\otimes 2},A\big)
				\arrow[r,"\sim"]
				\arrow[d,"\pull\Delta"'] \&
			\MM[2\times 2](A)^2 \arrow[d,"\cdot"] \\
			\Homalg\big(\MM(2),A\big)
				\arrow[r,"\sim"] \&
			\MM[2\times 2](A)
		\end{tikzcd}
	\end{center}
	conmute.
\end{frame}

\begin{frame}{Producto de matrices (cont.)}
	Debe cumplirse
	\begin{align*}
		f\circ\Delta\Big(
			\begin{bmatrix} a & b \\ c & d \end{bmatrix}
				\Big) & \,=\,
			f\Big(\begin{bmatrix} a' & b' \\ c' & d' \end{bmatrix}
				\Big)\,f\Big(
				\begin{bmatrix}
					a'' & b'' \\ c'' & d''
				\end{bmatrix}
				\Big) \\
		& \,=\, f\Big(\begin{bmatrix} a' & b' \\ c' & d' \end{bmatrix}
			\,\begin{bmatrix} a'' & b'' \\ c'' & d'' \end{bmatrix}
			\Big)
		\text{ .}
	\end{align*}
	%
	\begin{propoProductoDeMatrices}\label{propo:productodematrices}
		Si $\Delta:\,\MM(2)\rightarrow\MM(2)^{\otimes 2}$ es el
		morfismo determinado por
		% de \'{a}lgebras
		\begin{align*}
			\Delta\,\begin{bmatrix} a & b \\ c & d \end{bmatrix}
				& \,=\,
				\begin{bmatrix}
					a' & b' \\ c' & d'
				\end{bmatrix}\,
				\begin{bmatrix}
					a'' & b'' \\ c'' & d''
				\end{bmatrix}
			\text{ ,}
		\end{align*}
		%
		el diagrama conmuta.
		\begin{math}
			\Delta(ad-bc)=(a'd'-b'c')\,(a''d''-b''c'')
		\end{math}.
	\end{propoProductoDeMatrices}
\end{frame}

\begin{frame}{$\GL(2)$ y $\SL(2)$}
	\begin{align*}
		\GL(2) & \,=\,\MM(2)[t]/\generado{(ad-bc)\,t-1} \\
		\SL(2) & \,=\,\GL(2)/\generado{t-1} \,=\,
			\MM(2)/\generado{ad-bc-1}
		\text{ .}
	\end{align*}
	%
	Dada una $k$-\'{a}lgebra conmutativa $A$, existen biyecciones
	\begin{align*}
		\Homalg\big(\GL(2),A\big) & \,\simeq\,
			\GL[2](A) \quad\text{y} \\
		\Homalg\big(\SL(2),A\big) & \,\simeq\,
			\SL[2](A)
		\text{ ,}
	\end{align*}
	%
	pues, si
	\begin{math}
		\left[\begin{smallmatrix}
			\alpha & \beta \\ \gamma & \delta
		\end{smallmatrix}\right]\in\GL[2](A)
	\end{math}, existe \'{u}nico $f:\,\MM(2)[t]\rightarrow A$ tal que
	\begin{align*}
		f\,
		\left[\begin{matrix}
			a & b \\ c & d
		\end{matrix}\right]\,=\,
		\left[\begin{matrix}
			\alpha & \beta \\ \gamma & \delta
		\end{matrix}\right] & \quad\text{y}\quad
		f(t)\,=\,(\alpha\delta-\beta\gamma)^{-1}
		\text{ .}
	\end{align*}
	%
\end{frame}

\begin{frame}{$\GL(2)$ y $\SL(2)$ (cont.)}
	Queremos
	\begin{align*}
		\Delta & \,:\, \GL(2)\,\rightarrow\,\GL(2)^{\otimes 2}
			\,=\,\MM(2)^{\otimes 2}[t',t'']/I
		\text{ ,}
	\end{align*}
	%
	donde $I=\generado{(a'd'-b'c')\,t'-1,(a''d''-b''c'')\,t''-1}$.
	Definimos $\Delta:\,\MM(2)[t]\rightarrow\MM(2)^{\otimes 2}[t',t'']$,
	extendiendo por
	\begin{align*}
		\Delta(t) & \,=\,t'\,t''
		\text{ .}
	\end{align*}
	%
	Se cumple $\Delta((ad-bc)\,t-1)=0$ en $\GL(2)^{\otimes 2}$.
\end{frame}

\begin{frame}{$\GL(2)$ y $\SL(2)$ (cont.)}
	\begin{propoLinealGeneral}\label{propo:linealgeneral}
		$\Delta$, junto con $\varepsilon:\,\GL(2)\rightarrow k$ y
		$S:\,\GL(2)\rightarrow\GL(2)$ dados por
		\begin{align*}
			\varepsilon\,
				\begin{bmatrix} a & b \\ c & d \end{bmatrix}
				\,=\,\begin{bmatrix} 1 & \\ & 1 \end{bmatrix}
				& \quad\text{,}\quad
				\varepsilon(t)\,=\,1 \text{ ,} \\
			S\,\begin{bmatrix} a & b \\ c & d \end{bmatrix}
				\,=\,(ad-bc)^{-1}\,
				\begin{bmatrix} d & -b \\ -c & a \end{bmatrix}
				& \quad\text{,}\quad
				S(t) \,=\,t^{-1}
			\text{ ,}
		\end{align*}
		%
		se corresponden con el producto, la identidad y el inverso.
	\end{propoLinealGeneral}
\end{frame}

\subsection{Producto tensorial}

\begin{frame}{Producto tensorial de \'{a}lgebras}
	\begin{propoProductoTensorial}\label{propo:productotensorial}
		Sean $A,B$ $k$-\'{a}lgebras y sea $A\tensor[k]B$ el $k$-%
		m\'{o}dulo con producto
		\begin{align*}
			(a\tensor b)\,(a_1\tensor b_1) & \,=\,
				aa_1\tensor bb_1
			\text{ .}
		\end{align*}
		%
		$A\tensor[k]B$ es $k$-\'{a}lgebra y
		\begin{align*}
			\Homalg\big(A\tensor[k]B,C\big) & \,\simeq\,
				\Homalg\big(A,C\big)\,\times\,
				\Homalg\big(B,C\big)
			\text{ ,}
		\end{align*}
		%
		para toda \'{a}lgebra conmutativa $C$, dada por
		\begin{align*}
			& (f,g) \,\mapsto\, \big(\mu_C\circ(f\tensor g)\,:\,
				(a\tensor b)\mapsto f(a)\,g(b)\big)
			\text{ .}
		\end{align*}
		%
	\end{propoProductoTensorial}
\end{frame}

\begin{frame}{Relaci\'{o}n con $\Delta$}
	\begin{propoProductoDeCociente}\label{propo:productodecociente}
		Sea $A=k\{X\}/I$. Sean $X',X''$ copias de $X$ y sean
		$I'\triangleleft k\{X'\}$ e $I''\triangleleft k\{X''\}$ los
		ideales correspondientes a $I$. Entonces
		\begin{align*}
			A\tensor[k]A & \,\simeq\, A^{\otimes 2}\,:=\,
				k\{X'\sqcup X''\}/\generado{I',I'',X'X''-X''X'}
			\text{ ,}
		\end{align*}
		%
		v\'{\i}a $x'\mapsto x\tensor 1$ y $x''\mapsto 1\tensor x$.
	\end{propoProductoDeCociente}
	Por ejemplo, $k[x',x'']\simeq k[x]\tensor k[x]$.
	\begin{obsProductoDeMatrices}\label{obs:productodematrices}
		El ``producto de matrices'' $\Delta$ est\'{a} caracterizado por
		\begin{align*}
			\Delta\,\begin{bmatrix} a & b \\ c & d \end{bmatrix}
				& \,=\,
			\begin{bmatrix} a & b \\ c & d \end{bmatrix}\tensor
			\begin{bmatrix} a & b \\ c & d \end{bmatrix}
			\text{ .}
		\end{align*}
		%
		% y cumple $\Delta(ad-bc)=(ad-bc)\tensor(ad-bc)$.
	\end{obsProductoDeMatrices}
\end{frame}

\begin{frame}{Relaci\'{o}n con $\Delta$ (cont.)}
	\begin{obsRectaAfinAbeliana}\label{obs:rectaafinabeliana}
		$\Homalg\big(k[x],A\big)$ es abeliana:
		\begin{align*}
			\pull\Delta(f,g)(x) & \,=\,
				\mu_A(f\tensor g) (x\tensor 1 + 1\tensor x)
				\,=\, f(x) + g(x) \\
			\pull\Delta(g,f)(x) & \,=\, g(x) + f(x)
			\text{ ,}
		\end{align*}
		%
		$\Homalg\big(\GL(2),A\big)$, no:
		\begin{align*}
			\pull\Delta(f,g)(a) & \,=\,
				\mu_A\circ(f\tensor g)
					(a\tensor a+b\tensor c)
				\,=\,f(a)\,g(a)+f(b)\,g(c) \\
			\pull\Delta(g,f)(a) & \,=\,g(a)\,f(a)+g(b)\,f(c)
				\text{ .}
		\end{align*}
		%
	\end{obsRectaAfinAbeliana}
\end{frame}


\section{Co\'{a}lgebras y bi\'{a}lgebras}
\theoremstyle{plain}
\newtheorem{obsCoconmutativa}{Observaci\'{o}n}[section]
\newtheorem{ejemploCoalgebraProductoTensorial}[obsCoconmutativa]{Ejemplo}
\newtheorem{obsPolinomios}[obsCoconmutativa]{Observaci\'{o}n}
\newtheorem{obsBialgebraDeMatrices}[obsCoconmutativa]{Observaci\'{o}n}

%-------------

\subsection{Co\'{a}lgebras}
\begin{frame}{Definici\'{o}n}
	Una $k$-co\'{a}lgebra es un $k$-m\'{o}dulo $C$ y morfismos
	$\Delta:\,C\rightarrow C\tensor[k]C$ y $\varepsilon:\,C\rightarrow k$
	tales que los diagramas
	\begin{center}
		\begin{tikzcd}[column sep=small,ampersand replacement=\&]
			C\tensor C\tensor C \&
				C\tensor C\arrow[l,"\Delta\tensor\id"'] \\
			C\tensor C\arrow[u,"\id\tensor\Delta"] \&
				C \arrow[l,"\Delta"]\arrow[u,"\Delta"']
		\end{tikzcd}
		\begin{tikzcd}[column sep=small,ampersand replacement=\&]
			k\tensor C \& C\tensor C
				\arrow[l,"\varepsilon\tensor\id"']
				\arrow[r,"\id\tensor\varepsilon"]
				\& C\tensor k \\
			\& C
				\arrow[ur,"\simeq"']
				\arrow[u,"\Delta"']
				\arrow[ul,"\simeq"]
				\&
		\end{tikzcd}
	\end{center}
	conmutan. $(C,\Delta,\varepsilon)$ es coconmutativa, si
	$\Delta\circ\swap=\Delta$, donde $\swap(c\tensor c')=c'\tensor c$.
	$f:\,C\rightarrow D$ es morfismo de co\'{a}lgebras, si
	\begin{align*}
		\Delta_D\circ f & \,=\,(f\tensor f)\circ\Delta_C
			\quad\text{y} \\
		\varepsilon_D\circ f & \,=\,\varepsilon_C
	\end{align*}
	%
\end{frame}

\begin{frame}{Ejemplos}
	\begin{itemize}
		\item $(k,\Delta,\varepsilon)$ con $\Delta(1)=1\tensor 1$ y
			$\varepsilon(1)=1$;
		\item en el $k$-m\'{o}dulo $k[x]$ (polinomios),
			\begin{align*}
				\Delta(x) \,=\,x\tensor 1 + 1\tensor x
					& \quad\text{y}\quad
				\varepsilon(x)\,=\,0
				\text{ ;}
			\end{align*}
			%
		% \item en $k[x_1,\,\dots,\,x_n]$,
			% $\Delta(x_i)=x_i\tensor 1 + 1\tensor x_i$,
			% $\varepsilon(x_i)=0$;
		\item dado un conjunto $G$, en el $k$-m\'{o}dulo libre
			$k[G]$ con base $G$
			\begin{align*}
				\Delta(g) \,=\, g\tensor g
					& \quad\text{y}\quad
				\varepsilon(g)\,=\,1
				\text{ ;}
			\end{align*}
			%
		\item dada $(C,\Delta,\varepsilon)$,
			$C^\copp=(C,\Delta^\opp,\varepsilon)$, con
			$\Delta^\opp=\swap\circ\Delta$.
	\end{itemize}
	%
	\begin{obsCoconmutativa}\label{obs:coconmutativa}
		$k[G]^\copp=k[G]$.
	\end{obsCoconmutativa}
\end{frame}

\begin{frame}{Co\'{a}lgebra de matrices}
	$A=\MM[m\times m](k)$ con base $\{E_{ij}\}_{ij}$. Sea $\{x_{ij}\}_{ij}$
	la base dual en $\dual A$. Los morfismos de $k$-m\'{o}dulos
	determinados por
	\begin{align*}
		\Delta(x_{ij}) \,=\,\sum_{k=1}^{m}\,x_{ik}\tensor x_{kj}
			& \quad\text{y}\quad
		\varepsilon(x_{ij}) \,=\,\delta_{ij}
	\end{align*}
	%
	definen una co\'{a}lgebra $(\dual A,\Delta,\varepsilon)$:
	% ${\dual A}^{\copp}\not=\dual A$
	\begin{align*}
		(\id\tensor\Delta)\circ\Delta(x_{ij}) & \,=\,
			\sum_{k=1}^{m}\,x_{ik}\tensor\Delta(x_{kj}) \,=\,
			\sum_{k=1}^{m}\,\sum_{l=1}^{m}\,
				x_{ik}\tensor x_{kl}\tensor x_{lj} \\
		& \,=\, \sum_{l=1}^{m}\,\Delta(x_{il})\tensor x_{lj} \,=\,
			(\Delta\tensor\id)\circ\Delta(x_{ij})
		\text{ .}
	\end{align*}
	%
	\begin{obsCoconmutativa}\label{obs:nococonmutativa}
		${\dual A}^\copp\not=\dual A$.
	\end{obsCoconmutativa}
\end{frame}

\begin{frame}{Producto tensorial de co\'{a}lgebras}
	Dadas co\'{a}lgebras $(C,\Delta_C,\varepsilon_C)$,
	$(D,\Delta_D,\varepsilon_D)$, el $k$-m\'{o}dulo $C\tensor[k]D$ es
	co\'{a}lgebra con
	\begin{align*}
		\Delta & \,:=\,(\id[C]\tensor\swap\tensor\id[D])\circ
			(\Delta_C\tensor\Delta_D) \quad\text{y} \\
		\varepsilon & \,:=\,\varepsilon_C\tensor\varepsilon_D
	\end{align*}
	%
	($\tau(c\tensor d)=d\tensor c$).
	\begin{ejemploCoalgebraProductoTensorial}%
		\label{ejemplo:coalgebraproductotensorial}
		Dados conjuntos $G$ e $H$, el $k$-isomorfismo
		\begin{align*}
			k[G]\tensor[k]k[H] & \,\simeq\, k[G\times H]
			\text{ ,}
		\end{align*}
		%
		dado por $(g,h)\mapsto g\tensor h$, es isomorfismo de
		co\'{a}lgebras.
	\end{ejemploCoalgebraProductoTensorial}
\end{frame}

\subsection{Bi\'{a}lgebras}

\begin{frame}{La bi\'{a}lgebra $k$}
	Sobre el anillo $k$ tenemos $(k,\mu,\eta)$ y $(k,\Delta,\varepsilon)$.
	Son ``compatibles'': por ejemplo, evaluando en $1\tensor 1$,
	\begin{align*}
		\Delta\circ\mu & \,=\,
			(\mu\tensor\mu)\circ (\id\tensor\swap\tensor\id)\circ
				(\Delta\tensor\Delta)
				% \text{ ,} \\
		\text{ .}
	\end{align*}
	%
	$\mu:\,k\tensor k\rightarrow k$ y $\eta:\,k\rightarrow k$ son morfismos
	de co\'{a}lgebras; $\Delta:\,k\rightarrow k\tensor k$ y
	$\varepsilon:\,k\rightarrow k$ son morfismos de \'{a}lgebras.
	% \begin{align*}
		% \mu\circ (\varepsilon\tensor\varepsilon) & \,=\,
			% \varepsilon\circ\mu \,=\,\varepsilon\tensor\varepsilon
			% \text{ ,} \\
		% (\eta\tensor\eta)\circ\Delta & \,=\,\Delta\circ\eta \,=\,
			% \eta\tensor\eta	\quad\text{y} \\
		% \eta & \,=\,\varepsilon\circ\eta\,=\,\varepsilon
		% \text{ .}
	% \end{align*}
	% %
\end{frame}

\begin{frame}{Definici\'{o}n}
	Una bi\'{a}lgebra es un $k$-m\'{o}dulo $B$ con estructuras
	$(B,\mu,\eta)$ y $(B,\Delta,\varepsilon)$ tales que
	$\mu,\eta$ son morfismos de co\'{a}lgebras (equivalentemente,
	$\Delta,\varepsilon$ son morfismos de \'{a}lgebras):
	\begin{align*}
		\Delta\circ\mu & \,=\,
			(\mu\tensor\mu)\circ (\id\tensor\swap\tensor\id)\circ
				(\Delta\tensor\Delta)
				\text{ ,} \\
		\mu_k\circ (\varepsilon\tensor\varepsilon) & \,=\,
			\varepsilon\circ\mu \,=\,\varepsilon\tensor\varepsilon
			\text{ ,} \\
		(\eta\tensor\eta)\circ\Delta_k & \,=\,\Delta\circ\eta \,=\,
			\eta\tensor\eta	\quad\text{y} \\
		\eta_k & \,=\,\varepsilon\circ\eta\,=\,\varepsilon_k
		\text{ .}
	\end{align*}
	Un morfismo de bi\'{a}lgebras es un morfismo $f:\,B\rightarrow B'$ de
	\'{a}lgebras y co\'{a}lgebras:
	\begin{align*}
		\Delta_{B'}\circ f \,=\,(f\tensor f)\circ\Delta_B
			& \quad\text{,}\quad
			\varepsilon_{B'}\circ f \,=\,\varepsilon_B
			\text{ ,} \\
		f\circ\mu_B \,=\,\mu_{B'}\circ (f\tensor f)
			& \quad\text{y}\quad
			f\circ\eta_B \,=\,\eta_{B'}
		\text{ .}
	\end{align*}
	%
\end{frame}

\begin{frame}{Polinomios}
	En el \'{a}lgebra $k[x]$,
	\begin{align*}
		\Delta(x) \,=\,x\tensor 1 + 1\tensor x
			& \quad\text{y}\quad \varepsilon(x)\,=\,0
	\end{align*}
	%
	determinan morfismos de \'{a}lgebras. $(k[x],\Delta,\varepsilon)$ es
	co\'{a}lgebra:
	\begin{align*}
		(\Delta\tensor\id)\circ\Delta(x) & \,=\,
			(\Delta\tensor\id)(x\tensor 1 + 1\tensor x) \\
		& \,=\,(x\tensor 1 + 1\tensor x)\tensor 1 + 1\tensor 1\tensor x
			\text{ ,} \\
		(\id\tensor\Delta)\circ\Delta(x) & \,=\,
			(\id\tensor\Delta)(x\tensor 1 + 1\tensor x) \\
		& \,=\,x\tensor 1\tensor 1 + 1\tensor (x\tensor 1 + 1\tensor x)
		\text{ .}
	\end{align*}
	%
\end{frame}

\begin{frame}{Polinomios (cont.)}
	En $k[x_1,\,\dots,\,x_n]$,
	$\Delta(x_i)=x_i\tensor 1 + 1\tensor x_i$, $\varepsilon(x_i)=0$.
	$k[x_1,\,\dots,\,x_n]^\copp=k[x_1,\,\dots,\,x_n]$.

	\begin{obsPolinomios}\label{obs:polinomios}
		El isomorfismo de \'{a}lgebras
		$k[x',x'']\simeq k[x]\tensor k[x]$ dado por
		$\phi(x')\mapsto x\tensor 1$, $\phi(x'')\mapsto 1\tensor x$ es
		isomorfismo de co\'{a}lgebras:
		\begin{align*}
			(\phi\tensor\phi)\circ\Delta(x') & \,=\,
				\phi(x')\tensor\phi(1) + \phi(1)\tensor\phi(x')
				\\
			& \,=\,x\tensor 1\tensor 1\tensor 1 +
				1\tensor 1\tensor x\tensor 1 \\
			\Delta\circ\phi(x') & \,=\,
				(\id\tensor\swap\tensor\id)\circ
				(\Delta\tensor\Delta)(x\tensor 1) \\
			& \,=\,(\id\tensor\swap\tensor\id)(
				(x\tensor 1 +1\tensor x)\tensor 1\tensor 1)
			\text{ .}
		\end{align*}
		%
	\end{obsPolinomios}
\end{frame}

\begin{frame}{Bi\'{a}lgebra de matrices}
	En el \'{a}lgebra de polinomios $\MM(m)=k[x_{11},\,\cdots,\,x_{mm}]$,
	\begin{align*}
		\Delta(x_{ij}) \,=\,\sum_{k=1}^{m}\,
			x_{ik}\tensor x_{kj} & \quad\text{y}\quad
		\varepsilon(x_{ij}) \,=\,\delta_{ij}
	\end{align*}
	%
	determinan morfismos de \'{a}lgebras. Pero
	$(\MM(m),\Delta,\varepsilon)$ es co\'{a}lgebra (an\'{a}logo a
	co\'{a}lgebra de matrices).
	\begin{obsBialgebraDeMatrices}\label{obs:bialgebradematrices}
		Como co\'{a}lgebras
		$\MM(m)\not\simeq k[x_{11},\,\dots,\,x_{mm}]$
	\end{obsBialgebraDeMatrices}
\end{frame}

\begin{frame}{Bi\'{a}lgebra de un monoide}
	Sea $G$ un monoide con producto $\mu:\,G\times G\rightarrow G$ y unidad
	$e\in G$, $(k[G],\Delta,\varepsilon)$ la co\'{a}lgebra del conjunto.
	$(k[G],\mu,e)$ es \'{a}lgebra y $\mu$ y $1\mapsto e$ son morfismos de
	co\'{a}lgebras:
	\begin{align*}
		\Delta\circ\mu(x,y) & \,=\,\mu(x,y)\tensor\mu(x,y) \,=\,
			\mu_{\tensor}\big((x\tensor x),(y\tensor y)\big) \\
		& \,=\,\mu(\Delta(x),\Delta(y)) \quad\text{y} \\
		\varepsilon\circ\mu(x,y) & \,=\,1 \,=\,
			\mu(\varepsilon(x),\varepsilon(y))
		\text{ .}
	\end{align*}
	%
\end{frame}


\section{\'{A}lgebras de Hopf}
\theoremstyle{plain}
\newtheorem{defAntipoda}{Definici\'{o}n}[section]
\newtheorem{obsHopf}[defAntipoda]{Observaci\'{o}n}
\newtheorem{teoGrupoDeMorfismos}[defAntipoda]{Teorema}

%-------------

\subsection{La ant\'{\i}poda}

\begin{frame}{Convoluci\'{o}n y ant\'{\i}poda}
	Sean $(A,\mu,\eta)$, $(C,\Delta,\varepsilon)$. La
	\emph{convoluci\'{o}n} de $f,g\in\Hom[k](C,A)$, es
	\begin{align*}
		f \convol g & \,:=\,\mu\circ(f\tensor g)\circ\Delta
			\,\in\,\Hom[k](C,A)
		\text{ .}
	\end{align*}
	%
	Si $\Delta(x)=\sum_i\,x_i'\tensor x_i''$,
	$f \convol g(x) =\sum_i\,f(x_i')\,g(x_i'')$
	\begin{defAntipoda}\label{def:antipoda}
		Una \emph{ant\'{\i}poda} en $(H,\mu,\eta,\Delta,\varepsilon)$
		es $S\in\Endo[k](H)$ tal que
		\begin{align*}
			S\convol\id[H] & \,=\,\id[H]\convol S \,=\,
				\eta\circ\varepsilon
			\text{ .}
		\end{align*}
		%
		\begin{math}
			\sum_i\,x_i'S(x_i'')=\varepsilon(x)\cdot 1=
				\sum_i\,S(x_i')x_i''
		\end{math}. Si existe, es \'{u}nica.
	\end{defAntipoda}
\end{frame}

\begin{frame}{Definici\'{o}n}
	Un \emph{\'{a}lgebra de Hopf} es una bi\'{a}lgebra $H$ con
	ant\'{\i}poda. Un morfismo de \'{a}lgebras de Hopf es un morfismo de
	bi\'{a}lgebras.
	\begin{obsHopf}\label{obs:hopf}
		\begin{math}
			S:\,H\rightarrow H^{\opp\,\copp}=
				(H,\mu^\opp,\eta,\Delta^\opp,\varepsilon)
		\end{math} es morfismo de bi\'{a}lgebras.
		Si $H=k\{X\}/I$ es bi\'{a}lgebra, dado un morfismo de
		\'{a}lgebras $S:\,H\rightarrow H^\opp$, basta verificar la
		condici\'{o}n de ant\'{\i}poda en $X$.
	\end{obsHopf}
\end{frame}

\begin{frame}{El \'{a}lgebra de un grupo}
	Sean $G$ un grupo, $k[G]$ la bi\'{a}lgebra del monoide.
	\begin{align*}
		S(g) & \,=\,g^{-1}
	\end{align*}
	%
	$g\in G$, define una ant\'{\i}poda: $\Delta(g)=g\tensor g$ y
	$\varepsilon(g)=1$. Rec\'{\i}procamente, si $G$ es monoide y
	$S:\,k[G]\rightarrow k[G]$ es ant\'{\i}poda,
	\begin{align*}
		g\,S(g) & \,=\,S(g)\,g \,=\,\varepsilon(g)\,1\,=\,1
	\end{align*}
	%
	implica $S(g)\in G$ y es inverso de $g$.
\end{frame}

\subsection{Ejemplos}

\begin{frame}{$\GL(2)$ y $\SL(2)$}
	$\GL(2)$ y $\SL(2)$ son bi\'{a}lgebras conmutativas con
	$\Delta,\varepsilon$ dados por
	\begin{align*}
		\Delta\,\begin{bmatrix} a & b \\ c & d \end{bmatrix} \,=\,
			\begin{bmatrix} a & b \\ c & d \end{bmatrix} \tensor
			\begin{bmatrix} a & b \\ c & d \end{bmatrix}
			& \quad\text{,}\quad
			\Delta(t)\,=\,t\tensor t \text{ ,} \\
		\varepsilon\,\begin{bmatrix} a & b \\ c & d \end{bmatrix} \,=\,
			\begin{bmatrix} 1 & \\ & 1 \end{bmatrix}
			& \quad\text{,}\quad
			\varepsilon(t) \,=\,1
		\text{ .}
	\end{align*}
	%
	$\Delta$ no es coconmutativa:
	\begin{align*}
		\Delta(a) & \,=\,a\tensor a+b\tensor c\,\not=\,
			a\tensor a+c\tensor b \,=\,\swap\circ\Delta(a)
		\text{ .}
	\end{align*}
	%
	La ant\'{\i}poda est\'{a} dada por
	\begin{align*}
		S\,\begin{bmatrix} a & b \\ c & d \end{bmatrix} \,=\,
			(ad-bc)^{-1}\,
			\begin{bmatrix} d & -b \\ -c & a \end{bmatrix}
			& \quad\text{,}\quad
			S(t) \,=\,t^{-1}
		\text{ .}
	\end{align*}
	%
\end{frame}

\begin{frame}{El grupo de un \'{a}lgebra}
	Dada $(H,\Delta,\varepsilon)$,
	\begin{align*}
		\grouplike H & \,:=\,\big\{x\in H\,:\,x\not=0,\,
			\Delta(x)=x\tensor x\big\}
		\text{ .}
	\end{align*}
	%
	Si $H$ es bi\'{a}lgebra, es monoide con unidad $\Delta(1)=1\tensor 1$ y
	\begin{align*}
		\Delta(xy) & \,=\,\Delta(x)\,\Delta(y)\,=\,
			(x\tensor x)\,(y\tensor y) \,=\,xy\tensor xy
		\text{ .}
	\end{align*}
	%
	Si $H$ es de Hopf, $x\mapsto S(x)$ define un inverso en
	$\grouplike H$:
	\begin{align*}
		\swap\circ(S\tensor S)\circ\Delta & \,=\,
			\Delta\circ S
		\text{ .}
	\end{align*}
	%
	Si $H=k[G]$, entonces $\grouplike{k[G]}=G$.
\end{frame}

\begin{frame}{El grupo $\Homalg\big(H,A\big)$ (cont.)}
	\begin{teoGrupoDeMorfismos}\label{thm:grupodemorfismos}
		Sean $H$ un \'{a}lgebra de Hopf y $A$ un \'{a}lgebra
		conmutativa. Los conjuntos $\Homalg\big(H,A\big)$ son grupos
		con la convoluci\'{o}n heredada de $\Hom[k](H,A)$. El inverso
		de $\psi:\,H\rightarrow A$ est\'{a} dado por $\psi\circ S$.
	\end{teoGrupoDeMorfismos}
	Comprobar que
	\begin{itemize}
		% \item
			% \begin{math}
				% (\psi\convol\varphi)\circ\mu_H =
				% \mu_A\circ((\psi\convol\varphi)\tensor
					% (\psi\convol\varphi))
			% \end{math},
		% \item
			% \begin{math}
				% (\psi\convol\varphi)\circ\eta_H =\eta_A
			% \end{math},
		\item $\psi\convol\varphi\in\Homalg\big(H,A\big)$,
		\item $c =\eta_A\circ\varepsilon_H\in\Homalg\big(H,A\big)$ es
			unidad,
		\item $\psi\circ S\in\Homalg\big(H,A\big)$ es inverso.
	\end{itemize}
	%
\end{frame}

\begin{frame}{El grupo $\Homalg\big(H,A\big)$ (cont.)}
	Si $\psi,\varphi\in\Homalg\big(H,A\big)$, como $H$ es bi\'{a}lgebra,
	\begin{align*}
		(\psi\convol\varphi)\circ\mu_H & \,=\
			\mu_A\circ (\psi\tensor\varphi)\circ\Delta_H\circ\mu_H
				\\
		& \,=\,	\mu_A\,(\psi\tensor\varphi)\,(\mu_H\tensor\mu_H)\,
			(\id[H]\tensor\swap[H]\tensor\id[H])\,
			(\Delta_H\tensor\Delta_H) \\
		& \,=\,\mu_A\,((\mu_A\,(\psi\tensor\varphi)\,\Delta_H)\tensor
			(\mu_A\,(\psi\tensor\varphi)\,\Delta_H)) \\
		& \,=\,\mu_A((\psi\convol\varphi)\tensor(\psi\convol\varphi))
		\text{ .}
	\end{align*}
	%
	Si $c=\eta_A\circ\varepsilon_H$,
	\begin{align*}
		\psi\convol c & \,=\,\mu_A\,(\psi\tensor\eta_A\varepsilon_H)\,
								\Delta_H \\
		& \,=\,\mu_A\,(\id[A]\tensor\eta_A)\,(\psi\tensor\id[k])\,
			(\id[H]\tensor\varepsilon_H)\,\Delta_H \\
		& \,=\,\psi\tensor\id[k]\,=\,\psi
		\text{ .}
	\end{align*}
	%
\end{frame}

\begin{frame}{El grupo $\Homalg\big(H,A\big)$ (cont.)}
	$G=(\Homalg\big(H,A\big),\convol,\eta_A\circ\varepsilon_H)$ es un
	monoide y podemos definir $(k[G],\mu,\eta,\Delta,\varepsilon)$. Sea
	$S_H$ la ant\'{\i}poda en $H$ y sea
	$S(\psi)=\psi\circ S_H\in\Hom[k](H,A)$. Se verifica que
	\begin{align*}
		\mu\circ(\id\tensor S)\circ\Delta & \,=\,
			\eta\circ\varepsilon\,=\,
			\mu\circ(S\tensor\id)\circ\Delta
	\end{align*}
	%
	Por ejemplo, $\eta\circ\varepsilon(\psi)=\eta(1)=\eta_A\varepsilon_H$ y
	\begin{align*}
		\mu\circ(\id\tensor S)\circ\Delta(\psi) & \,=\,
			\mu(\psi\tensor S(\psi)) \,=\,\psi\convol S(\psi) \\
		& \,=\, \mu_A\,(\psi\tensor\psi)\,(\id[H]\tensor S_H)\,
								\Delta_H \\
		& \,=\,\psi\,\mu_H\,(\id[H]\tensor S_H)\,\Delta_H \,=\,
			(\psi\,\eta_H)\,\varepsilon_H \\
		& \,=\,\eta_A\varepsilon_H
		\text{ .}
	\end{align*}
	%
\end{frame}

\begin{frame}{El grupo $\Homalg\big(H,A\big)$ (cont.)}
	Resta verificar que $S(\psi)\in\Homalg\big(H,A\big)$:
	\begin{align*}
		(\psi\,S_H)\,\mu_H & \,=\,\psi\,(\mu_H\,\swap[H])\,
			(S_H\tensor S_H) \,=\,
			\mu_A\,(\psi\tensor\psi)\,\swap[H]\,(S_H\tensor S_H) \\
		& \,=\,(\mu_A\,\swap[A])\,((\psi\,S_H)\tensor (\psi\,S_H))
	\end{align*}
	%
	Si $\mu_A\circ\swap[A]=\mu_A$, entonces $\psi\circ S_H$ respeta
	productos. En cuanto a la unidad,
	\begin{align*}
		(\psi\,S_H)\,\eta_H & \,=\,\psi\,\eta_H\,=\,\eta_H
		\text{ .}
	\end{align*}
\end{frame}



\section{Grupos afines}
\theoremstyle{plain}
\newtheorem{coroGrupoDeMorfismos}{Corolario}

%-------------

\subsection{Repaso}

\begin{frame}{El grupo $\Homalg\big(H,-\big)$}
	El producto, la unidad y el inverso en $\Homalg\big(H,A\big)$ est\'{a}n
	dados por
	\begin{align*}
		\pull{\Delta_H} & \,:\,\Homalg\big(H\tensor H,A\big)
			\,\rightarrow\,\Homalg\big(H,A\big) \text{ ,} \\
		\pull{\varepsilon_H} & \,:\,\Homalg\big(k,A\big)
			\,\rightarrow\,\Homalg\big(H,A\big) \quad\text{y} \\
		\pull{S_H} & \,:\,\Homalg\big(H,A\big) \,\rightarrow\,
			\Homalg\big(H,A\big)
		\text{ ,}
	\end{align*}
	%
	si identificamos
	\begin{align*}
		\Homalg\big(H\tensor H,A\big) & \,\simeq\,
			\Homalg\big(H,A\big)\times\Homalg\big(H,A\big) \\
		\Homalg\big(k,A\big) & \,=\,\{\eta_A\}\simeq\{1\}
	\end{align*}
	%
\end{frame}

\begin{frame}{El grupo $\Homalg\big(H,-\big)$ (cont.)}
	$G_A=\Homalg\big(H,A\big)$,
	\begin{math}
		G_A^{\otimes i}=\Homalg\big(H^{\otimes i},A\big) % \simeq
			% G_A\times\,\cdots\,\times G_A
	\end{math}.
	\begin{center}
		\begin{tikzcd}[ampersand replacement=\&]
			H\tensor H\tensor H \&
				H\tensor H
					\arrow[l,"\id\tensor\Delta"'] \\
			H\tensor H \arrow[u,"\Delta\tensor\id"] \&
				H \arrow[u,"\Delta"']
					\arrow[l,"\Delta"]
		\end{tikzcd}
		$\rightsquigarrow$
		\begin{tikzcd}[ampersand replacement=\&]
			G_A^{\otimes 3} \arrow[r,"\pull{(\id\tensor\Delta)}"]
				\arrow[d,"\pull{(\Delta\tensor\id)}"'] \&
				G_A^{\otimes 2} \arrow[d,"\pull\Delta"] \\
			G_A^{\otimes 2} \arrow[r,"\pull\Delta"'] \&
				G_A
		\end{tikzcd}
	\end{center}
	Expl\'{\i}citamente, $f\convol (g\convol h)=(f\convol g)\convol h$.
\end{frame}

\begin{frame}{El funtor $\Homalg\big(H,-\big)$}
	Si $\varphi:\,A\rightarrow B$ es morfismo de \'{a}lgebras, se obtiene
	una funci\'{o}n $\push\varphi:\,G_A\,\rightarrow\,G_B$ que es,
	adem\'{a}s, morfismo de grupos:
	\begin{center}
		\begin{tikzcd}[column sep=small,ampersand replacement=\&]
			G_A^{\otimes 2} \arrow[r,"\pull\Delta_A"]
				\arrow[d,"\push\varphi"'] \&
				G_A \arrow[d,"\push\varphi"] \\
			G_B^{\otimes 2} \arrow[r,"\pull\Delta_B"'] \&
				G_B
		\end{tikzcd}
	\end{center}
	\begin{coroGrupoDeMorfismos}\label{coro:grupodemorfismos}
		Existe $G:\,\CommAlg[k]\rightarrow\Grp$ tal que
		\begin{align*}
			U\circ G & \,=\,\Homalg\big(H,-\big)
		\end{align*}
		%
	\end{coroGrupoDeMorfismos}
\end{frame}

\begin{frame}{Grupos en $\CommAlg[k]\rightarrow\Set$}
	En las categor\'{\i}as $\CommAlg[k]\rightarrow\Set$ y
	$\CommAlg[k]\rightarrow\Grp$ hay productos y objetos terminales:
	\begin{align*}
		(G\times G')(\varphi) & \,=\,
			G(\varphi)\times G'(\varphi)\,:\, \\
			& \qquad
			G(A)\times G'(A)\,\rightarrow\,G(B)\times G'(B) \\
		t \,:\, G(A) & \,\xrightarrow{\cdot}\,\mathbf{1}(A)
	\end{align*}
	%
	Podemos definir grupos.
\end{frame}

\begin{frame}{El grupo $\Homalg\big(H,-\big)$ (cont.)}
	Existen transformaciones naturales
	\begin{align*}
		\pull\Delta:\,UG\times UG\xrightarrow{\cdot} UG
			&\quad\text{,}\quad
		\pull\varepsilon:\,U\mathbf{1}\xrightarrow{\cdot}UG
			\quad\text{y}\quad
		\pull S:\,UG\xrightarrow{\cdot} UG
	\end{align*}
	%
	tales que
	\begin{align*}
		\pull\Delta\circ(\id\times\pull\Delta) & \,=\,
			\pull\Delta\circ(\pull\Delta\times\id) \\
		\pull\Delta\circ((\pull\varepsilon\circ t)\times\id)\circ\diag
			& \,=\,\id\,=\,
			\pull\Delta\circ(\id\times (\pull\varepsilon\circ t))
				\circ\diag \\
		\pull\Delta\circ(\id\times\pull S)\circ\diag & \,=\,
			\pull\varepsilon\circ t \,=\,
			\pull\Delta\circ(\pull S\times\id)\circ\diag
	\end{align*}
	%
\end{frame}

\subsection{Equivalencia}

\begin{frame}{Representabilidad}
	Sean $G,G':\,\CommAlg[k]\rightarrow\Grp$, $U:\,\Grp\rightarrow\Set$,
	$\push U:\,\Grp^{\CommAlg[k]}\rightarrow\Set^{\CommAlg[k]}$.
	\begin{itemize}
		\item Dada $\tilde\tau:\,UG\xrightarrow\cdot UG'$,
			?`existe $\tau:\,G\xrightarrow\cdot G'$ tal que
			$\push U\tau=\tilde\tau$?
		\item Dadas $\tau_1,\tau_2:\,G\xrightarrow\cdot G'$ tales que
			$\push U\tau_1=\push U\tau_2$, ?`$\tau_1=\tau_2$?
	\end{itemize}
	Porque $U$ es fiel, $\push U\tau_1=\push U\tau_2$ implica
	$\tau_1=\tau_2$.
\end{frame}

\begin{frame}{Representabilidad (cont.)}
	$\tilde\tau:\,UG\xrightarrow\cdot UG'$ induce
	\begin{center}
		\begin{tikzcd}[column sep=small,ampersand replacement=\&]
			U(G(A)) \arrow[r,"\tilde\tau_A"]
				\arrow[d,"U(G\varphi)"'] \&
				U(G'(A)) \arrow[d,"U(G'\varphi)"] \\
			U(G(B)) \arrow[r,"\tilde\tau_B"'] \& U(G'(B))
		\end{tikzcd}
	\end{center}
	Que $\tilde\tau_A$ sea morfismo de grupos significa que existe
	\begin{center}
		\begin{tikzcd}[column sep=small,ampersand replacement=\&]
			UG(A)\times UG(A) \arrow[r,"m^{G}_A"]
				\arrow[d,"\tilde\tau_A\times\tilde\tau_A"'] \&
				UG(A) \arrow[d,"\tilde\tau_A"] \\
			UG'(A)\times UG'(A) \arrow[r,"m^{G'}_A"'] \& UG'(A)
		\end{tikzcd}
	\end{center}
	La multiplicaci\'{o}n deber\'{\i}a ser natural en $G$, tambi\'{e}n.
\end{frame}

\begin{frame}{Representabilidad (cont.)}
	$U\circ G=\Homalg\big(H,-\big)$ y $U\circ G'=\Homalg\big(H',-\big)$,
	por el lema de Yoneda,
	\begin{align*}
		\Nat(U\circ G,U\circ G') & \,\simeq\,U\circ G'(H) \,=\,
			\Homalg\big(H',H\big)
	\end{align*}
	%
	(morfismos de \emph{\'{a}lgebras}) v\'{\i}a $\phi\mapsto\pull\phi$
	\begin{itemize}
		\item Dada $\phi$, ?`existe $\tau:\,G\xrightarrow\cdot G'$ tal
			que $\push U\tau =\pull\phi$?
	\end{itemize}
	%
\end{frame}

\begin{frame}{Representabilidad (cont.)}
	El diagrama
	\begin{center}
		\begin{tikzcd}[ampersand replacement=\&]
			UG(A)\times UG(A) \arrow[r,"\pull\phi\times\pull\phi"]
				\arrow[d,"\pull\Delta_H"'] \&
			UG'(A)\times UG'(A)\arrow[d,"\pull\Delta_{H'}"'] \\
			UG(A)\arrow[r,"\pull\phi"'] \& UG'(A)
		\end{tikzcd}
	\end{center}
	conmuta si y s\'{o}lo si $\phi:\,H'\rightarrow H$ es morfismo de
	\'{a}lgebras de Hopf.
\end{frame}

\begin{frame}{Relaci\'{o}n con grupos afines}
	La aplicaci\'{o}n
	\begin{align*}
		H\,\mapsto\,\Homalg\big(H,-\big) &\quad\text{,}\quad
			\phi\,\mapsto\,\pull\phi
	\end{align*}
	%
	define un funtor contravariante fiel y pleno de la categor\'{\i}a de
	\'{a}lgebras de Hopf en la categor\'{\i}a de grupos afines,
	$(G,m,u,\sigma)$ donde
	\begin{itemize}
		\item $G:\,\CommAlg[k]\rightarrow\Grp$ es funtor,
		\item $UG$ es representable: existe $H$ tal que
			$UG\simeq\Homalg\big(H,-\big)$,
		\item $m,u,\sigma$ son transformaciones naturales que hacen de
			$UG$ un grupo en $\Set^{\CommAlg[k]}$.
	\end{itemize}
\end{frame}


\end{document}
