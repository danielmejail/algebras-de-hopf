\theoremstyle{plain}
\newtheorem{coroGrupoDeMorfismos}{Corolario}

%-------------

\subsection{Repaso}

\begin{frame}{El grupo $\Homalg\big(H,-\big)$}
	El producto, la unidad y el inverso en $\Homalg\big(H,A\big)$ est\'{a}n
	dados por
	\begin{align*}
		\pull{\Delta_H} & \,:\,\Homalg\big(H\tensor H,A\big)
			\,\rightarrow\,\Homalg\big(H,A\big) \text{ ,} \\
		\pull{\varepsilon_H} & \,:\,\Homalg\big(k,A\big)
			\,\rightarrow\,\Homalg\big(H,A\big) \quad\text{y} \\
		\pull{S_H} & \,:\,\Homalg\big(H,A\big) \,\rightarrow\,
			\Homalg\big(H,A\big)
		\text{ ,}
	\end{align*}
	%
	si identificamos
	\begin{align*}
		\Homalg\big(H\tensor H,A\big) & \,\simeq\,
			\Homalg\big(H,A\big)\times\Homalg\big(H,A\big) \\
		\Homalg\big(k,A\big) & \,=\,\{\eta_A\}\simeq\{1\}
	\end{align*}
	%
\end{frame}

\begin{frame}{El grupo $\Homalg\big(H,-\big)$ (cont.)}
	$G_A=\Homalg\big(H,A\big)$,
	\begin{math}
		G_A^{\otimes i}=\Homalg\big(H^{\otimes i},A\big) % \simeq
			% G_A\times\,\cdots\,\times G_A
	\end{math}.
	\begin{center}
		\begin{tikzcd}[ampersand replacement=\&]
			H\tensor H\tensor H \&
				H\tensor H
					\arrow[l,"\id\tensor\Delta"'] \\
			H\tensor H \arrow[u,"\Delta\tensor\id"] \&
				H \arrow[u,"\Delta"']
					\arrow[l,"\Delta"]
		\end{tikzcd}
		$\rightsquigarrow$
		\begin{tikzcd}[ampersand replacement=\&]
			G_A^{\otimes 3} \arrow[r,"\pull{(\id\tensor\Delta)}"]
				\arrow[d,"\pull{(\Delta\tensor\id)}"'] \&
				G_A^{\otimes 2} \arrow[d,"\pull\Delta"] \\
			G_A^{\otimes 2} \arrow[r,"\pull\Delta"'] \&
				G_A
		\end{tikzcd}
	\end{center}
	Expl\'{\i}citamente, $f\convol (g\convol h)=(f\convol g)\convol h$.
\end{frame}

\begin{frame}{El funtor $\Homalg\big(H,-\big)$}
	Si $\varphi:\,A\rightarrow B$ es morfismo de \'{a}lgebras, se obtiene
	una funci\'{o}n $\push\varphi:\,G_A\,\rightarrow\,G_B$ que es,
	adem\'{a}s, morfismo de grupos:
	\begin{center}
		\begin{tikzcd}[column sep=small,ampersand replacement=\&]
			G_A^{\otimes 2} \arrow[r,"\pull\Delta_A"]
				\arrow[d,"\push\varphi"'] \&
				G_A \arrow[d,"\push\varphi"] \\
			G_B^{\otimes 2} \arrow[r,"\pull\Delta_B"'] \&
				G_B
		\end{tikzcd}
	\end{center}
	\begin{coroGrupoDeMorfismos}\label{coro:grupodemorfismos}
		Existe $G:\,\CommAlg[k]\rightarrow\Grp$ tal que
		\begin{align*}
			U\circ G & \,=\,\Homalg\big(H,-\big)
		\end{align*}
		%
	\end{coroGrupoDeMorfismos}
\end{frame}

\begin{frame}{Grupos en $\CommAlg[k]\rightarrow\Set$}
	En las categor\'{\i}as $\CommAlg[k]\rightarrow\Set$ y
	$\CommAlg[k]\rightarrow\Grp$ hay productos y objetos terminales:
	\begin{align*}
		(G\times G')(\varphi) & \,=\,
			G(\varphi)\times G'(\varphi)\,:\, \\
			& \qquad
			G(A)\times G'(A)\,\rightarrow\,G(B)\times G'(B) \\
		t \,:\, G(A) & \,\xrightarrow{\cdot}\,\mathbf{1}(A)
	\end{align*}
	%
	Podemos definir grupos.
\end{frame}

\begin{frame}{El grupo $\Homalg\big(H,-\big)$ (cont.)}
	Existen transformaciones naturales
	\begin{align*}
		\pull\Delta:\,UG\times UG\xrightarrow{\cdot} UG
			&\quad\text{,}\quad
		\pull\varepsilon:\,U\mathbf{1}\xrightarrow{\cdot}UG
			\quad\text{y}\quad
		\pull S:\,UG\xrightarrow{\cdot} UG
	\end{align*}
	%
	tales que
	\begin{align*}
		\pull\Delta\circ(\id\times\pull\Delta) & \,=\,
			\pull\Delta\circ(\pull\Delta\times\id) \\
		\pull\Delta\circ((\pull\varepsilon\circ t)\times\id)\circ\diag
			& \,=\,\id\,=\,
			\pull\Delta\circ(\id\times (\pull\varepsilon\circ t))
				\circ\diag \\
		\pull\Delta\circ(\id\times\pull S)\circ\diag & \,=\,
			\pull\varepsilon\circ t \,=\,
			\pull\Delta\circ(\pull S\times\id)\circ\diag
	\end{align*}
	%
\end{frame}

\subsection{Equivalencia}

\begin{frame}{Representabilidad}
	Sean $G,G':\,\CommAlg[k]\rightarrow\Grp$, $U:\,\Grp\rightarrow\Set$,
	$\push U:\,\Grp^{\CommAlg[k]}\rightarrow\Set^{\CommAlg[k]}$.
	\begin{itemize}
		\item Dada $\tilde\tau:\,UG\xrightarrow\cdot UG'$,
			?`existe $\tau:\,G\xrightarrow\cdot G'$ tal que
			$\push U\tau=\tilde\tau$?
		\item Dadas $\tau_1,\tau_2:\,G\xrightarrow\cdot G'$ tales que
			$\push U\tau_1=\push U\tau_2$, ?`$\tau_1=\tau_2$?
	\end{itemize}
	Porque $U$ es fiel, $\push U\tau_1=\push U\tau_2$ implica
	$\tau_1=\tau_2$.
\end{frame}

\begin{frame}{Representabilidad (cont.)}
	$\tilde\tau:\,UG\xrightarrow\cdot UG'$ induce
	\begin{center}
		\begin{tikzcd}[column sep=small,ampersand replacement=\&]
			U(G(A)) \arrow[r,"\tilde\tau_A"]
				\arrow[d,"U(G\varphi)"'] \&
				U(G'(A)) \arrow[d,"U(G'\varphi)"] \\
			U(G(B)) \arrow[r,"\tilde\tau_B"'] \& U(G'(B))
		\end{tikzcd}
	\end{center}
	Que $\tilde\tau_A$ sea morfismo de grupos significa que existe
	\begin{center}
		\begin{tikzcd}[column sep=small,ampersand replacement=\&]
			UG(A)\times UG(A) \arrow[r,"m^{G}_A"]
				\arrow[d,"\tilde\tau_A\times\tilde\tau_A"'] \&
				UG(A) \arrow[d,"\tilde\tau_A"] \\
			UG'(A)\times UG'(A) \arrow[r,"m^{G'}_A"'] \& UG'(A)
		\end{tikzcd}
	\end{center}
	La multiplicaci\'{o}n deber\'{\i}a ser natural en $G$, tambi\'{e}n.
\end{frame}

\begin{frame}{Representabilidad (cont.)}
	$U\circ G=\Homalg\big(H,-\big)$ y $U\circ G'=\Homalg\big(H',-\big)$,
	por el lema de Yoneda,
	\begin{align*}
		\Nat(U\circ G,U\circ G') & \,\simeq\,U\circ G'(H) \,=\,
			\Homalg\big(H',H\big)
	\end{align*}
	%
	(morfismos de \emph{\'{a}lgebras}) v\'{\i}a $\phi\mapsto\pull\phi$
	\begin{itemize}
		\item Dada $\phi$, ?`existe $\tau:\,G\xrightarrow\cdot G'$ tal
			que $\push U\tau =\pull\phi$?
	\end{itemize}
	%
\end{frame}

\begin{frame}{Representabilidad (cont.)}
	El diagrama
	\begin{center}
		\begin{tikzcd}[ampersand replacement=\&]
			UG(A)\times UG(A) \arrow[r,"\pull\phi\times\pull\phi"]
				\arrow[d,"\pull\Delta_H"'] \&
			UG'(A)\times UG'(A)\arrow[d,"\pull\Delta_{H'}"'] \\
			UG(A)\arrow[r,"\pull\phi"'] \& UG'(A)
		\end{tikzcd}
	\end{center}
	conmuta si y s\'{o}lo si $\phi:\,H'\rightarrow H$ es morfismo de
	\'{a}lgebras de Hopf.
\end{frame}

\begin{frame}{Relaci\'{o}n con grupos afines}
	La aplicaci\'{o}n
	\begin{align*}
		H\,\mapsto\,\Homalg\big(H,-\big) &\quad\text{,}\quad
			\phi\,\mapsto\,\pull\phi
	\end{align*}
	%
	define un funtor contravariante fiel y pleno de la categor\'{\i}a de
	\'{a}lgebras de Hopf en la categor\'{\i}a de grupos afines,
	$(G,m,u,\sigma)$ donde
	\begin{itemize}
		\item $G:\,\CommAlg[k]\rightarrow\Grp$ es funtor,
		\item $UG$ es representable: existe $H$ tal que
			$UG\simeq\Homalg\big(H,-\big)$,
		\item $m,u,\sigma$ son transformaciones naturales que hacen de
			$UG$ un grupo en $\Set^{\CommAlg[k]}$.
	\end{itemize}
\end{frame}
