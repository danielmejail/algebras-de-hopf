\theoremstyle{plain}
\newtheorem{defAlgebra}{Definici\'{o}n}[section]
\newtheorem{ejemploLibre}[defAlgebra]{Ejemplos}
\newtheorem{propoLibre}[defAlgebra]{Proposici\'{o}n}
\newtheorem{ejemploCocienteDeLibre}[defAlgebra]{Ejemplo}
\newtheorem{propoRectaAfin}[defAlgebra]{Proposici\'{o}n}
\newtheorem{propoMultiplicativo}[defAlgebra]{Proposici\'{o}n}
\newtheorem{coroMultiplicativo}[defAlgebra]{Corolario}
\newtheorem{propoProductoDeMatrices}[defAlgebra]{Proposici\'{o}n}
\newtheorem{obsProductoDeMatrices}[defAlgebra]{Observaci\'{o}n}
\newtheorem{propoLinealGeneral}[defAlgebra]{Proposici\'{o}n}
\newtheorem{propoProductoTensorial}[defAlgebra]{Proposici\'{o}n}
\newtheorem{propoProductoDeCociente}[defAlgebra]{Proposici\'{o}n}
\newtheorem{obsRectaAfinAbeliana}[defAlgebra]{Observaci\'{o}n}

%------------


\subsection{Definiciones}

\begin{frame}{Definici\'{o}n y algunas construcciones}
	$k$: anillo conmutativo (con unidad).

	\begin{defAlgebra}\label{def:algebra}
		Una $k$-\'{a}lgebra es un anillo $A$ junto con un morfismo de
		anillos $\eta_A:\,k\rightarrow A$.
		\begin{align*}
			\Homalg(A,B) & \,=\,
				\Big\{f:\,A\rightarrow B
				\text{ de anillos, } f\circ\eta_A=\eta_B
				\Big\}
		\end{align*}
		%
	\end{defAlgebra}
	%
	$A$ es $k$-m\'{o}dulo con $(\lambda,a)\mapsto\eta_A(\lambda)\,a$ y
	$\mu_A:\,A\times A\rightarrow A$ es $k$-bilineal.
\end{frame}

\begin{frame}{El \'{a}lgebra libre}
	Dado un conjunto $X$, una \emph{palabra} en $X$ es $x_1\,\cdots\,x_n$ o
	$\varnothing$. Definimos $k\{X\}$, el $k$-m\'{o}dulo libre en las
	palabras en $X$. Es un \'{a}lgebra con
	\begin{align*}
		(x_{i_1}\,\cdots\,x_{i_n})\,(x_{i_{n+1}}\,\cdots\,x_{i_m})
			& \,=\,x_{i_1}\,\cdots\,x_{i_n}\,x_{i_{n+1}}\,\cdots\,
				x_{i_m}
		\text{ .}
	\end{align*}
	%
	\begin{ejemploLibre}\label{ejemplo:libre}
		\begin{itemize}
			\item $k\{x\}=k[x]$, polinomios en una variable;
			\item $k\{x,y\}\not=k[x,y]$, pues $xy\not=yx$.
		\end{itemize}
	\end{ejemploLibre}
	\begin{propoLibre}\label{propo:libre}
		Dados un conjunto $X$, una $k$-\'{a}lgebra $A$ y una
		\emph{funci\'{o}n} $f:\,X\rightarrow A$, existe un \'{u}nico
		morfismo $\tilde f:\,k\{X\}\rightarrow A$ tal que
		$\tilde f(x)=f(x)$, si $x\in X$.
	\end{propoLibre}
\end{frame}

\begin{frame}{El \'{a}lgebra libre (cont.)}
	Existe una biyecci\'{o}n natural
	\begin{align*}
		\Homalg\big(k\{X\},A\big) & \,\simeq\,
			\Hom[\Set]\big(X,\olvido A\big)
		\text{ .}
	\end{align*}
	%
	En particular,
	\begin{math}
		\Homalg\big(k\{x,y\},A\big)=A^2
	\end{math}, v\'{\i}a $f\mapsto (f(x),f(y))$.

	Un poco m\'{a}s en general,
	\begin{align*}
		\Homalg\big(k\{X\}/I,A\big) & \,\simeq\,
			\Big\{f\in\Hom[\Set]\big(X,\olvido A\big)\,:\,
				\tilde f(I)=0\Big\}
		\text{ .}
	\end{align*}
	%
	\begin{ejemploCocienteDeLibre}\label{ejemplo:cocientedelibre}
		Para el \'{a}lgebra $k[x,y]\simeq k\{x,y\}/\generado{xy-yx}$,
		\begin{align*}
			\Homalg\big(k[x,y],A\big) & \,\simeq\,
				\big\{(a,b)\in A^2\,:\,ab=ba\big\}
			\text{ .}
		\end{align*}
		%
	\end{ejemploCocienteDeLibre}
\end{frame}

\subsection{Ejemplos}

\begin{frame}{La recta y el plano afines}
	Asumimos $A$ conmutativa. Existen biyecciones
	\begin{align*}
		\Homalg\big(k[x],A\big) & \,\simeq\,A \text{ ,} \\
		\Homalg\big(k[x',x''],A\big) & \,\simeq\,A^2 \quad\text{y} \\
		\Homalg\big(k,A\big) & \,\simeq\,\{0\} \quad
			\text{(\phantom)}=\{\eta_A\}\text{\phantom()}
		\text{ .}
	\end{align*}
	%
	Queremos expresar las leyes de grupo abeliano de $A$ de manera
	universal:
	\begin{align*}
		+ \,:\,A\,\times\,A\rightarrow A & \quad\text{,}\quad
		0 \,:\,\{0\}\,\rightarrow\,A \quad\text{y}\quad
		- \,:\,A\,\rightarrow\,A
		\text{ .}
	\end{align*}
	%
\end{frame}

\begin{frame}{La recta y el plano afines (cont.)}
	\begin{propoRectaAfin}\label{propo:rectaafin}
		V\'{\i}a las biyecciones, los morfismos
		$\Delta:\,k[x]\rightarrow k[x',x'']$,
		$\varepsilon:\,k[x]\rightarrow k$ y $S:\,k[x]\rightarrow k[x]$,
		determinados por
		\begin{align*}
			\Delta(x) \,=\,x'+x'' & \quad\text{,}\quad
				\varepsilon(x)\,=\,0 \quad\text{y}\quad
				S(x)\,=\,-x
			\text{ ,}
		\end{align*}
		%
		se corresponden con $+$, $0$ y $-$, respectivamente.
	\end{propoRectaAfin}
	\begin{proof}
		$\Delta$ induce
		\begin{math}
			\pull\Delta:\,\Homalg\big(k[x',x''],A\big)\rightarrow
				\Homalg\big(k[x],A\big)
		\end{math} dada por $\pull\Delta(f)=f\circ\Delta$. Vale
		\begin{align*}
			\big(\pull\Delta f\big)(x) & \,=\,f(x'+x'')\,=\,
				f(x')+f(x'')
			\text{ .}
		\end{align*}
		%
	\end{proof}
\end{frame}

\begin{frame}{La recta y el plano afines (cont.)}
	Dados $f,g:\,k[x]\rightarrow A$, definimos la ``suma''
	$\pull\Delta(f,g):\,k[x]\rightarrow A$: si $f(x)=a$ y $g(x)=b$,
	\begin{align*}
		\pull\Delta(f,g)(x) & \,=\,a+b
		\text{ .}
	\end{align*}
	%
	La suma satisface:
	\begin{itemize}
		\item
			\begin{math}
				\pull\Delta(f,\pull\Delta(g,h))=
					\pull\Delta(\pull\Delta(f,g),h)
			\end{math};
		\item
			\begin{math}
				\pull\Delta(f,\pull\varepsilon(\eta_A))=
					\pull\Delta(\pull\varepsilon(\eta_A),f)
					=f
			\end{math};
		\item
			\begin{math}
				\pull\Delta(f,\pull S(f))=
					\pull\Delta(\pull S(f),f)=
					\pull\varepsilon(\eta_A)
			\end{math};
		\item
			\begin{math}
				\pull\Delta(f,g)=\pull\Delta(g,f)
			\end{math}.
	\end{itemize}
	%
	\begin{align*}
		\pull\Delta(f,\pull\Delta(g,h))(x) & \,=\,
			f(x)+g(x)+h(x)\,=\,
			\pull\Delta(\pull\Delta(f,g),h)(x)
		\text{ .}
	\end{align*}
	%
\end{frame}

\begin{frame}{La recta y el plano afines (cont.)}
	Todo morfismo de $k$-\'{a}lgebras $\varphi:\,A\rightarrow B$ induce
	$\push\varphi(f)=\varphi\circ f$. Esta \emph{funci\'{o}n} es morfismo
	de grupos:
	\begin{center}
		\begin{tikzcd}[column sep=small,
			ampersand replacement=\&]
			\Homalg\big(k[x',x''],A\big)
				\arrow[r,"\pull\Delta"]
				\arrow[d,"\push\varphi"'] \&
			\Homalg\big(k[x],A\big)
				\arrow[d,"\push\varphi"] \\
			\Homalg\big(k[x',x''],B\big)
				\arrow[r,"\pull\Delta"'] \&
			\Homalg\big(k[x],B\big)
		\end{tikzcd}
	\end{center}
	\begin{math}
		\pull\Delta:\,\Homalg\big(k[x',x''],-\big)\xrightarrow{\cdot}
			\Homalg\big(k[x],-\big)
	\end{math} es una transformaci\'{o}n natural. $\pull\varepsilon$ y
	$\pull S$ tambi\'{e}n.
\end{frame}

\begin{frame}{La recta y el plano afines (cont.)}
	% $\Homalg\big(k[x],-\big):\,\AlgCom[k]\rightarrow\Set$ se factoriza
	% por $\Grp$:
	Existe $G:\,\CommAlg[k]\rightarrow\Grp$ tal que
	\begin{align*}
		U\circ G & \,=\,\Homalg\big(k[x],-\big)
		\text{ .}
	\end{align*}
	%
	La funci\'{o}n $(f\mapsto f(x)):\,\Homalg\big(k[x],A\big)\rightarrow A$
	es un isomorfismo de grupos $\tau_A:\,G(A)\rightarrow (A,+)$ una t.n.:
	\begin{align*}
		\tau_B\circ\push\varphi(f) & \,=\,(\varphi\circ f)(x)\,=\,
			\varphi(f(x))\,=\,\push\varphi\circ\tau_A(f)
	\end{align*}
	%
	Si $(-,+):\,\CommAlg[k]\rightarrow\Grp$ el grupo aditivo subyacente,
	\begin{align*}
		\tau & \,:\,G\,\xrightarrow{\cdot}\,(-,+)
	\end{align*}
	%
	es un isomorfismo natural.
\end{frame}

\begin{frame}{El grupo multiplicativo}
	$A^\times:\,\CommAlg[k]\rightarrow\Grp$ el grupo multiplicativo
	subyacente.
	\begin{align*}
		\times \,:\,A^\times\,\times\,A^\times\rightarrow A^\times
			& \quad\text{,}\quad
		1 \,:\,\{1\}\,\rightarrow\,A^\times \quad\text{y}\quad
		\null^{-1}\,:\,A^\times\,\rightarrow\,A^\times
		\text{ .}
	\end{align*}
	%
	Existen biyecciones ($\tau_A:\,f\mapsto f(\bar x)$)
	\begin{align*}
		\Homalg\big(k[x,x^{-1}],A\big) & \,\simeq\,A^\times \\
		\Homalg\big(k[x',x'',x'\null^{-1},x''\null^{-1}],A\big)
			& \,\simeq\,A^\times\times A^\times
		\text{ ,}
	\end{align*}
	%
	donde
	\begin{align*}
		k[x,x^{-1}] & \,:=\,k[x,y]/\generado{xy-1} \text{ ,} \\
		k[x',x'',x'\null^{-1},x''\null^{-1}] & \,:=\,
			k[x',y',x'',y'']/\generado{x'y'-1,x''y''-1}
		\text{ .}
	\end{align*}
	%
\end{frame}

\begin{frame}{El grupo multiplicativo (cont.)}
	\begin{propoMultiplicativo}\label{propo:multiplicativo}
		Los morfismos
		\begin{math}
			\Delta:\,k[x,x^{-1}]\rightarrow
				k[x',x'',x'\null^{-1},x''\null^{-1}]
		\end{math},
		\begin{math}
			\varepsilon:\,k[x,x^{-1}]\rightarrow k
		\end{math} y
		\begin{math}
			S:\,k[x,x^{-1}]\rightarrow k[x,x^{-1}]
		\end{math} determinados por
		\begin{align*}
			\Delta(x) \,=\,x'x'' & \quad\text{,}\quad
				\varepsilon(x)\,=\,1 \quad\text{y}\quad
				S(x)\,=\,x^{-1}
		\end{align*}
		%
		se corresponden con $\times$, $1$ y $\null^{-1}$.
	\end{propoMultiplicativo}
	\begin{coroMultiplicativo}\label{coro:multiplicativo}
		$\pull\Delta,\pull\varepsilon,\pull S$ son t.n. de funtores de
		tipo $\CommAlg[k]\rightarrow\Set$ y existe
		$G:\,\CommAlg[k]\rightarrow\Grp$ tal que
		\begin{align*}
			U\circ G & \,=\,\Homalg\big(k[x,x^{-1}],-\big)
			\text{ .}
		\end{align*}
		%
		Las funciones $\tau_A$ inducen un isomorfismo natural
		$\tau:\,G\xrightarrow{\cdot}(-,\times)$.
	\end{coroMultiplicativo}
\end{frame}

\begin{frame}{Producto de matrices}
	Sea $\MM(2)=k[a,b,c,d]$.
	\begin{math}
		f\mapsto f\big(
			\left[\begin{smallmatrix}
				a & b \\ c & d
			\end{smallmatrix}\right]\big)=
			\left[\begin{smallmatrix}
				f(a) & f(b) \\ f(c) & f(d)
			\end{smallmatrix}\right]
	\end{math} induce
	\begin{align*}
		\Homalg\big(\MM(2),A\big) & \,\simeq\,A^4\,=\,
			\MM[2\times 2](A)
		\text{ .}
	\end{align*}
	%
	Duplicamos las variables:
	$\MM(2)^{\otimes 2}=k[a',b',c',d',a'',b'',c'',d'']$ y buscamos
	$\Delta:\,\MM(2)\rightarrow\MM(2)^{\otimes 2}$ tal que
	\begin{center}
		\begin{tikzcd}[column sep=small,ampersand replacement=\&]
			\Homalg\big(\MM(2)^{\otimes 2},A\big)
				\arrow[r,"\sim"]
				\arrow[d,"\pull\Delta"'] \&
			\MM[2\times 2](A)^2 \arrow[d,"\cdot"] \\
			\Homalg\big(\MM(2),A\big)
				\arrow[r,"\sim"] \&
			\MM[2\times 2](A)
		\end{tikzcd}
	\end{center}
	conmute.
\end{frame}

\begin{frame}{Producto de matrices (cont.)}
	Debe cumplirse
	\begin{align*}
		f\circ\Delta\Big(
			\begin{bmatrix} a & b \\ c & d \end{bmatrix}
				\Big) & \,=\,
			f\Big(\begin{bmatrix} a' & b' \\ c' & d' \end{bmatrix}
				\Big)\,f\Big(
				\begin{bmatrix}
					a'' & b'' \\ c'' & d''
				\end{bmatrix}
				\Big) \\
		& \,=\, f\Big(\begin{bmatrix} a' & b' \\ c' & d' \end{bmatrix}
			\,\begin{bmatrix} a'' & b'' \\ c'' & d'' \end{bmatrix}
			\Big)
		\text{ .}
	\end{align*}
	%
	\begin{propoProductoDeMatrices}\label{propo:productodematrices}
		Si $\Delta:\,\MM(2)\rightarrow\MM(2)^{\otimes 2}$ es el
		morfismo determinado por
		% de \'{a}lgebras
		\begin{align*}
			\Delta\,\begin{bmatrix} a & b \\ c & d \end{bmatrix}
				& \,=\,
				\begin{bmatrix}
					a' & b' \\ c' & d'
				\end{bmatrix}\,
				\begin{bmatrix}
					a'' & b'' \\ c'' & d''
				\end{bmatrix}
			\text{ ,}
		\end{align*}
		%
		el diagrama conmuta.
		\begin{math}
			\Delta(ad-bc)=(a'd'-b'c')\,(a''d''-b''c'')
		\end{math}.
	\end{propoProductoDeMatrices}
\end{frame}

\begin{frame}{$\GL(2)$ y $\SL(2)$}
	\begin{align*}
		\GL(2) & \,=\,\MM(2)[t]/\generado{(ad-bc)\,t-1} \\
		\SL(2) & \,=\,\GL(2)/\generado{t-1} \,=\,
			\MM(2)/\generado{ad-bc-1}
		\text{ .}
	\end{align*}
	%
	Dada una $k$-\'{a}lgebra conmutativa $A$, existen biyecciones
	\begin{align*}
		\Homalg\big(\GL(2),A\big) & \,\simeq\,
			\GL[2](A) \quad\text{y} \\
		\Homalg\big(\SL(2),A\big) & \,\simeq\,
			\SL[2](A)
		\text{ ,}
	\end{align*}
	%
	pues, si
	\begin{math}
		\left[\begin{smallmatrix}
			\alpha & \beta \\ \gamma & \delta
		\end{smallmatrix}\right]\in\GL[2](A)
	\end{math}, existe \'{u}nico $f:\,\MM(2)[t]\rightarrow A$ tal que
	\begin{align*}
		f\,
		\left[\begin{matrix}
			a & b \\ c & d
		\end{matrix}\right]\,=\,
		\left[\begin{matrix}
			\alpha & \beta \\ \gamma & \delta
		\end{matrix}\right] & \quad\text{y}\quad
		f(t)\,=\,(\alpha\delta-\beta\gamma)^{-1}
		\text{ .}
	\end{align*}
	%
\end{frame}

\begin{frame}{$\GL(2)$ y $\SL(2)$ (cont.)}
	Queremos
	\begin{align*}
		\Delta & \,:\, \GL(2)\,\rightarrow\,\GL(2)^{\otimes 2}
			\,=\,\MM(2)^{\otimes 2}[t',t'']/I
		\text{ ,}
	\end{align*}
	%
	donde $I=\generado{(a'd'-b'c')\,t'-1,(a''d''-b''c'')\,t''-1}$.
	Definimos $\Delta:\,\MM(2)[t]\rightarrow\MM(2)^{\otimes 2}[t',t'']$,
	extendiendo por
	\begin{align*}
		\Delta(t) & \,=\,t'\,t''
		\text{ .}
	\end{align*}
	%
	Se cumple $\Delta((ad-bc)\,t-1)=0$ en $\GL(2)^{\otimes 2}$.
\end{frame}

\begin{frame}{$\GL(2)$ y $\SL(2)$ (cont.)}
	\begin{propoLinealGeneral}\label{propo:linealgeneral}
		$\Delta$, junto con $\varepsilon:\,\GL(2)\rightarrow k$ y
		$S:\,\GL(2)\rightarrow\GL(2)$ dados por
		\begin{align*}
			\varepsilon\,
				\begin{bmatrix} a & b \\ c & d \end{bmatrix}
				\,=\,\begin{bmatrix} 1 & \\ & 1 \end{bmatrix}
				& \quad\text{,}\quad
				\varepsilon(t)\,=\,1 \text{ ,} \\
			S\,\begin{bmatrix} a & b \\ c & d \end{bmatrix}
				\,=\,(ad-bc)^{-1}\,
				\begin{bmatrix} d & -b \\ -c & a \end{bmatrix}
				& \quad\text{,}\quad
				S(t) \,=\,t^{-1}
			\text{ ,}
		\end{align*}
		%
		se corresponden con el producto, la identidad y el inverso.
	\end{propoLinealGeneral}
\end{frame}

\subsection{Producto tensorial}

\begin{frame}{Producto tensorial de \'{a}lgebras}
	\begin{propoProductoTensorial}\label{propo:productotensorial}
		Sean $A,B$ $k$-\'{a}lgebras y sea $A\tensor[k]B$ el $k$-%
		m\'{o}dulo con producto
		\begin{align*}
			(a\tensor b)\,(a_1\tensor b_1) & \,=\,
				aa_1\tensor bb_1
			\text{ .}
		\end{align*}
		%
		$A\tensor[k]B$ es $k$-\'{a}lgebra y
		\begin{align*}
			\Homalg\big(A\tensor[k]B,C\big) & \,\simeq\,
				\Homalg\big(A,C\big)\,\times\,
				\Homalg\big(B,C\big)
			\text{ ,}
		\end{align*}
		%
		para toda \'{a}lgebra conmutativa $C$, dada por
		\begin{align*}
			& (f,g) \,\mapsto\, \big(\mu_C\circ(f\tensor g)\,:\,
				(a\tensor b)\mapsto f(a)\,g(b)\big)
			\text{ .}
		\end{align*}
		%
	\end{propoProductoTensorial}
\end{frame}

\begin{frame}{Relaci\'{o}n con $\Delta$}
	\begin{propoProductoDeCociente}\label{propo:productodecociente}
		Sea $A=k\{X\}/I$. Sean $X',X''$ copias de $X$ y sean
		$I'\triangleleft k\{X'\}$ e $I''\triangleleft k\{X''\}$ los
		ideales correspondientes a $I$. Entonces
		\begin{align*}
			A\tensor[k]A & \,\simeq\, A^{\otimes 2}\,:=\,
				k\{X'\sqcup X''\}/\generado{I',I'',X'X''-X''X'}
			\text{ ,}
		\end{align*}
		%
		v\'{\i}a $x'\mapsto x\tensor 1$ y $x''\mapsto 1\tensor x$.
	\end{propoProductoDeCociente}
	Por ejemplo, $k[x',x'']\simeq k[x]\tensor k[x]$.
	\begin{obsProductoDeMatrices}\label{obs:productodematrices}
		El ``producto de matrices'' $\Delta$ est\'{a} caracterizado por
		\begin{align*}
			\Delta\,\begin{bmatrix} a & b \\ c & d \end{bmatrix}
				& \,=\,
			\begin{bmatrix} a & b \\ c & d \end{bmatrix}\tensor
			\begin{bmatrix} a & b \\ c & d \end{bmatrix}
			\text{ .}
		\end{align*}
		%
		% y cumple $\Delta(ad-bc)=(ad-bc)\tensor(ad-bc)$.
	\end{obsProductoDeMatrices}
\end{frame}

\begin{frame}{Relaci\'{o}n con $\Delta$ (cont.)}
	\begin{obsRectaAfinAbeliana}\label{obs:rectaafinabeliana}
		$\Homalg\big(k[x],A\big)$ es abeliana:
		\begin{align*}
			\pull\Delta(f,g)(x) & \,=\,
				\mu_A(f\tensor g) (x\tensor 1 + 1\tensor x)
				\,=\, f(x) + g(x) \\
			\pull\Delta(g,f)(x) & \,=\, g(x) + f(x)
			\text{ ,}
		\end{align*}
		%
		$\Homalg\big(\GL(2),A\big)$, no:
		\begin{align*}
			\pull\Delta(f,g)(a) & \,=\,
				\mu_A\circ(f\tensor g)
					(a\tensor a+b\tensor c)
				\,=\,f(a)\,g(a)+f(b)\,g(c) \\
			\pull\Delta(g,f)(a) & \,=\,g(a)\,f(a)+g(b)\,f(c)
				\text{ .}
		\end{align*}
		%
	\end{obsRectaAfinAbeliana}
\end{frame}
