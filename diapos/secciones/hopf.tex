\theoremstyle{plain}
\newtheorem{defAntipoda}{Definici\'{o}n}[section]
\newtheorem{obsHopf}[defAntipoda]{Observaci\'{o}n}
\newtheorem{teoGrupoDeMorfismos}[defAntipoda]{Teorema}

%-------------

\subsection{La ant\'{\i}poda}

\begin{frame}{Convoluci\'{o}n y ant\'{\i}poda}
	Sean $(A,\mu,\eta)$, $(C,\Delta,\varepsilon)$. La
	\emph{convoluci\'{o}n} de $f,g\in\Hom[k](C,A)$, es
	\begin{align*}
		f \convol g & \,:=\,\mu\circ(f\tensor g)\circ\Delta
			\,\in\,\Hom[k](C,A)
		\text{ .}
	\end{align*}
	%
	Si $\Delta(x)=\sum_i\,x_i'\tensor x_i''$,
	$f \convol g(x) =\sum_i\,f(x_i')\,g(x_i'')$
	\begin{defAntipoda}\label{def:antipoda}
		Una \emph{ant\'{\i}poda} en $(H,\mu,\eta,\Delta,\varepsilon)$
		es $S\in\Endo[k](H)$ tal que
		\begin{align*}
			S\convol\id[H] & \,=\,\id[H]\convol S \,=\,
				\eta\circ\varepsilon
			\text{ .}
		\end{align*}
		%
		\begin{math}
			\sum_i\,x_i'S(x_i'')=\varepsilon(x)\cdot 1=
				\sum_i\,S(x_i')x_i''
		\end{math}. Si existe, es \'{u}nica.
	\end{defAntipoda}
\end{frame}

\begin{frame}{Definici\'{o}n}
	Un \emph{\'{a}lgebra de Hopf} es una bi\'{a}lgebra $H$ con
	ant\'{\i}poda. Un morfismo de \'{a}lgebras de Hopf es un morfismo de
	bi\'{a}lgebras.
	\begin{obsHopf}\label{obs:hopf}
		\begin{math}
			S:\,H\rightarrow H^{\opp\,\copp}=
				(H,\mu^\opp,\eta,\Delta^\opp,\varepsilon)
		\end{math} es morfismo de bi\'{a}lgebras.
		Si $H=k\{X\}/I$ es bi\'{a}lgebra, dado un morfismo de
		\'{a}lgebras $S:\,H\rightarrow H^\opp$, basta verificar la
		condici\'{o}n de ant\'{\i}poda en $X$.
	\end{obsHopf}
\end{frame}

\begin{frame}{El \'{a}lgebra de un grupo}
	Sean $G$ un grupo, $k[G]$ la bi\'{a}lgebra del monoide.
	\begin{align*}
		S(g) & \,=\,g^{-1}
	\end{align*}
	%
	$g\in G$, define una ant\'{\i}poda: $\Delta(g)=g\tensor g$ y
	$\varepsilon(g)=1$. Rec\'{\i}procamente, si $G$ es monoide y
	$S:\,k[G]\rightarrow k[G]$ es ant\'{\i}poda,
	\begin{align*}
		g\,S(g) & \,=\,S(g)\,g \,=\,\varepsilon(g)\,1\,=\,1
	\end{align*}
	%
	implica $S(g)\in G$ y es inverso de $g$.
\end{frame}

\subsection{Ejemplos}

\begin{frame}{$\GL(2)$ y $\SL(2)$}
	$\GL(2)$ y $\SL(2)$ son bi\'{a}lgebras conmutativas con
	$\Delta,\varepsilon$ dados por
	\begin{align*}
		\Delta\,\begin{bmatrix} a & b \\ c & d \end{bmatrix} \,=\,
			\begin{bmatrix} a & b \\ c & d \end{bmatrix} \tensor
			\begin{bmatrix} a & b \\ c & d \end{bmatrix}
			& \quad\text{,}\quad
			\Delta(t)\,=\,t\tensor t \text{ ,} \\
		\varepsilon\,\begin{bmatrix} a & b \\ c & d \end{bmatrix} \,=\,
			\begin{bmatrix} 1 & \\ & 1 \end{bmatrix}
			& \quad\text{,}\quad
			\varepsilon(t) \,=\,1
		\text{ .}
	\end{align*}
	%
	$\Delta$ no es coconmutativa:
	\begin{align*}
		\Delta(a) & \,=\,a\tensor a+b\tensor c\,\not=\,
			a\tensor a+c\tensor b \,=\,\swap\circ\Delta(a)
		\text{ .}
	\end{align*}
	%
	La ant\'{\i}poda est\'{a} dada por
	\begin{align*}
		S\,\begin{bmatrix} a & b \\ c & d \end{bmatrix} \,=\,
			(ad-bc)^{-1}\,
			\begin{bmatrix} d & -b \\ -c & a \end{bmatrix}
			& \quad\text{,}\quad
			S(t) \,=\,t^{-1}
		\text{ .}
	\end{align*}
	%
\end{frame}

\begin{frame}{El grupo de un \'{a}lgebra}
	Dada $(H,\Delta,\varepsilon)$,
	\begin{align*}
		\grouplike H & \,:=\,\big\{x\in H\,:\,x\not=0,\,
			\Delta(x)=x\tensor x\big\}
		\text{ .}
	\end{align*}
	%
	Si $H$ es bi\'{a}lgebra, es monoide con unidad $\Delta(1)=1\tensor 1$ y
	\begin{align*}
		\Delta(xy) & \,=\,\Delta(x)\,\Delta(y)\,=\,
			(x\tensor x)\,(y\tensor y) \,=\,xy\tensor xy
		\text{ .}
	\end{align*}
	%
	Si $H$ es de Hopf, $x\mapsto S(x)$ define un inverso en
	$\grouplike H$:
	\begin{align*}
		\swap\circ(S\tensor S)\circ\Delta & \,=\,
			\Delta\circ S
		\text{ .}
	\end{align*}
	%
	Si $H=k[G]$, entonces $\grouplike{k[G]}=G$.
\end{frame}

\begin{frame}{El grupo $\Homalg\big(H,A\big)$ (cont.)}
	\begin{teoGrupoDeMorfismos}\label{thm:grupodemorfismos}
		Sean $H$ un \'{a}lgebra de Hopf y $A$ un \'{a}lgebra
		conmutativa. Los conjuntos $\Homalg\big(H,A\big)$ son grupos
		con la convoluci\'{o}n heredada de $\Hom[k](H,A)$. El inverso
		de $\psi:\,H\rightarrow A$ est\'{a} dado por $\psi\circ S$.
	\end{teoGrupoDeMorfismos}
	Comprobar que
	\begin{itemize}
		% \item
			% \begin{math}
				% (\psi\convol\varphi)\circ\mu_H =
				% \mu_A\circ((\psi\convol\varphi)\tensor
					% (\psi\convol\varphi))
			% \end{math},
		% \item
			% \begin{math}
				% (\psi\convol\varphi)\circ\eta_H =\eta_A
			% \end{math},
		\item $\psi\convol\varphi\in\Homalg\big(H,A\big)$,
		\item $c =\eta_A\circ\varepsilon_H\in\Homalg\big(H,A\big)$ es
			unidad,
		\item $\psi\circ S\in\Homalg\big(H,A\big)$ es inverso.
	\end{itemize}
	%
\end{frame}

\begin{frame}{El grupo $\Homalg\big(H,A\big)$ (cont.)}
	Si $\psi,\varphi\in\Homalg\big(H,A\big)$, como $H$ es bi\'{a}lgebra,
	\begin{align*}
		(\psi\convol\varphi)\circ\mu_H & \,=\
			\mu_A\circ (\psi\tensor\varphi)\circ\Delta_H\circ\mu_H
				\\
		& \,=\,	\mu_A\,(\psi\tensor\varphi)\,(\mu_H\tensor\mu_H)\,
			(\id[H]\tensor\swap[H]\tensor\id[H])\,
			(\Delta_H\tensor\Delta_H) \\
		& \,=\,\mu_A\,((\mu_A\,(\psi\tensor\varphi)\,\Delta_H)\tensor
			(\mu_A\,(\psi\tensor\varphi)\,\Delta_H)) \\
		& \,=\,\mu_A((\psi\convol\varphi)\tensor(\psi\convol\varphi))
		\text{ .}
	\end{align*}
	%
	Si $c=\eta_A\circ\varepsilon_H$,
	\begin{align*}
		\psi\convol c & \,=\,\mu_A\,(\psi\tensor\eta_A\varepsilon_H)\,
								\Delta_H \\
		& \,=\,\mu_A\,(\id[A]\tensor\eta_A)\,(\psi\tensor\id[k])\,
			(\id[H]\tensor\varepsilon_H)\,\Delta_H \\
		& \,=\,\psi\tensor\id[k]\,=\,\psi
		\text{ .}
	\end{align*}
	%
\end{frame}

\begin{frame}{El grupo $\Homalg\big(H,A\big)$ (cont.)}
	$G=(\Homalg\big(H,A\big),\convol,\eta_A\circ\varepsilon_H)$ es un
	monoide y podemos definir $(k[G],\mu,\eta,\Delta,\varepsilon)$. Sea
	$S_H$ la ant\'{\i}poda en $H$ y sea
	$S(\psi)=\psi\circ S_H\in\Hom[k](H,A)$. Se verifica que
	\begin{align*}
		\mu\circ(\id\tensor S)\circ\Delta & \,=\,
			\eta\circ\varepsilon\,=\,
			\mu\circ(S\tensor\id)\circ\Delta
	\end{align*}
	%
	Por ejemplo, $\eta\circ\varepsilon(\psi)=\eta(1)=\eta_A\varepsilon_H$ y
	\begin{align*}
		\mu\circ(\id\tensor S)\circ\Delta(\psi) & \,=\,
			\mu(\psi\tensor S(\psi)) \,=\,\psi\convol S(\psi) \\
		& \,=\, \mu_A\,(\psi\tensor\psi)\,(\id[H]\tensor S_H)\,
								\Delta_H \\
		& \,=\,\psi\,\mu_H\,(\id[H]\tensor S_H)\,\Delta_H \,=\,
			(\psi\,\eta_H)\,\varepsilon_H \\
		& \,=\,\eta_A\varepsilon_H
		\text{ .}
	\end{align*}
	%
\end{frame}

\begin{frame}{El grupo $\Homalg\big(H,A\big)$ (cont.)}
	Resta verificar que $S(\psi)\in\Homalg\big(H,A\big)$:
	\begin{align*}
		(\psi\,S_H)\,\mu_H & \,=\,\psi\,(\mu_H\,\swap[H])\,
			(S_H\tensor S_H) \,=\,
			\mu_A\,(\psi\tensor\psi)\,\swap[H]\,(S_H\tensor S_H) \\
		& \,=\,(\mu_A\,\swap[A])\,((\psi\,S_H)\tensor (\psi\,S_H))
	\end{align*}
	%
	Si $\mu_A\circ\swap[A]=\mu_A$, entonces $\psi\circ S_H$ respeta
	productos. En cuanto a la unidad,
	\begin{align*}
		(\psi\,S_H)\,\eta_H & \,=\,\psi\,\eta_H\,=\,\eta_H
		\text{ .}
	\end{align*}
\end{frame}

