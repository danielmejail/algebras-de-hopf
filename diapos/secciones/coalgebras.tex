\theoremstyle{plain}
\newtheorem{obsCoconmutativa}{Observaci\'{o}n}[section]
\newtheorem{ejemploCoalgebraProductoTensorial}[obsCoconmutativa]{Ejemplo}
\newtheorem{obsPolinomios}[obsCoconmutativa]{Observaci\'{o}n}
\newtheorem{obsBialgebraDeMatrices}[obsCoconmutativa]{Observaci\'{o}n}

%-------------

\subsection{Co\'{a}lgebras}
\begin{frame}{Definici\'{o}n}
	Una $k$-co\'{a}lgebra es un $k$-m\'{o}dulo $C$ y morfismos
	$\Delta:\,C\rightarrow C\tensor[k]C$ y $\varepsilon:\,C\rightarrow k$
	tales que los diagramas
	\begin{center}
		\begin{tikzcd}[column sep=small,ampersand replacement=\&]
			C\tensor C\tensor C \&
				C\tensor C\arrow[l,"\Delta\tensor\id"'] \\
			C\tensor C\arrow[u,"\id\tensor\Delta"] \&
				C \arrow[l,"\Delta"]\arrow[u,"\Delta"']
		\end{tikzcd}
		\begin{tikzcd}[column sep=small,ampersand replacement=\&]
			k\tensor C \& C\tensor C
				\arrow[l,"\varepsilon\tensor\id"']
				\arrow[r,"\id\tensor\varepsilon"]
				\& C\tensor k \\
			\& C
				\arrow[ur,"\simeq"']
				\arrow[u,"\Delta"']
				\arrow[ul,"\simeq"]
				\&
		\end{tikzcd}
	\end{center}
	conmutan. $(C,\Delta,\varepsilon)$ es coconmutativa, si
	$\Delta\circ\swap=\Delta$, donde $\swap(c\tensor c')=c'\tensor c$.
	$f:\,C\rightarrow D$ es morfismo de co\'{a}lgebras, si
	\begin{align*}
		\Delta_D\circ f & \,=\,(f\tensor f)\circ\Delta_C
			\quad\text{y} \\
		\varepsilon_D\circ f & \,=\,\varepsilon_C
	\end{align*}
	%
\end{frame}

\begin{frame}{Ejemplos}
	\begin{itemize}
		\item $(k,\Delta,\varepsilon)$ con $\Delta(1)=1\tensor 1$ y
			$\varepsilon(1)=1$;
		\item en el $k$-m\'{o}dulo $k[x]$ (polinomios),
			\begin{align*}
				\Delta(x) \,=\,x\tensor 1 + 1\tensor x
					& \quad\text{y}\quad
				\varepsilon(x)\,=\,0
				\text{ ;}
			\end{align*}
			%
		% \item en $k[x_1,\,\dots,\,x_n]$,
			% $\Delta(x_i)=x_i\tensor 1 + 1\tensor x_i$,
			% $\varepsilon(x_i)=0$;
		\item dado un conjunto $G$, en el $k$-m\'{o}dulo libre
			$k[G]$ con base $G$
			\begin{align*}
				\Delta(g) \,=\, g\tensor g
					& \quad\text{y}\quad
				\varepsilon(g)\,=\,1
				\text{ ;}
			\end{align*}
			%
		\item dada $(C,\Delta,\varepsilon)$,
			$C^\copp=(C,\Delta^\opp,\varepsilon)$, con
			$\Delta^\opp=\swap\circ\Delta$.
	\end{itemize}
	%
	\begin{obsCoconmutativa}\label{obs:coconmutativa}
		$k[G]^\copp=k[G]$.
	\end{obsCoconmutativa}
\end{frame}

\begin{frame}{Co\'{a}lgebra de matrices}
	$A=\MM[m\times m](k)$ con base $\{E_{ij}\}_{ij}$. Sea $\{x_{ij}\}_{ij}$
	la base dual en $\dual A$. Los morfismos de $k$-m\'{o}dulos
	determinados por
	\begin{align*}
		\Delta(x_{ij}) \,=\,\sum_{k=1}^{m}\,x_{ik}\tensor x_{kj}
			& \quad\text{y}\quad
		\varepsilon(x_{ij}) \,=\,\delta_{ij}
	\end{align*}
	%
	definen una co\'{a}lgebra $(\dual A,\Delta,\varepsilon)$:
	% ${\dual A}^{\copp}\not=\dual A$
	\begin{align*}
		(\id\tensor\Delta)\circ\Delta(x_{ij}) & \,=\,
			\sum_{k=1}^{m}\,x_{ik}\tensor\Delta(x_{kj}) \,=\,
			\sum_{k=1}^{m}\,\sum_{l=1}^{m}\,
				x_{ik}\tensor x_{kl}\tensor x_{lj} \\
		& \,=\, \sum_{l=1}^{m}\,\Delta(x_{il})\tensor x_{lj} \,=\,
			(\Delta\tensor\id)\circ\Delta(x_{ij})
		\text{ .}
	\end{align*}
	%
	\begin{obsCoconmutativa}\label{obs:nococonmutativa}
		${\dual A}^\copp\not=\dual A$.
	\end{obsCoconmutativa}
\end{frame}

\begin{frame}{Producto tensorial de co\'{a}lgebras}
	Dadas co\'{a}lgebras $(C,\Delta_C,\varepsilon_C)$,
	$(D,\Delta_D,\varepsilon_D)$, el $k$-m\'{o}dulo $C\tensor[k]D$ es
	co\'{a}lgebra con
	\begin{align*}
		\Delta & \,:=\,(\id[C]\tensor\swap\tensor\id[D])\circ
			(\Delta_C\tensor\Delta_D) \quad\text{y} \\
		\varepsilon & \,:=\,\varepsilon_C\tensor\varepsilon_D
	\end{align*}
	%
	($\tau(c\tensor d)=d\tensor c$).
	\begin{ejemploCoalgebraProductoTensorial}%
		\label{ejemplo:coalgebraproductotensorial}
		Dados conjuntos $G$ e $H$, el $k$-isomorfismo
		\begin{align*}
			k[G]\tensor[k]k[H] & \,\simeq\, k[G\times H]
			\text{ ,}
		\end{align*}
		%
		dado por $(g,h)\mapsto g\tensor h$, es isomorfismo de
		co\'{a}lgebras.
	\end{ejemploCoalgebraProductoTensorial}
\end{frame}

\subsection{Bi\'{a}lgebras}

\begin{frame}{La bi\'{a}lgebra $k$}
	Sobre el anillo $k$ tenemos $(k,\mu,\eta)$ y $(k,\Delta,\varepsilon)$.
	Son ``compatibles'': por ejemplo, evaluando en $1\tensor 1$,
	\begin{align*}
		\Delta\circ\mu & \,=\,
			(\mu\tensor\mu)\circ (\id\tensor\swap\tensor\id)\circ
				(\Delta\tensor\Delta)
				% \text{ ,} \\
		\text{ .}
	\end{align*}
	%
	$\mu:\,k\tensor k\rightarrow k$ y $\eta:\,k\rightarrow k$ son morfismos
	de co\'{a}lgebras; $\Delta:\,k\rightarrow k\tensor k$ y
	$\varepsilon:\,k\rightarrow k$ son morfismos de \'{a}lgebras.
	% \begin{align*}
		% \mu\circ (\varepsilon\tensor\varepsilon) & \,=\,
			% \varepsilon\circ\mu \,=\,\varepsilon\tensor\varepsilon
			% \text{ ,} \\
		% (\eta\tensor\eta)\circ\Delta & \,=\,\Delta\circ\eta \,=\,
			% \eta\tensor\eta	\quad\text{y} \\
		% \eta & \,=\,\varepsilon\circ\eta\,=\,\varepsilon
		% \text{ .}
	% \end{align*}
	% %
\end{frame}

\begin{frame}{Definici\'{o}n}
	Una bi\'{a}lgebra es un $k$-m\'{o}dulo $B$ con estructuras
	$(B,\mu,\eta)$ y $(B,\Delta,\varepsilon)$ tales que
	$\mu,\eta$ son morfismos de co\'{a}lgebras (equivalentemente,
	$\Delta,\varepsilon$ son morfismos de \'{a}lgebras):
	\begin{align*}
		\Delta\circ\mu & \,=\,
			(\mu\tensor\mu)\circ (\id\tensor\swap\tensor\id)\circ
				(\Delta\tensor\Delta)
				\text{ ,} \\
		\mu_k\circ (\varepsilon\tensor\varepsilon) & \,=\,
			\varepsilon\circ\mu \,=\,\varepsilon\tensor\varepsilon
			\text{ ,} \\
		(\eta\tensor\eta)\circ\Delta_k & \,=\,\Delta\circ\eta \,=\,
			\eta\tensor\eta	\quad\text{y} \\
		\eta_k & \,=\,\varepsilon\circ\eta\,=\,\varepsilon_k
		\text{ .}
	\end{align*}
	Un morfismo de bi\'{a}lgebras es un morfismo $f:\,B\rightarrow B'$ de
	\'{a}lgebras y co\'{a}lgebras:
	\begin{align*}
		\Delta_{B'}\circ f \,=\,(f\tensor f)\circ\Delta_B
			& \quad\text{,}\quad
			\varepsilon_{B'}\circ f \,=\,\varepsilon_B
			\text{ ,} \\
		f\circ\mu_B \,=\,\mu_{B'}\circ (f\tensor f)
			& \quad\text{y}\quad
			f\circ\eta_B \,=\,\eta_{B'}
		\text{ .}
	\end{align*}
	%
\end{frame}

\begin{frame}{Polinomios}
	En el \'{a}lgebra $k[x]$,
	\begin{align*}
		\Delta(x) \,=\,x\tensor 1 + 1\tensor x
			& \quad\text{y}\quad \varepsilon(x)\,=\,0
	\end{align*}
	%
	determinan morfismos de \'{a}lgebras. $(k[x],\Delta,\varepsilon)$ es
	co\'{a}lgebra:
	\begin{align*}
		(\Delta\tensor\id)\circ\Delta(x) & \,=\,
			(\Delta\tensor\id)(x\tensor 1 + 1\tensor x) \\
		& \,=\,(x\tensor 1 + 1\tensor x)\tensor 1 + 1\tensor 1\tensor x
			\text{ ,} \\
		(\id\tensor\Delta)\circ\Delta(x) & \,=\,
			(\id\tensor\Delta)(x\tensor 1 + 1\tensor x) \\
		& \,=\,x\tensor 1\tensor 1 + 1\tensor (x\tensor 1 + 1\tensor x)
		\text{ .}
	\end{align*}
	%
\end{frame}

\begin{frame}{Polinomios (cont.)}
	En $k[x_1,\,\dots,\,x_n]$,
	$\Delta(x_i)=x_i\tensor 1 + 1\tensor x_i$, $\varepsilon(x_i)=0$.
	$k[x_1,\,\dots,\,x_n]^\copp=k[x_1,\,\dots,\,x_n]$.

	\begin{obsPolinomios}\label{obs:polinomios}
		El isomorfismo de \'{a}lgebras
		$k[x',x'']\simeq k[x]\tensor k[x]$ dado por
		$\phi(x')\mapsto x\tensor 1$, $\phi(x'')\mapsto 1\tensor x$ es
		isomorfismo de co\'{a}lgebras:
		\begin{align*}
			(\phi\tensor\phi)\circ\Delta(x') & \,=\,
				\phi(x')\tensor\phi(1) + \phi(1)\tensor\phi(x')
				\\
			& \,=\,x\tensor 1\tensor 1\tensor 1 +
				1\tensor 1\tensor x\tensor 1 \\
			\Delta\circ\phi(x') & \,=\,
				(\id\tensor\swap\tensor\id)\circ
				(\Delta\tensor\Delta)(x\tensor 1) \\
			& \,=\,(\id\tensor\swap\tensor\id)(
				(x\tensor 1 +1\tensor x)\tensor 1\tensor 1)
			\text{ .}
		\end{align*}
		%
	\end{obsPolinomios}
\end{frame}

\begin{frame}{Bi\'{a}lgebra de matrices}
	En el \'{a}lgebra de polinomios $\MM(m)=k[x_{11},\,\cdots,\,x_{mm}]$,
	\begin{align*}
		\Delta(x_{ij}) \,=\,\sum_{k=1}^{m}\,
			x_{ik}\tensor x_{kj} & \quad\text{y}\quad
		\varepsilon(x_{ij}) \,=\,\delta_{ij}
	\end{align*}
	%
	determinan morfismos de \'{a}lgebras. Pero
	$(\MM(m),\Delta,\varepsilon)$ es co\'{a}lgebra (an\'{a}logo a
	co\'{a}lgebra de matrices).
	\begin{obsBialgebraDeMatrices}\label{obs:bialgebradematrices}
		Como co\'{a}lgebras
		$\MM(m)\not\simeq k[x_{11},\,\dots,\,x_{mm}]$
	\end{obsBialgebraDeMatrices}
\end{frame}

\begin{frame}{Bi\'{a}lgebra de un monoide}
	Sea $G$ un monoide con producto $\mu:\,G\times G\rightarrow G$ y unidad
	$e\in G$, $(k[G],\Delta,\varepsilon)$ la co\'{a}lgebra del conjunto.
	$(k[G],\mu,e)$ es \'{a}lgebra y $\mu$ y $1\mapsto e$ son morfismos de
	co\'{a}lgebras:
	\begin{align*}
		\Delta\circ\mu(x,y) & \,=\,\mu(x,y)\tensor\mu(x,y) \,=\,
			\mu_{\tensor}\big((x\tensor x),(y\tensor y)\big) \\
		& \,=\,\mu(\Delta(x),\Delta(y)) \quad\text{y} \\
		\varepsilon\circ\mu(x,y) & \,=\,1 \,=\,
			\mu(\varepsilon(x),\varepsilon(y))
		\text{ .}
	\end{align*}
	%
\end{frame}
